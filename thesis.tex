% Streamovanie je realizované pomocou Faraday::UploadIO komponentu aplikácie.
% popisat use case
% k pravidlam syntaxe popis alebo odstranit cislovane odkazy
% uvod do cassandry
% named windows
% názorné priklady - 3 sposoby vkladania dat - zo suboru / handleru / webova aplikacia
% odstranit duplicitny priklad schemy
% obrazok z nosql do cassandry
% obrazok z esper do views?

%= plugin pre zobrazenie schemy / xml
%= vytvorenie databazy
%= statement = vyraz, pattern - vzor, observer
%= rest api vracia metadata (strankovanie, schema...)
%= insert into clause
%= std:lastevent, win:length...
%= UUID, rozdiel v cassandra DAO
%= vytvorenie cassandra tabulky a priklad vyhladania - preco takyto primary key
%= cassandra join, order, count
%= komplexna schema
%= priklad udalosti
%= klauzuly select from ...
%= priklad objektoveho modelu


%\documentclass[12pt, a4paper]{book}
\documentclass[12pt, a4paper, oneside]{book} %isis

% ----- Nastavenie pisma -----
\usepackage[slovak]{babel}
%\usepackage[T1]{fontenc}
\usepackage[IL2]{fontenc}
\usepackage[utf8]{inputenc}

% ----- Vkladanie obrazkov -----
\usepackage{color, graphicx}
\graphicspath{{./images/}}

% ----- URL -----
\usepackage{url}
\usepackage[unicode]{hyperref} %odkazy vnutri dokumentu

% ----- Odd Even page -----
\usepackage{changepage}
\strictpagecheck

% ----- Zdrojové kódy -----
\usepackage{listings}
% info - http://en.wikibooks.org/wiki/LaTeX/Packages/Listings
% comments - http://lenaherrmann.net/2010/05/20/javascript-syntax-highlighting-in-the-latex-listings-package
\lstset{
%  language=bash,                  % the language of the code
  basicstyle=\footnotesize,       % the size of the fonts that are used for the code
  numbers=none,                   % where to put the line-numbers
  numberstyle=\tiny\color{gray},  % the style that is used for the line-numbers
  stepnumber=2,                   % the step between two line-numbers. If it's 1, each line will be numbered
  numbersep=5pt,                  % how far the line-numbers are from the code
  backgroundcolor=\color{white},  % choose the background color. You must add \usepackage{color}
  showspaces=false,               % show spaces adding particular underscores
  showstringspaces=false,         % underline spaces within strings
  showtabs=false,                 % show tabs within strings adding particular underscores
  frame=lines,                    % adds a frame around the code
  rulecolor=\color{black},        % if not set, the frame-color may be changed on line-breaks within not-black text (e.g. commens (green here))
  tabsize=2,                      % sets default tabsize to 2 spaces
  captionpos=b,                   % sets the caption-position to bottom
  breaklines=true,                % sets automatic line breaking
  breakatwhitespace=false,        % sets if automatic breaks should only happen at whitespace
  title=\lstname,                 % show the filename of files included with \lstinputlisting;
                                  % also try caption instead of title
  keywordstyle=\color{blue},      % keyword style
  stringstyle=\color{mauve},      % string literal style
  escapeinside={\%*}{*)},         % if you want to add a comment within your code
  morekeywords={*,...},            % if you want to add more keywords to the set
	extendedchars=true,
	literate={á}{{\'a}}1						%http://www.latexsearch.com/sandbox.do
	{é}{{\'e}}1
	{í}{{\'i}}1
	{ý}{{\'y}}1
	{ó}{{\'o}}1
	{ú}{{\'u}}1
	{ľ}{{\v{l}}}1
	{š}{{\v{s}}}1
	{č}{{\v{c}}}1
	{ť}{{\v{t}}}1
%	comment=[l]{\#},
%  commentstyle=\color{blue}
}
\lstset{morekeywords={select, where, order, by, limit, and, or, from, asc, desc, having, output, group, as}}
\renewcommand\lstlistingname{Výpis}
	
% ----- Zoznam skratiek -----
\usepackage[nolist,footnote]{acronym}

% ----- Okraje -----
\usepackage[top=2.5cm,left=2cm,right=3cm,bottom=2.5cm]{geometry}
%\usepackage[top=2.5cm,left=3cm,right=2cm,bottom=2.5cm]{geometry} %isis

\linespread{1.5}

\include{latex/i10preamble}

\begin{document}

% ----- Info o práci -----
%\title{Esper Web: Webové rozhranie pre spracovanie udalostí v reálnom čase}
\title{Esper Web: Webové rozhraní pro zpracování událostí v reálném čase}

\author{Martin Kravec}
\date{Marec 23, 2015}

%--------------------------------------------------------------
\frontmatter

\maketitle \thispagestyle{empty} \emptydoublepage

\chapter*{Prehlásenie}
Prehlasujem, že som diplomovú prácu spracoval samostatne, a že som uviedol všetky použité pramene a literatúru, z ktorej som čerpal.

\emptydoublepage

\chapter*{Poďakovanie}
Chcem poďakovať všetkým, ktorí ma podporovali a pomáhali mi, najmä svojim rodičom a starým rodičom za trpezlivosť a podporu pri štúdiu. V neposlednom rade ďakujem vedúcemu mojej práce, Mgr. Zbyňkovi Šlajchrtovi za cenné rady, ktoré mi dal.

\emptydoublepage

%\chapter*{Abstract\markboth{Abstract}{Abstract}}
%\addcontentsline{toc}{chapter}{\protect\numberline{}Abstract}
%\label{abstract}

\chapter*{Abstrakt}
Táto diplomová práca sa zaoberá problematikou spracovania komplexných udalostí. V teoretickej časti nájdeme vysvetlenie základných pojmov a popis použitých technológií. Čitateľ sa tiež zoznámi so základmi práce v Esperi.

Praktická časť práce sa zaoberá vytvorením administračného rozhrania, ktoré po napojení na serverovú časť aplikácie umožní použitie základnej funkcionality Esper engine aj bez predchádzajúcej znalosti programovania. Súčasťou praktickej časti je postup prípravy systému, inštalácie a spustenia aplikácie, keďže ide o netriviálnu úlohu.

\section*{Kľúčové slová}
cep, nosql, esper, cassandra, rest, rails, gui \emptydoublepage

\chapter*{Abstract}
This thesis deals with the complex event processing problem. In theoretical part you can find explanation of basic terms and description of used technologies. Reader also becomes fammiliar with basics of work with Esper.

Practical part is about building administrative interface, which after connecting to server part of application enables user to use basic functionality of Esper engine without previous programming knowledge. Part of this chapter is about system set-up, installation and launching the application as it is not considered to be a trivial task.

\section*{Keywords}
cep, nosql, esper, cassandra, rest, rails, gui \emptydoublepage

\tableofcontents \emptydoublepage

%--------------------------------------------------------------
\mainmatter

%Reaktivni programovani
Statement - zobraz vsetky tweety od kontaktu s menom..

Problemy
- nutna verzia java-1.7\_72 kvoli chybe pri nacitani xml pouzitim integrovanej jre kniznice
- her
- results table must have statement\_id pretoze by vznikli duplicity pri viacerych pouzivateloch ak by bolo pouzite statement.name - preto musi mat statement id nie name
- eventy typu xml nie je mozne prijimat cez esperio

\chapter*{Úvod}
\addcontentsline{toc}{chapter}{\protect\numberline{}Úvod}

Vo svojej práci som čerpal z množstva materiálov dostupných prevažne online. Čiastočne som využil aj prácu Tomáša Zvalu\cite{tomaszvala} o vysokej dostupnosti dát na virtuálnych serveroch, ktorú som rozšíril napríklad o pohľad na systémy zabezpečujúce automatické migrovanie procesov v prípade výpadku, podrobnejší popis vybraných oblastí a testy rýchlostí súborových systémov. Využil som tiež prácu, ktorú napísal Radek Zima\cite{radekzima}, konkrétne časť popisujúcu možnosti ukladania dát.


\emptydoublepage
\chapter{Definícia pojmov}
\label{chap:pojmy}

\section{CEP}
	\ac{CEP} je definovaný ako sada nástrojov a techník na analýzu a kontrolu komplexného sledu vzájomne prepojených udalostí, na ktorých sú postavené moderné distribuované informačné systémy. Táto technológia pomáha ľuďom v IT obore rýchlo identifikovať a vyriešiť mnohé problémy. \cite{power-of-events}

	\ac{CEP} je dôležitou súčasťou vývoja reaktívnych aplikácií a monitorovacích programov. Použitie nachádza v mnohých oblastiach, napríklad pre automatické obchodné systémy, detekcie podvodov, analýzu sentimentu trhu, optimalizáciu dodávateľského reťazca, transporte a logistike alebo systémoch pre rýchlu pomoc. Predpokladá sa tiež nárast jeho využitie v informačných systémoch spolu so zvyšujúcim sa počtom decentralizovaných zdrojov dát ako blogov a rozvojom tagovacích a senzorových technológií.

	\ac{CEP} stavia na dvoch nutných podmienkach: \cite{web:cep-editorial}
	\begin{itemize}
		\item Oddeľuje tvorcov a príjemcov informácií. Tvorcovia nevyžadujú informácie o príjemcoch, rovnako ako príjemci nepotrebujú vedieť kto dáta produkuje.
		\item CEP systémy okrem predávania informácií medzi tvorcami a príjemcami vo forme udalostí, ale umožňujú detekciu vzťahov medzi udalosťami.
	\end{itemize}
	
	Príkladom takého vzťahu je dočasný vzťah definovaný pomocou korelačných pravidiel (event pattern). Pomocou agregácie a kompozície je možné generovať nové udalosti a z nich ďalej odvodiť ďalšie udalosti. Korelácia udalostí navyše pomáha redukovať množstvo dát a pomáha tak budovať škálovateľné systémy.
	
	\ac{CEP} sa podobne ako iné technológie vyvíjal v čase a je postavený na základe definovanom \ac{SEP} a \ac{ESP}.
	\begin{itemize}
		\item SEP je najjednoduchším prípadom, kde je udalosť spracovaná izolovane bez ohľadu na ostatné udalosti.
		\item ESP spracováva prúdy udalostí ako kolekcie, identifikuje typy udalostí použitím kontinuálnych výrazov a selektuje zaujímavé udalosti.
		\item CEP je rozšírenie ESP o mechanizmy \ac{ECA} umožňujúce vykonávanie definovaných príkazov v závislosti na spracovávanej udalosti vyhovujúcej stanoveným podmienkam.
	\end{itemize}
	
	Pre prácu s komplexnými udalosťami je potrebné najprv pochopiť čo sa pod týmto pojmom skrýva. V ďalšom texte aj pri samotnej práci v administračnom rozhraní sa stretneme s typmi udalostí a samotnými udalosťami. Tieto pojmy sú definované nasledovne \cite{etzion2011event}:
	
	\begin{description}
		\item[Udalosť] je definovaná ako výskyt v danom systéme alebo doméne. Je to niečo čo sa stalo, alebo je predpokladané že sa stalo v danej doméne. Slovo udalosť je tiež používané vo význame programátorského objektu, ktorý reprezentuje výskyt udalosti v počítačovom systéme.
		
		\item[Typ udalosti] je špecifikáciou pre skupinu objektov udalostí, ktoré majú rovnaký účel a štruktúru. Každý objekt udalosti je považovaný za inštanciu typu udalosti.
	\end{description}

	Pred samotným spracovaním udalostí je nutné definovať ich typ, pravidlá spracovania a spracovanie výstupu. Po vykonaní týchto úkonov stačí na vstup procesného engine posielať udalosti daného typu. Model fungovania takéhoto systému vidíme na obrázku \ref{image_cep-model}. Na rovnakom princípe funguje serverová časť programu, ktorou sa bude bližšie zaoberať kapitola \ref{chap:implementacia}.
	\myFigure{cep-model}{Model spracovania udalostí \cite{web:softwarearchitekturen}}{Model spracovania udalostí}
	
	Na rozdiel do databázových systémov sú dotazy na do prúdu udalostí vyhodnocované priebežne, v momente výskytu danej udalosti. Aj keď databázy zvyčajne pracujú s údajmi o udalostiach (napríklad históriou objednávok), dotazy v databázach sú jednorázové a ad-hoc, oproti konečnej množine dát. Naopak dotazy v CEP sú predom definované a množinou dát je nekonečný tok udalostí cez nich prechádzajúci \cite{web:ceptranslated}.
		
	Samotné spracovanie udalostí je postavené na logike, ktorú definujú pravidlá spracovania (statementy). Tieto pravidlá si môžeme predstaviť ako filter, cez ktorý prichádzajúce udalosti pretekajú. Ak udalosť filtru vyhovuje, prevedú sa vopred definované akcie a udalosť je odoslaná na výstup.	Tieto pravidlá je možné definovať pomocou jazyka EPL, ktorý je syntaxou veľmi podobný SQL.
	
	EPL umožňuje definovať dátové okná, ktoré predstavujú obdobu pohľadov (view) ako ich poznáme z databázových prostredí. CEP sa ale zaoberá prácou s tokmi dát, nie statickým pohľadom na nich prináša EPL možnosť rozšíriť tieto náhľady o definíciu rozsahu skúmaných dát. Príkladom použitia takejto definície je definícia dátového okna, ktoré bude udržiavať špecifický počet prijatých udalostí, prípadne udalosti prijaté v istom časovom rozsahu. Príklad takýchto obmedzení vidíme na obrázku \ref{image_cep-windows}. Bližšie príklady EPL syntaxe nájdeme v kapitole \ref{chap:technologie}.

	\myFigure{cep-windows}{Príklad dátového okna obmedzeného časom a počtom udalostí \cite{web:softwarearchitekturen}}{Príklad dátového okna obmedzeného časom a počtom udalostí}
	
	Detekcia udalostí však sama o sebe dostačujúca, systémy riadené udalosťami tiež automaticky reagujú na detekované udalosti podľa predom stanovených pravidiel. Tie zvyčajne pozostávajú z notifikácie (užívateľa alebo ďalšieho systému), jednoduchých akcií (automatický nákup akcií, aktivácia požiarneho systému) alebo interakciou s iným systémom (spustením nového procesu).

	Na trhu existuje viacero komerčných CEP riešení od firiem ako Tibco, Oracle či IBM. Zo zástupcov open-source produktov sú to napríklad JBoss Drools Fusion alebo Esper od firmy EsperTech, na ktorom bude postavená aj táto diplomová práca.

	%From \cite{web:cep-editorial}.
	%http://www.tibco.com/blog/2009/08/21/cep-versus-esp-an-essay-or-maybe-a-rant/

\section{NoSQL}
	Priekopníkom vzniku NoSQL databázy boli vedúce internetové spoločnosti ako Google, Facebook, Amazon a LinkedIn - prekonávali tak limitácie konceptu relačných databáz pri použití v moderných webových aplikáciách. Dnes používajú organizácie NoSQL na riešenie problémov, ktoré priniesli štyri trendy:
	\begin{description}
		\item[Big Users] Pred pár rokmi bolo 1000 užívateľov konkrétnej aplikácie veľa, 10 000 užívateľov v extrémnych prípadoch. Dnes sú pripojených k internetu 2 bilióny ľudí, ktorí strávia online okolo 35 biliónov hodín mesačne. Pre webové aplikácie teda nie je výnimočné mať milióny rôznych užívateľov denne.
		
		\item[Big Data] Obrovský nárast používania internetu vyúsťuje v nárast údajov, ktoré užívatelia a aplikácie produkujú. Podľa odhadu bola v roku 2013 veľkosť uložených údajov 4,4 zetabajtov s predpokladom exponenciálneho rastu. Do roku 2020 sa predpokladá desať násobný nárast \cite{web:idc-bigdata}. Užívateľské informácie, geografické dáta, sociálne grafy, údaje vyprodukované užívateľmi a aplikáciami a senzorové dáta sú príkladom nekonečných generátorov dát.
		
		\item[Internet of Things] je trend, ktorý je definovaný stále rastúcim množstvom prepojených zariadení - ktoré generujú dáta. Dnes je k internetu pripojených okolo 20 biliónov zariadení, vrátane telefónov, tabletov, zariadení v nemocniciach, autách či skladoch. Tieto zariadenia získavajú údaje o svojom okolí, pohybe alebo počasí z ich 50 biliónov senzorov.
		Napríklad telemetrické údaje, ktoré sú čiastočne štruktúrované a kontinuálne a pre SQL databázy s pevne definovanou schémou a štrukturovaným spôsobom ukladania dát predstavujú problém.
		
		\item[Cloud Computing] Množstvo aplikácií je dnes postavených na cloud infraštruktúre a využíva trojvrstvovú architektúru. V tej je k aplikáciám pristupované cez internet pomocou mobilnej aplikácie internetového prehliadača. Load balancery zodpovedajú za rozloženie záťaže a smerujú prichádzajúce požiadavky na jednotlivé servery, ktoré obsluhujú logiku aplikácie. Pri rastúcej záťaži nie je problémom pridať do konfigurácie load balancera ďalší server a takto rozložiť záťaž.
		Problém nastáva v databázovej vrstve. Relačné databázy boli zvyčajne pôvodným riešením, avšak ich použitie je čoraz viac problematické pretože databáza je centralizovaná a škálovateľná vertikálne, nie horizontálne. To predstavuje nevýhodu pre aplikácie rastúce dynamicky. NoSQL databázy sú distribuované a škálovateľné horizontálne, čo umožňuje podobne ako pri load balanceroch jednoducho pridať ďalší server a rozložiť záťaž.
	\end{description}
	
	NoSQL používa rozdielny dátový model ako relačné databázy. Relačný model rozprestrie dáta do viacerých vzájomne prepojených tabuliek obsahujúcich riadky a stĺpce. Pri získavaní informácií z relačnej databázy musia byť údaje z týchto tabuliek spojené. Podobne pri zápise musia byť údaje rozložené do jednotlivých tabuliek.

	Naproti tomu NoSQL dátový model agreguje ukladané dáta do jedného objektu - napríklad dokumentu v prípade objektových databáz. V NoSQL nie je definovaná operácia JOIN, čo vedie k duplikácií informácií vo viacerých tabuľkách. Tento problém však vyvažuje dnes lacný úložný priestor, flexibilita dátového modelu, zlepšený výkon operácií čítania a zápisu a škálovateľnosť tohoto systému.
	
	Ďalším veľkým rozdielom je že relačné databázy musia definovať schému tabuľky, naproti NoSQL databázam, ktoré fungujú aj bez definovanej schémy. Úpravy schém tabuľky obsahujúcej údaje je často problematické, čo môže byť v dobe kedy je nutný kontinuálny vývoj aplikácie a neustále pridávanie novej funkcionality nepríjemné. Naproti tomu sú napríklad databázy dokumentov bez schémy, čo umožňuje úpravy jednotlivých položiek bez narušenia ostatnej funkcionality. Príklad tabuľky s dynamickou štruktúrou vidíme na obrázku \ref{image_dynamic_column_family}.

	\myFigure{dynamic_column_family}{Dynamická štruktúra tabuľky databázy Cassandra}{Dynamická štruktúra tabuľky}
	
	Pre vyriešenie problémov s rastúcim počtom užívateľov a údajov je nutné aplikácie škálovať - a to horizontálne alebo vertikálne.
	\begin{itemize}
		\item Vertikálne škálovanie predstavuje centralizovaný prístup založený na vylepšovaní konfigurácie serverov, prípadne ich nahradení výkonnejšími servermi.
		\item Horizontálne škálovanie predstavuje distribuovaný prístup, kde sú do konfigurácie pridané ďalšie servery, ktoré umožňujú rozloženie záťaže.
	\end{itemize}
	Pred príchodom NoSQL bolo obvyklé vertikálne škálovanie. Pri raste množstva dát boli potrebné výkonnejšie servery s väčšou RAM, výkonnejšími procesormi a väčším diskovým priestorom. Cena a komplikovanosť takýchto serverov však neúmerne stúpa so zvyšujúcimi sa požiadavkami, ako je možné vidieť na obrázku \ref{image_scale-up-out}. Po istej úrovni už nie je možné konfiguráciu serveru vylepšiť, je nutné zaobstarať ďalší a rozloženie zaťaženia databázy riešiť na aplikačnej úrovni, čo je veľmi náročná úloha pre programátorov a správcov databázy. 
	
	\myFigure{scale-up-out}{Škálovanie aplikácie vzhľadom na jej cenu}{Škálovanie aplikácie vzhľadom na jej cenu}
	
	\ac{NoSQL} zahrňuje širokú škálu rôznych databázových technológií v reakcii na prudký nárast množstva ukladaných dát, frekvencie prístupu k údajom a výkonnostným požiadavkom, ktoré z tohoto nárastu  vyplývajú.
	
	Relačné databázy neboli navrhnuté so zreteľom na požiadavky a nároky s ktorými sa dnešné aplikácie stretávajú. Tiež neumožňujú využiť výhody lacných úložísk dát a výpočtového výkonu.
	
\section{Deklaratívne programovanie}
	\cite{dekprog}
	
	
	
%	http://www.mongodb.com/nosql-explained
%	http://www.couchbase.com/nosql-resources/what-is-no-sql
%	http://www.go-gulf.com/blog/online-time/

\chapter{Technológie}
\section{Esper}
	Esper je komponenta, ktorá umožňuje spracovanie komplexných udalostí \ac{CEP}. Umožňuje vývoj aplikácií spracovávajúcich veľké množstvo udalostí - v reálnom čase ako aj historických. Tieto udalosti je možné filtrovať a analyzovať podľa potreby a reagovať v reálnom čase na predom definované stavy.  Esper je dostupný v troch verziách:
	\begin{description}
		\item[Esper] je dostupné ako open source s možnosťou komerčnej podpory. Táto verzia obsahuje základ potrebný pre použitie esperu, užívateľ však musí jednotlivé statementy, schémy a nastavenia realizovať programovo. Je preto náročný na použitie pre ľudí, ktorí nevedia programovať. Riešenie je vhodné pre firmy, ktoré buď nevyužijú platenú verziu alebo majú špecifické požiadavky na výsledný produkt a sú schopné túto verziu podľa svojich potrieb upraviť.
		
		\item[Esper HA] je riešenie umožňujúce vysokú dostupnosť Esperu. Zabezpečuje že stav je po vypnutí alebo havárii obnoviteľný. Statementy, schémy a iné nastavenia si EsperHA pri reštarte uchováva, čo je výhoda oproti predchádzajúcej verzii - kde je nutné tieto úkony riešiť programovo. Táto verzia je vhodná pre projekty závislé na vysokej dostupnosti esperu a subjekty, pre ktoré je kritická neustála kontrola prichádzajúcich udalostí.
		EsperHA je spoplatnený, dostupná je trial len verzia, pre ktorej použitie je nutné identifikovať sa ako spoločnosť. Cena nie je na webových stránkach dostupná.
		
		\item[Esper Enterprise Edition] je kompletný produkt "na kľúč", obsahujúci všetky komponenty potrebné pre nasadenie do podniku. V jednom balíku je obsiahnuté GUI pre správu esperu, restové služby poskytujúce prístup zvonku, \ac{EPL} editor, zobrazovacie nástroje umožňujúce kontinuálne zobrazenie výsledkov v grafoch a tabuľkách. EsperEE je možné skombinovať s EsperEA pre dodatočné zabezpečenie vysokej dostupnosti. EsperEE je spoplatnený, rovnako ako pri EsperEA je dostupná trial verzia po splnení určitých podmienok. Cena nie je dostupná, tieto dve dve riešenia sú určené predovšetkým pre podnikový sektor.
	\end{description}
	
	Pre tento projekt je použitá verzia Esper, ktorú som rozšíril o prístup k základným funkciám pomocou restovej api a persistenciu niektorých nastavení a nájdených výsledkov. Táto verzia je dostupná pod GNU General Public License (GPL) (GPL v2).

	\subsection{Procesný model}

	\subsection{Typ udalosti}
	Každá udalosť spracovávaná esperom je definovaná schémou, takzvaným typom udalosti. Tie môžu byť definované pri štarte aplikácie, alebo programovo počas behu. EPL tiež obsahuje klauzulu CREATE SCHEMA umožňujúcu definovanie typu udalosti pomocou EPL. Prehľad základných typov udalostí je v nasledujúcej tabuľke.

	\myTable{
	\begin{tabular}{ | l | p{10cm} | }
		\hline
		Trieda	&	Popis	\\ \hline
		java.lang.Object	&	Akýkoľvek Java POJO (plain-old java object) s getter metódami. Takáto definícia je najjednoduchšia na úkor možnosti úprav počas behu programu.	\\ \hline
		java.util.Map	&	Udalosti definované ako implementácia java.util.Map interface, kde každá hodnota záznamu je vlastnosť udalosti.	\\ \hline
		Object[] (pole objektov)	&	Udalosti definované objektovým poľom, kde každá hodnota poľa je vlastnosť udalosti.	\\ \hline
		org.w3c.dom.Node	&	XML objektový model dokumentu popisujúci štruktúru udalosti.	\\ \hline
	\end{tabular}
	}{Základné typy udalostí}{table:event-types}
	
	Definície typu udalosti sú rozšíriteľné zásuvnými modulmi. Aplikácia môže používať kombináciu týchto typov, nemusí všetky typy definovať jedným spôsobom. Definície typov udalostí je možné reťaziť, kedy typom udalosti môže byť iná komplexná udalosť.
	
	Z dôvodu nutnosti pridávania a mazania udalostí počas behu programu nemôže byť v mojej implementácii použitá definícia typu pomocou POJO. A pretože klient musí mať možnosť definovať typ, bola zvolená definícia pomocou XML schémy. Jednoduchý príklad schémy udalosti znázorňuje nasledujúci výpis. 
	
	\begin{lstlisting}[label=lst:sample-schema,caption=Jednoduchý príklad XML schémy udalosti]
<?xml version="1.0" encoding="UTF-8"?>
<xs:schema xmlns:xs="http://www.w3.org/2001/XMLSchema">
	<xs:element name="TweetEvent">
		<xs:complexType>
			<xs:sequence>
				<xs:element name="username" type="xs:string"></xs:element>
				<xs:element name="message" type="xs:string"></xs:element>
			</xs:sequence>
		</xs:complexType>
	</xs:element>
</xs:schema>		
	\end{lstlisting}

	Po definovaní typu udalosti sa na ne môžeme odkazovať klauzulou FROM v EPL statementoch. Tie sú bližšie popísané v nasledujúcej sekcii.

	\subsection{EPL}
		Event Processing Language je jazyk umožňujúci definovanie statementov a patternov v CEP. Syntaxou je podobný SQL. Obsahuje klauzuly SELECT, FROM, WHERE, GROUP BY, HAVING a ORDER BY. Namiesto tabuliek však pracuje so streamami, kde riadok tabuľky nahrádza prichádzajúca udalosť. Streamy udalostí je možné spájať pomocou joinov, filtrovať a agregovať.
		
		Klauzula INSERT INTO je použiteľná na presmerovanie udalostí na iný stream pre dodatočné spracovanie. Klauzula UPDATE slúži na úpravu vlastností udalosti a je aplikovaná pred spracovaním statementu. 
		
		EPL statementy môžu obsahovať definíciu náhľadov (view). Tieto majú viacero funkcií, ako okná náhľadu na stream udalostí, tvorenie štatistík v vlastností udalosti alebo zoskupovanie udalostí. Náhľady môžu byť reťazené. Zabudované náhľady sú napríklad win:length - aplikuje statement na definovaný počet udalostí, win:time - aplikuje statement na udalosti obmedzené časovo alebo std:lastevent - ktorý obsahuje poslednú prijatú udalosť.
		
		EPL definuje koncept pomenovaných okien (named windows), ktoré slúžia ako štruktúra uchovávajúca udalosti. Je možné do nej vkladať nové udalosti a mazať staré. Výhodou tejto štruktúry je možnosť jej použitia viacerými statementami, pretože je globálna, teda zdieľaná v rozsahu daného service providera.	
		
		Pomocou EPL môžeme tiež definovať premenné, ktoré slúžia na vkladanie parametrov do statementu a definovanie schém udalostí. Tými sa zaoberá nasledujúca sekcia.
		
		\subsubsection{Syntax}
		Každý EPL statement musí obsahovať minimálne klauzulu SELECT a FROM.

		SELECT klauzula môže obsahovať náhradný znak * alebo vymenovať požadované vlastnosti udalosti. Tiež definuje typ výslednej udalosti publikovanej statementom. SELECT poskytuje aj nepovinné klauzuly istream (input), irstream (input \& remove) a rstream (remove), ktoré definujú streamy, ktorých udalosti sa majú poslať na UpdateListener a observery statementu. Prednastavené je použitie nastavenia istream.
		
		FROM klauzula špecifikuje jeden alebo viac streamov, pomenovaných okien alebo tabuliek (od verzie esper 5.1). Tie môžu byť pomenované klauzulou AS. Pre join je potrebné definovať viacero streamov. Podporovaný je tiež join so relačnou databázou ako zdrojom dát. To je možné využiť napríklad na prístup k historickým dátam.

		EPL umožňuje definovať pattern, a to ako samostatný výraz alebo ako súčasť statementu. Pattern sa môže vyskytovať kdekoľvek v klauzule FROM, vrátane join. Môžu preto byť použité v kombinácii s klauzulami WHERE, GROUP BY, HAVING a INSERT INTO.
		
		%TODO Patterny

		V nasledujúcich výpisoch sú príklady statementov, zobrazujúcich príklady syntaxe popisovanej v predchádzajúcom texte.
				
		\begin{lstlisting}[label=lst:epl-simple,caption=Jednoduchý EPL statement]
		select * from TweetEvent.win:time(60 sec) where message='happy'
		\end{lstlisting}
		
		\begin{lstlisting}[label=lst:epl-join,caption=Jednoduchý EPL statements použitím join]
		select * from TickEvent.std:unique(symbol) as t, NewsEvent.std:unique(symbol) as n
		where t.symbol = n.symbol
		\end{lstlisting}
		
		\begin{lstlisting}[label=lst:epl-pattern,caption=EPL statement s použitím patternu]
		select a.custId, sum(a.price + b.price)
		from pattern [every a=ServiceOrder -> 
			b=ProductOrder(custId = a.custId) where timer:within(1 min)].win:time(2 hour) 
		where a.name in ('Repair', b.name)
		group by a.custId
		having sum(a.price + b.price) > 100
		\end{lstlisting}
				
		Reprezentácia udalostí formou pojo, xml..
		\subsubsection{Objektový model}	

	\subsection{Api}
		\subsubsection{Konfigurácia}
		EPServiceProvider, epadministrator, epruntime
		
		\subsubsection{Vytváranie statementov}
		\subsubsection{Spracovanie výsledkov}

\section{Cassandra}
	...
	\subsection{Cassandra Query Language}
	\ac{CQL} je jazyk ...
	
\section{Frameworky}
	\subsection{Spring}
	\subsection{RubyOnRails}
	\subsection{Sinatra}
	\subsection{Bootstrap}

\chapter{Návrh a Implementácia}
\label{chap:implementacia}

Táto kapitola popisuje implementačné detaily projektu. Kompletný projekt obsahuje štyri časti, serverovú api, jej administračné rozhranie a dva príklady použitia. Administračné rozhranie ako aj oba príklady použitia sú závislé na serverovej časti, na ktorej restové rozhranie sa pripájajú. Serverová api je nezávislá, jedinou prerekvizitou je databáza cassandra, ktorá musí byť spustená ako prvá. Závislosti komponentov aplikácie zobrazuje diagram \ref{image_components}.
\myFigure{components}{Diagram komponentov aplikácie}{Diagram komponentov aplikácie}

Jednotlivé komponenty nájdeme vo výslednom projekte v samostatných adresároch. Projekt navyše obsahuje adresár so vzorovými schémami a udalosťami.

\section{ThesisApi}
	Esper je voľne dostupný pre dve vývojové platformy - javu a .net. Pre serverovú časť aplikácie som zvolil javu s použitím spring frameworku.

	Vo finálnej verzii som ponechal aj súbory v balíčku sample, ktoré slúžia na testovanie aplikácie. Ich použitím je možné jednorázovo spustiť serverovú časť aplikácie, načítať schémy a definovanými pravidlami vyhodnotiť udalosti uložené v csv súboroch v resources adresári. Takéto jednorázové spustenie uľahčuje hľadanie chýb, prípadne testovanie novej funkcionality aplikácie.

	
model
exportall - getall

	\subsection{Konfiguračné súbory}
		Aplikácie využíva konfiguračné súbory umiestnené v adresári resources, ktoré sú načítané pri spustení. Najdôležitejšími z nich sú konfigurácie spring frameworku a databázy:
		\begin{description}
			\item[application-config.xml:] Konfiguračný súbor spring frameworku, ktorého najzaujímavejšou časťou je konfigurácia jdbc. Tá špecifikuje adaptér pripojenia k databáze konfigurácie, konkrétne parametre pripojenia sú načítané zo súboru database.properties.
			\item[database.properties:] Súbor obsahuje nastavenie pripojenia k databázi. Jeho úpravou môžeme jednoducho nahradiť použitý databázový systém iným.
		\end{description}
		Ďalej v tomto adresári nájdeme vzorové schémy udalostí a samotné udalosti použité pre testovanie alebo konfiguráciu logovania.
	
	\subsection{Databáza}
		Aplikácia využíva dve databázy, jednu na ukladanie konfiguračných dát a druhú na udalosti nájdené esperom. Je to tak kvôli predpokladu že esper bude produkovať veľké množstvo dát, preto je vhodné použitie databázy optimalizovanej pre zápis.
		
		Prístup k databázam zabezpečujú triedy v balíčku DAO, ktoré sú pre použitie injektované do ostatných objektov aplikácie. Balíček obsahuje tri triedy, dva z nich slúžia na správu schém a statementov a využívajú konfiguráciu načítanú zo súboru database.properties. Tretia trieda slúži na prácu s výsledkami statementov o jej konfiguráciu sa stará trieda Constants.java. Implementácia triedy obsluhujúcej výsledky je riešená trochu odlišne, kde sú všetky dotazy špecifikované na začiatku a v konštruktore inicializované do objektu PreparedStatement. Je to takto z dôvodu optimalizácie rýchlosti zápisu.			
	
		\subsubsection{Databáza konfigurácie}
		Ako databázu pre ukladanie stavu esperu a jednotlivých konfiguračných položiek som použil \ac{HSQLDB}. Táto databáza je napísaná v jave a pre ukladanie tabuliek ponúka pamäťový aj diskový mód. Použil som ju, pretože je jednoduchá, nenáročná na systém a je možné ju stiahnuť ako jednu zo závislostí aplikácie pomocou mavenu. Databáza je jednoducho nahraditeľná iným riešením, keďže k nej je pristupované pomocou jdbc. Pre zmenu je potrebné upraviť konfiguračný súbor application-config.xml a zmeniť položku dataSource.
		
		Databáza obsahuje tabuľku schém a statementov, ktoré sú načítavané pri štarte Api. Jej štruktúru vidíme na obrázku \ref{image_db-model-1}.
		\myFigure{db-model-1}{Model databázy konfigurácie}{Model databázy konfigurácie}
		
		Pri ukončení aplikácie je nutné sa od databázy odpojiť aby boli zmeny dočasne uložené v pamäti presunuté na disk. Toto odpojenie zaobstaráva spúšťací súbor aplikácie, volaním príkazu "SHUTDOWN". Odpojenie je nutné vykonať aj pri nastavení automatického ukončenia relácie v parametroch pripojenia k databáze.
			
		\subsubsection{Databáza udalostí}
		Databáza udalostí slúži na ukladanie výsledkov vyhovujúcich niektorému z definovaných statementov. Ako konkrétne riešenie som použil databázu Cassandra. Cassandra je jedným z predstaviteľov NoSQL databáz. Jednou z jej výhod je možnosť rozšíriteľnosti v prípade veľkého objemu dát a optimalizácia pre zápis. Použitie tejto databázy je v rámci "proof of concept" princípu, kde by bolo jednoduchšie pracovať napríklad s HSQLDB ako v prípade konfigurácie, avšak kvôli možnosti generovania veľkého množstva udalostí esperom je použitá databáza na to vhodná.
		\myFigure{db-model-2}{Model databázy udalostí}{Model databázy udalostí}
	
	\subsection{Spring framework}
		Ako základ serverovej časti aplikácie som použil Spring framework. Jeho hlavnou úlohou je injektovanie závislostí vo väčšine tried, no využil som aj rozšírenia pre databázu, webový prístup a pribalený webový server tomcat.
		Spring Framework poskytuje v aplikácii podporu pre dependency injection, správu transakcií a prístup k dátam pomocou jdbc. Serverová časť aplikácie využíva nasledujúce knižnice:
		\begin{description}
			\item[spring-core:] Pomocou anotácie @Autowired sú v aplikácii riešené závislosti väčšiny komponent.
			\item[spring-jdbc:] Prístup do databázy je realizovaný pomocou spring triedy NamedParameterJdbcTemplate, ktorá oproti jdbc pridáva možnosť prístupu k vygenerovanému id nového záznamu.
			\item[spring-webmvc:] Prístup k dátam a ovládaniu serverovej časti aplikácie je umožnený pomocou restovej api.
			\item[spring-boot-starter-web:] Táto závislosť umožňuje použitie pribaleného tomcat serveru. Ten sa spustí pri štarte aplikácie a sprístupní restovú api.
		\end{description}
		
		Súčasťou spring frameworku je webové rozšírenie umožňujúce tvorbu restových služieb. V tejto implementácii som ich rozdelil do štyroch zdrojových súborov podľa oblastí, ktoré obsluhujú. Prvé dva SchemaController a StatementsController poskytujú prístup k správe schém a statementov, ResultController umožňuje prácu s výsledkami statementov a EsperController sa stará o spracovanie prichádzajúcich udalostí. Každá trieda je anotovaná pomocou @RestController a každá metóda definuje unikátnu cestu a HTTP metódu, pomocou ktorej sa vzdialene volá. Za zmienku tiež stojí závislosť ResultController na StatementsController, kde cesta k volaniu jednotlivých metód je vnorená. Výpis \ref{lst:rest-api-path} zobrazuje cesty ku controllerom, pri ktorých si môžeme všimnúť že akceptujú vstup vo forme JSON. V nasledujúcich kapitolách budú tiež detailne špecifikované prístupové cesty k jednotlivým REST metódam. 
		
		\begin{lstlisting}[label=lst:rest-api-path,caption=Definícia ciest REST API]
	@RequestMapping(value = "esper", produces = "application/json")
	@RequestMapping(value = "schemas", produces = "application/json")
	@RequestMapping(value = "statements", produces = "application/json")
	@RequestMapping(value = "/statements/{statement_id}/results", produces = "application/json")
		\end{lstlisting}
		
		Spustenie aplikácie je implementované v triede Application.java, v ktorej je načítaná konfigurácia. Pomocou rozšírenia spring boot sú použité prednastavené (nekonfigurované) hodnoty podľa prístupu convention-over-configuration. Následne je automaticky spustený aplikačný server tomcat, ktorý je súčasťou distribúcie.

	\subsection{Esper engine}
		Esper je v aplikácii implementovaný triedou EsperManager, ktorá poskytuje prístup ku konfiguračným nástrojom a sprístupňuje handler prichádzajúcich udalostí. Pri spustení aplikácie načítava konfiguračné údaje z databázy a inicializuje esper. Najdôležitejšou úlohou tejto triedy je správa schém a statementov.
		\begin{description}
			\item[Schémy] sú reprezentované XML (org.w3c.dom.Node) dokumentom. Oproti POJO reprezentácii umožňuje tento formát vytváranie schém počas behu čo je z pohľadu užívateľa nutnosťou. V prípade potreby by bolo možné použiť udalosti reprezentované pomocou java.util.Map alebo objektovým poľom, avšak to vyžaduje definovanie formátu prenosu a následné spracovanie do požadovaného formátu klientom, čo nie je veľmi intuitívne. Použitie XML reprezentácie klientovi umožní komunikovať priamo s esperom.
			
			\item[Udalosti] ktoré engine spracováva musia byť v XML formáte. Toto obmedzenie je zapríčinené XML reprezentáciou schém. Ako zdroje udalostí týmto eliminujeme adaptéry CSV a HTTP z esperio knižnice. Na prijímanie XML udalostí slúži rest služba. Tá spracováva vstupný stream dvoma spôsobmi:
			\begin{itemize}
				\item Ako jednotlivé udalosti, kde koreňový element udalosti reprezentuje názov schémy reprezentujúcej udalosť.
				\item Ako stream udalostí, kde je koreňový element nazvaný "events" a obaľuje jednotlivé udalosti definované v predošlom bode. Koreňový element je v tomto prípade ignorovaný.
			\end{itemize}
			
			\item[Statementy] pridávané do esper engine môžu obsahovať len meno a epl výraz. Preto je ku každému statementu priradený aj užívateľský objekt - statementBean s dodatočnými informáciami:
			\begin{itemize}
				\item ID ktoré unikátne identifikuje statement v rámci celej aplikácie, nie len daného esper provideru
				\item TTL hovoriaci ako dlho má byť nájdená udalosť perzistentná.
				\item STATE udávajúci či je konkrétny statement spustený alebo zastavený
			\end{itemize}
				

			Pri pridávaní statementu do esper engine je kontrolovaná unikátnosť mena v rámci daného esper providera. V prípade že existuje statement s rovnakým menom je meno doplnené o "--N", kde N značí najbližšie voľné celé číslo. Takto upravený statement je následne uložený.
			
			Pri pridávaní XML schém udalostí je nutné ich konvertovať do typu Document. To zaobstaráva trieda Helpers.java, ktorá sa zároveň stará napríklad o konvertovanie typov udalostí do JSON formátu.
			
		\end{description}
		
		Aplikácia definuje jeden globálny listener, ktorý je priradený všetkým statementom. Ten v prípade výskytu udalosti vyhovujúcej niektorému z definovaných statementov uloží udalosť vo forme obsahu json objektu do databázy.
		
		Esper umožňuje export výsledkov pomocou JSONRenderer a XMLRenderer, avšak tieto triedy produkujú formátovaný výstup, čo je v serverovej časti aplikácie neželané. Preto je konverzia realizovaná upravenými verziami týchto tried, ktoré nepridávajú formátovacie znaky a produkujú validné XML a JSON dokumenty. Predvolene je použitý upravený JSONRenderer.
		
		Pri ukladaní týchto týchto výsledkov do databázy je nastavený TTL, ktorý bol definovaný pri vytvorení statementu. Ak TTL pri vytvorení statementu nebolo definované použije sa predvolené nastavenie, kde sú výsledky persistentné až kým ich užívateľ manuálne neodstráni.

	\subsection{Maven}
		Zostavenie projektu je jednou z nevyhnutných súčastí tvorby java aplikácií. Na uľahčenie tohoto procesu je možné použiť viacero nástrojov, ktorých hlavnými predstaviteľmi sú maven, gradle a ant. Api komponent tejto aplikácie je zostavený pomocou nástroja maven. 
		
		Maven uľahčuje prácu vo viacerých oblastiach, a to \cite{web:maven-doc}:
		\begin{itemize}
			\item Uľahčenie prekladu aplikácie
			\item Poskytnutie jednotného riešenia pre zostavenie aplikácie
			\item Poskytnutie informácií o projekte
			\item Poskytnutie vzorov pre vývoj aplikácií
			\item Umožnenie transparentnej migrácie nových vlastností programu
		\end{itemize}
		Pre túto prácu je najdôležitejšia prvá oblasť. Vďaka mavenu nemusí distribúcia projektu obsahovať všetky knižnice závislostí, ani nemusia byť jednotlivo sťahované pri zostavovaní projektu. Všetky potrebné závislosti sú definované v súbore pom.xml a pri preklade automaticky stiahnuté z internetu. Tento súbor zároveň definuje vlastnosti projektu ako názov, verziu, verziu javy použitú pre zostavenie, spôsob generovania dokumentácie a iné detaily.
		
		Maven tiež umožňuje vytvárať v pom.xml súboroch závislosti a odkazovať sa na externé konfigurácie. Táto funkcionalita je v projekte využitá pri spring závislostiach, kde je verzia niektorých komponent definovaná v externom rodičovskom súbore. Rodičom spring závislostí projektu je spring-boot-starter-parent.
		
		%Väčšina funkcionality mavenu sa skladá z pluginov, k

\section{ThesisWeb}
	Web časť projektu slúži ako administračné rozhranie. Je riešené formou webovej aplikácie, ktorej základ je framework Ruby On Rails. Ten umožňuje rýchle vytváranie stránok, kde prevažuje prístup "convention over configuration". Dôležitou súčasťou frameworku sú gemy, ktoré reprezentujú závislosti projektu. Jedným z najpodstatnejších v tomto projekte je gem "her", ktorý zabezpečuje komunikáciu s Api časťou projektu.
	
	Grafická stránka tohoto projektu sa opiera o framework Bootstrap. Ten v základnej konfigurácii poskytuje html komponenty s vylepšeným UI. Tiež opravuje niektoré chyby kompatibility pri zobrazovaní stránok v rôznych prehliadačoch.
	
	Web nevyužíva žiadne persistentné úložisko dát. Všetky realizované zmeny sú posielané na restovú api, ktorá zmeny spracuje a uloží. Zobrazované údaje sú tiež získavané zo vzdialeného zdroja.
		
	presmerovanie v pripade ze engine je offline
	napojenie cez konfiguraciu v her
	usecase


	
\section{UseCase}
	Na predvedenie možností použitia aplikácie som vytvoril 2 jednoduché programy. Prvý Twitter Stream demonštruje možnosť zasielania udalostí na server a Contacts Web predstavuje možnosť alternatívneho použitia esperu. 
	
	\subsection{Twitter Stream}
	Twitter Stream predstavuje možnosť zasielanie streamu udalostí na REST rozhranie serveru. Skript je napísaný v Ruby a demonštruje ako sa je možné pomocou necelých 30 riadkov kódu napojiť na Twitter Api a presmerovať vzorku tweetov na server. Prijatý tweet je vo forme objektu, ktorý je nutné pred preposlaním formátovať do XML dokumentu. Skript pred spustením odošle na server požiadavku pre registrovanie schémy tweetu, ktorá sa nachádza v samostatnom súbore.
	
	Podobne ako je predvedené v Twitter Stream je možné naprogramovať ďalšie generátory udalostí. Jedinou podmienkou je definovanie schémy (čo je možné aj pomocou administračného rozhrania) pred spustením generátora a streamovanie udalostí vo forme XML dokumentov na URL definovanú v administračnom rozhraní.
	
	\subsection{Contacts Web}
	Contacts Web je webová aplikácia, zostavená pomocou frameworku Sinatra a Bootstrap. Aplikácia obsahuje prezentačnú a aplikačnú vrstvu, databázová vrstva je riešená pomocou REST volaní na server. Vďaka tomu je možné logiku aplikácie zmestiť do necelých 30 riadkov kódu.
	
	Logika aplikácie pozostáva z obsluhy požiadavkov na vytvorenie nového kontaktu, zmazanie existujúceho a výpis existujúcich kontaktov. Všetky tieto požiadavky sú posielané na server.
	
	Možnosť filtrovania údajov pomocou preddefinovaných pravidiel.
	Deklaratívne programovanie.
	
\chapter{Inštalácia}
\section{Príprava systému}

\subsubsection{Operačný systém}
0. Operačný systém : DreamSpark - Windows 8 Professional N 64-bit (English)

\subsubsection{Java}
problem pri pridavani schemy definovanej xml dokumentom
fix: com.espertech.esper.client.ConfigurationException: Failed to read schema via URL 'null'
\myTable{
\begin{tabular}{ | l | c | }
	\hline
	Verzia	&	Stav	\\
	\hline
	jdk1.7.0\_25	&	OK	\\ \hline
	jdk1.7.0\_71	&	NOK	\\ \hline
	jdk1.7.0\_72	&	OK	\\ \hline
	jdk1.8.0\_25	&	NOK	\\ \hline
	jdk1.8.0\_31	&	OK	\\
	\hline
\end{tabular}
}{Verzie Javy kompatibilné s esper xml schémami}{table:vmware-parametre}
	
\subsubsection{Maven}
Maven 3.2.3	http://maven.apache.org/
\begin{lstlisting}[label=lst:crm,caption=Čiastočný výpis konfigurácie crm]
PATH = C:\Apache\apache-maven-3.2.3\bin
JAVA_HOME - C:\Program Files\Java\jdk1.8.0_31
mvn install
\end{lstlisting}

\subsubsection{Cassandra}
=== Apache Cassandra 2.1.2	http://cassandra.apache.org/
\begin{lstlisting}
PATH = C:\Apache\apache-cassandra-2.1.2\bin
\end{lstlisting}

\subsubsection{RubyOnRails}
RailsInstaller 3.1.0	http://railsinstaller.org/en
bundle install
	a. problem s certifikatom (fix: https://gist.github.com/fnichol/867550)
		Error:SSL\_connect returned=1 errno=0 state=SSLv3 read server certificate B: certificate verify failed,
	b. http://nodejs.org/
		ExecJS::ProgramError - TypeError: Object doesn't support this property or method


\section{Spustenie aplikácie}
Git pull projekt, ked je system pripraveny
https://github.com/kravciak/ThesisWeb/archive/master.zip

\subsubsection{Api}
mvn install
mvn -run..?
\subsubsection{Web}
bundle install
rails s
\subsubsection{Twitter Stream}
ruby twitter.rb - stream
\subsubsection{Clients Web}
ruby client.rb - sinatra

\section{Použitie}
popis GUI..
\chapter*{Záver}
\addcontentsline{toc}{chapter}{\protect\numberline{}Záver}


\emptydoublepage

% ====== Zoznam skratiek ======
\begin{acronym}
	\acro{CEP}{Complex event processing}
	\acro{JDBC}{Java database connectivity technology}
	\acro{EPL}{Event Processing Language}
	\acro{CQL}{Cassandra Query Language}
	\acro{HSQLDB}{HyperSQL DataBase}
	\acro{SEP}{Simple Event Processing}
	\acro{ESP}{Event Stream Processing}
	\acro{CEP}{Complex Event Processing}
	\acro{ECA}{Event Condition Action}
	\acro{POJO}{Plain Old Java Object}
	%\acro{}{}
	%\acro{}{}
	%\acro{}{}
	%\acro{}{}
	%\acro{}{}
\end{acronym}


%--------------------------------------------------------------
\backmatter
%Bibliografia
%\phantomsection
%\nocite{*}
\bibliographystyle{ieeetr}
\bibliography{bibliografia}	\emptydoublepage

\listoffigures \emptydoublepage
\listoftables

\end{document}

% Dlhe url v bibliografii
%http://tex.stackexchange.com/questions/10924/underfull-hbox-in-bibliography
