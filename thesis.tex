% na zaciatok pouzitia odkial sa da stiahnut
% k pravidlam syntaxe popis alebo odstranit cislovane odkazy
% vytvorenie databazy
% uvod do cassandry
% named windows
% názorné priklady - 3 sposoby vkladania dat - zo suboru / handleru / webova aplikacia
% odstranit duplicitny priklad schemy
% obrazok z nosql do cassandry
% obrazok z esper do views?

% statement = vyraz, pattern - vzor, observer
%= rest api vracia metadata (strankovanie, schema...)
%= insert into clause
%= std:lastevent, win:length...
%= UUID, rozdiel v cassandra DAO
%= vytvorenie cassandra tabulky a priklad vyhladania - preco takyto primary key
%= cassandra join, order, count
%= komplexna schema
%= priklad udalosti
%= klauzuly select from ...
%= priklad objektoveho modelu


%\documentclass[12pt, a4paper]{book}
\documentclass[12pt, a4paper, oneside]{book} %isis

% ----- Nastavenie pisma -----
\usepackage[slovak]{babel}
%\usepackage[T1]{fontenc}
\usepackage[IL2]{fontenc}
\usepackage[utf8]{inputenc}

% ----- Vkladanie obrazkov -----
\usepackage{color, graphicx}
\graphicspath{{./images/}}

% ----- URL -----
\usepackage{url}
\usepackage[unicode]{hyperref} %odkazy vnutri dokumentu

% ----- Odd Even page -----
\usepackage{changepage}
\strictpagecheck

% ----- Zdrojové kódy -----
\usepackage{listings}
% info - http://en.wikibooks.org/wiki/LaTeX/Packages/Listings
% comments - http://lenaherrmann.net/2010/05/20/javascript-syntax-highlighting-in-the-latex-listings-package
\lstset{
%  language=bash,                  % the language of the code
  basicstyle=\footnotesize,       % the size of the fonts that are used for the code
  numbers=none,                   % where to put the line-numbers
  numberstyle=\tiny\color{gray},  % the style that is used for the line-numbers
  stepnumber=2,                   % the step between two line-numbers. If it's 1, each line will be numbered
  numbersep=5pt,                  % how far the line-numbers are from the code
  backgroundcolor=\color{white},  % choose the background color. You must add \usepackage{color}
  showspaces=false,               % show spaces adding particular underscores
  showstringspaces=false,         % underline spaces within strings
  showtabs=false,                 % show tabs within strings adding particular underscores
  frame=lines,                    % adds a frame around the code
  rulecolor=\color{black},        % if not set, the frame-color may be changed on line-breaks within not-black text (e.g. commens (green here))
  tabsize=2,                      % sets default tabsize to 2 spaces
  captionpos=b,                   % sets the caption-position to bottom
  breaklines=true,                % sets automatic line breaking
  breakatwhitespace=false,        % sets if automatic breaks should only happen at whitespace
  title=\lstname,                 % show the filename of files included with \lstinputlisting;
                                  % also try caption instead of title
  keywordstyle=\color{blue},      % keyword style
  stringstyle=\color{mauve},      % string literal style
  escapeinside={\%*}{*)},         % if you want to add a comment within your code
  morekeywords={*,...},            % if you want to add more keywords to the set
	extendedchars=true,
	literate={á}{{\'a}}1						%http://www.latexsearch.com/sandbox.do
	{é}{{\'e}}1
	{í}{{\'i}}1
	{ý}{{\'y}}1
	{ó}{{\'o}}1
	{ú}{{\'u}}1
	{ľ}{{\v{l}}}1
	{š}{{\v{s}}}1
	{č}{{\v{c}}}1
	{ť}{{\v{t}}}1
%	comment=[l]{\#},
%  commentstyle=\color{blue}
}
\lstset{morekeywords={select, where, order, by, limit, and, or, from, asc, desc, having, output, group, as}}
\renewcommand\lstlistingname{Výpis}
	
% ----- Zoznam skratiek -----
\usepackage[nolist,footnote]{acronym}

% ----- Okraje -----
\usepackage[top=2.5cm,left=2cm,right=3cm,bottom=2.5cm]{geometry}
%\usepackage[top=2.5cm,left=3cm,right=2cm,bottom=2.5cm]{geometry} %isis

\linespread{1.5}

%-----------------------------------------------------------------------------------------------------------------------------------------
%\textwidth100mm
%\marginparsep10mm
%\marginparwidth30mm


%\newlength{\fullwidth} % Width of text plus margin notes
%\setlength{\fullwidth}{\textwidth}
%\addtolength{\fullwidth}{\marginparsep}
%\addtolength{\fullwidth}{\marginparwidth}

%----------------------------------------------------------------------------------
% \myBigFigure	[ LABEL_PREFIX (optional) ]
%				{ FILENAME (without extension) }
%				{ CAPTION TEXT }
%				{ SHORT VERSION OF CAPTION TEXT }
%
%picture using full width of the page
\newcommand{\myBigFigure}[4][image]
{
\begin{figure}[t!bp]
	\checkoddpage
	\ifcpoddpage
		%nothing
	\else
		\hspace{-\marginparsep}\hspace{-\marginparwidth}
	\fi
	%use minipage to center the label beneath the figure
	\begin{minipage}{\fullwidth}
		\includegraphics[width= \fullwidth]{#2}
		\caption[#4]{#3}
		\label{#1_#2}
	\end{minipage}
\end{figure}
}


%----------------------------------------------------------------------------------
% \myFrameBigFigure	[ LABEL_PREFIX (optional) ]
%					{ FILENAME (without extension) }
%					{ CAPTION TEXT }
%					{ SHORT VERSION OF CAPTION TEXT }
%
%picture with frame using the full width of the page
\newcommand{\myFrameBigFigure}[4][image]
{
\begin{figure}[t!bp]
	\checkoddpage
	\ifcpoddpage
		%nothing
	\else
		\hspace{-\marginparsep}\hspace{-\marginparwidth}
	\fi
	%use minipage to center the label beneath the figure
	\begin{minipage}{\fullwidth}
	\frame{%
		\includegraphics[width= \fullwidth]{#2}%
		}
		\caption[#4]{#3}
		\label{#1_#2}
	\end{minipage}
\end{figure}
}

%----------------------------------------------------------------------------------
% \myHUGEFigure	[ LABEL_PREFIX (optional) ]
%				{ FILENAME (without extension) }
%				{ CAPTION TEXT }
%				{ SHORT VERSION OF CAPTION TEXT }
%
%landscape picture using the full width of the rotated page
\newcommand{\myHugeFigure}[4][image]
{
\begin{sidewaysfigure}[t!bp]
	
		\includegraphics[width= \textheight]{#2}
		\caption[#4]{#3}
		\label{#1_#2}
	
\end{sidewaysfigure}
}

%----------------------------------------------------------------------------------
% \myFigure	[ LABEL_PREFIX (optional) ]
%			{ FILENAME (without extension) }
%			{ CAPTION TEXT }
%			{ SHORT VERSION OF CAPTION TEXT }
%
%picture using the width of the text column
\newcommand{\myFigure}[4][image]%
{%
\begin{figure}[ht!bp]%
	\begin{center}%
		\includegraphics[width= \textwidth]{#2}%
		\caption[#4]{#3}
		\label{#1_#2}%
	\end{center}%
\end{figure}%
}%

%----------------------------------------------------------------------------------
% \myImgRef	[ LABEL_PREFIX (optional) ]
%			{ LABEL OF THE IMAGE }
%
%reference to an image
\newcommand{\myImgRef}[2][image]%
{%
	\ref{#1_#2}%
}%

%----------------------------------------------------------------------------------
% \myBigTable	{ YOUR TABULAR DEFINITION }
%			{ CAPTION TEXT }
%			{ TABLE_LABLE }
%
%table using the full width of the page
\newcommand{\myBigTable}[3]%
{%
\begin{table}[htdp]%
	\checkoddpage%
	\ifcpoddpage%
		%nothing
	\else%
		\hspace{-\marginparsep}\hspace{-\marginparwidth}%
	\fi%
	\begin{minipage}{\fullwidth}%
		\begin{center}%
			#1%
			\caption{#2}%
			\label{#3}%
		\end{center}%	
	\end{minipage}%
\end{table}%
}%

%----------------------------------------------------------------------------------
% \myTable	{ YOUR TABULAR DEFINITION }
%			{ CAPTION TEXT }
%			{ TABLE_LABLE }
%
%table using the width of the text column
\newcommand{\myTable}[3]%
{%
\begin{table}[htdp]%
	\begin{center}%
		#1%
		\caption{#2}%
		\label{#3}%
	\end{center}%	
\end{table}%
}%

%----------------------------------------------------------------------------------
% \myTxtRef	{ LABLE }
%
%references chapters or sections, outputs number and title, e.g., 5.3---"Yaddahyaddah"
\newcommand{\myTxtRef}[1]
{%
	\ref{#1}---``\nameref{#1}''%
}

%----------------------------------------------------------------------------------
% \myUnderscore
%
%typesets a 'nice' underscore for URLs
\newcommand{\myUnderscore}{$\underline{\hspace{0.5em}}$}

%----------------------------------------------------------------------------------
%\myTilde
%
%typesets a 'nice' tilde for URLs
\newcommand{\myTilde}{$\sim$}

%----------------------------------------------------------------------------------
% \myURL	{ TYPESET VERSION OF ANCHOR }
%			{ PRISTINE URL }
%			{ TYPESET VERSION OF URL }
%
%typesets a URL
%the typographically correct version appears as a footnote,
%the anchor appears in the text, the link points to the pristine URL
\newcommand{\myURL}[3]%
{%
	\textcolor{blue}{%
		\href{#2}{#1}%
	}%
	\footnote{#3}
}

%----------------------------------------------------------------------------------
% \mySimpleURL	{ TYPESET VERSION OF ANCHOR }
%				{ PRISTINE URL }
%
%typesets a URL
%the URL appears as a footnote,
%the anchor appears in the text, the link points to the URL
\newcommand{\mySimpleURL}[2]%
{%
	\textcolor{blue}{%
		\href{#2}{#1}%
	}%
	\footnote{#2}
}

%----------------------------------------------------------------------------------
% \myProjectURL	{ TYPESET VERSION OF ANCHOR }
%				{ PRISTINE URL INSIDE PROJECT DIRECTORY }
%				{ TYPESET VERSION OF URL INSIDE PROJECT DIRECTORY }
%
%typesets a URL to hci/public from where the contents of the WebServer folder from oliver can be accessed
%the typographically correct version appears as a footnote,
%the anchor appears in the text, the link points to the pristine URL
\newcommand{\myProjectURL}[3]%
{%
	\textcolor{blue}{%
		\href{http://hci.rwth-aachen.de/public/#2}{#1}%
	}%
	\footnote{http://hci.rwth-aachen.de/public/#3}%
}

%----------------------------------------------------------------------------------
% \mnote	{ MARGIN NOTE }
%
%puts a comment into the margin in small sans-serif font
\newcommand{\mnote}[1]{\marginpar{\raggedright\textsf{{\footnotesize{#1}}}}}

%----------------------------------------------------------------------------------
% \todo	{ TODO MARGIN NOTE }
%
%puts a 'todo' comment into the margin in red
\definecolor{red}{rgb}{1,0,0}
\newcommand{\todo}[1]{\mnote{\textcolor{red}{ToDo: #1}}}

%----------------------------------------------------------------------------------
% \chapterquote	{ QUOTATION }
%				{ SOURCE }
%
%outputs a quote with its source, can be used as an introduction to chapters
\newcommand{\chapterquote}[2]{
\begin{quotation}
    \begin{flushright}
	\noindent\emph{``{#1}''\\[1.5ex]---{#2}}
    \end{flushright}
\end{quotation}
}

%----------------------------------------------------------------------------------
% \myDefBox	{ TERM }
%			{ DEFINITION }
%
%outputs a margin note and a colored box (width of the text column) containing a term and its definition
\newcommand{\myDefBox}[2]
{%
	\setlength{\fboxrule}{1mm}%
	\fcolorbox{orange_med}{orange_light}%
	{%
		\parbox{\myDefBoxWidth}{{\bfseries\scshape#1:}\\#2}%
	}%
	\mnote{Definition:\\\emph{#1}}
}

%----------------------------------------------------------------------------------
% \myBigDefBox	{ TERM }
%				{ DEFINITION }
%
%outputs a colored box (width of the page) containing a term and its definition
\newcommand{\myBigDefBox}[2]
{%
	\begin{figure}[h!]
	\setlength{\fboxrule}{1mm}%
	\checkoddpage%
	\ifcpoddpage%
		%nothing
	\else%
		\hspace{-\marginparsep}\hspace{-\marginparwidth}%
	\fi%
	\fcolorbox{orange_med}{orange_light}%
	{%
		\parbox{\myBigDefBoxWidth}{{\bfseries\scshape#1:}\\#2}%
	}%
	\end{figure}
}

%----------------------------------------------------------------------------------
% \myDownloadURL	{ TYPESET DOWNLOAD NAME }
%					{ PRISTINE VERSION OF FILENAME }
%					{ TYPESET VERSION OF FILENAME }
%
%outputs a colored box containing a download link
\newcommand{\myDownloadURL}[3]{%
\checkoddpage%
	\ifcpoddpage%
		%nothing
	\else%
		\hspace{-\marginparsep}\hspace{-\marginparwidth}%
	\fi%
\setlength{\fboxrule}{1mm}%
\fcolorbox{green_med}{green_light}{%
\begin{minipage}{\myBigDefBoxWidth}%
\begin{center}%
\myProjectURL{#1}{folder/#2}{folder/#3}%
\end{center}%
\end{minipage}%
}%
}

%----------------------------------------------------------------------------------
% \emptydoublepage
%
% Clear double page without any header or footer at end of chapters
\newcommand{\emptydoublepage}{\clearpage\thispagestyle{empty}\cleardoublepage}

%----------------------------------------------------------------------------------
% \pagebreak	[ SOME STRANGE LATEX VALUE ]
%
%pagebreaks for the final print version (last resort weapon against wrong pagebreaks by LaTeX)
\newcommand{\PB}[1][3]
{%
	\pagebreak[#1]%
}

\begin{document}

% ----- Info o práci -----
%\title{Esper Web: Webové rozhranie pre spracovanie udalostí v reálnom čase}
\title{Esper Web: Webové rozhraní pro zpracování událostí v reálném čase}

\author{Martin Kravec}
\date{Marec 23, 2015}

%--------------------------------------------------------------
\frontmatter

\maketitle \thispagestyle{empty} \emptydoublepage

\chapter*{Prehlásenie}
Prehlasujem, že som diplomovú prácu spracoval samostatne, a že som uviedol všetky použité pramene a literatúru, z ktorej som čerpal.

\emptydoublepage

\chapter*{Poďakovanie}
Chcem poďakovať všetkým, ktorí ma podporovali a pomáhali mi, najmä svojim rodičom a starým rodičom za trpezlivosť a podporu pri štúdiu. V neposlednom rade ďakujem vedúcemu mojej práce, Mgr. Zbyňkovi Šlajchrtovi za cenné rady, ktoré mi dal.

\emptydoublepage

%\chapter*{Abstract\markboth{Abstract}{Abstract}}
%\addcontentsline{toc}{chapter}{\protect\numberline{}Abstract}
%\label{abstract}

\chapter*{Abstrakt}
Táto diplomová práca sa zaoberá problematikou spracovania komplexných udalostí. V teoretickej časti nájdeme vysvetlenie základných pojmov a popis použitých technológií. Čitateľ sa tiež zoznámi so základmi práce v Esperi.

Praktická časť práce sa zaoberá vytvorením administračného rozhrania, ktoré po napojení na serverovú časť aplikácie umožní použitie základnej funkcionality Esper engine aj bez predchádzajúcej znalosti programovania. Súčasťou praktickej časti je postup prípravy systému, inštalácie a spustenia aplikácie, keďže ide o netriviálnu úlohu.

\section*{Kľúčové slová}
cep, nosql, esper, cassandra, rest, rails, gui \emptydoublepage

\chapter*{Abstract}
This thesis deals with the complex event processing problem. In theoretical part you can find explanation of basic terms and description of used technologies. Reader also becomes fammiliar with basics of work with Esper.

Practical part is about building administrative interface, which after connecting to server part of application enables user to use basic functionality of Esper engine without previous programming knowledge. Part of this chapter is about system set-up, installation and launching the application as it is not considered to be a trivial task.

\section*{Keywords}
cep, nosql, esper, cassandra, rest, rails, gui \emptydoublepage

\tableofcontents \emptydoublepage

%--------------------------------------------------------------
\mainmatter

%Reaktivni programovani
Statement - zobraz vsetky tweety od kontaktu s menom..

Problemy
- nutna verzia java-1.7\_72 kvoli chybe pri nacitani xml pouzitim integrovanej jre kniznice
- her
- results table must have statement\_id pretoze by vznikli duplicity pri viacerych pouzivateloch ak by bolo pouzite statement.name - preto musi mat statement id nie name
- eventy typu xml nie je mozne prijimat cez esperio

\chapter*{Úvod}
\addcontentsline{toc}{chapter}{\protect\numberline{}Úvod}

Rýchly nárast množstva dát produkovaných užívateľmi a aplikáciami prináša problémy s ich spracovaním a vyhodnocovaním. Pri analýze statických dát z bežnej databázy narážame na obmedzenia spôsobené dávkovým spracovaním dotazov. Technológia spracovania \ac{CEP} prináša možnosť spracovania prúdov informácií a ich komplexných závislostí v reálnom čase. Definuje pojem udalosť, ktorá predstavuje jednotku informácie s ktorou systém pracuje. Bližšie sa touto technológiou zaoberá kapitola \ref{chap:pojmy}.

Prvotným cieľom tejto diplomovej práce je vytvorenie grafického nástroja, ktorý umožní vykonávanie základných operácií potrebných pre prácu s esper engine. Tými sú správa schém a príkazov, odosielanie udalostí a zobrazenie nájdených výsledkov. Použitím tohoto nástroja odpadá užívateľovi potreba znalosti programovacích platforiem java a net, pre ktoré je Esper oficiálne dostupný.

Druhotným cieľom je rozdelenie väzby medzi administračným rozhraním a esper engine a umožnenie prístupu k základným operáciám definovaným v prvom cieli pomocou restovej api. Toto rozdelenie umožní vytváranie ďalších aplikácií, ktoré budú využívať esper engine taktiež bez nutnosti znalosti programovania v jave alebo NET platforme.

So splnením druhého cieľa je tiež viazaný koncept deklaratívnej tvorby webových aplikácií. Bežne sa stretneme s imperatívnym programovacím prístupom, kde je funkcionalita riešená presne definovanými algoritmami, teda postupom ako danú úlohu vyriešiť. Deklaratívne programovanie naproti tomu hovorí čo sa má vykonať, nie ako sa to má vykonať.
Využitie restovej api pre komunikáciu s esper engine umožňuje vytvárať jednoduché webové aplikácie, ktorých dátová časť je riešená deklaratívne. Pred použitím je nutné definovať schému dát napríklad pomocou administračného rozhrania a nastaviť príkazy, ktoré budú zachytávať a filtrovať prichádzajúce dáta. Následne sa stačí odosielať nové záznamy na definovanú URL. Tie vyhovujúce definovaným filtrom budú prístupné vo výsledkoch daného príkazu.

Tretím cieľom práce je umožnenie práce s historickými dátami. Jedným spôsobom realizácie bude možnosť odosielania súborov obsahujúcich dáta vo formáte XML. Druhým zaujímavejším riešením bude možnosť presmerovania výsledkov konkrétneho príkazu na vstup esper engine ako nový zdroj dát.

Súčasťou práce je úvod do problematiky NoSQL databáz, ktorých použitie je vhodné hlavne pri spracovaní veľkého množstva dát. Keďže esper engine je stavaný na takéto úlohy bude pre ukladanie výsledkov použitá práve NoSQL databáza.

Použitie diplomovej práce vyžaduje základnú znalosť esper syntaxe a je koncipovaná ako pre začiatočníkov, ktorí môžu k esperu pristupovať bez znalosti javy tak pre pokročilých užívateľov, ktorým môže uľahčiť a sprehľadniť prácu. Súčasťou práce je tiež postup prípravy systému a inštalácia, keďže tieto úkony nie sú triviálne.

Pri písaní práce som čerpal prevažne z online dokumentácie esperu a ruby on rails frameworku. Tiež som využil bakalársku prácu Štefana Repčeka \cite{bp-repcek} a diplomovú prácu Jána Dema \cite{dp-demo} ako množstvo iných dostupných materiálov dostupných prevažne online.

\emptydoublepage
\chapter{Definícia pojmov}
\label{chap:pojmy}

\section{CEP}
	\ac{CEP} je definovaný ako sada nástrojov a techník na analýzu a kontrolu komplexného sledu vzájomne prepojených udalostí, na ktorých sú postavené moderné distribuované informačné systémy. Táto technológia pomáha ľuďom v IT obore rýchlo identifikovať a vyriešiť mnohé problémy. \cite{power-of-events}

	\ac{CEP} je dôležitou súčasťou vývoja reaktívnych aplikácií a monitorovacích programov. Použitie nachádza v mnohých oblastiach, napríklad pre automatické obchodné systémy, detekcie podvodov, analýzu sentimentu trhu, optimalizáciu dodávateľského reťazca, transporte a logistike alebo systémoch pre rýchlu pomoc. Predpokladá sa tiež nárast jeho využitie v informačných systémoch spolu so zvyšujúcim sa počtom decentralizovaných zdrojov dát ako blogov a rozvojom tagovacích a senzorových technológií.

	\ac{CEP} stavia na dvoch nutných podmienkach: \cite{web:cep-editorial}
	\begin{itemize}
		\item Oddeľuje tvorcov a príjemcov informácií. Tvorcovia nevyžadujú informácie o príjemcoch, rovnako ako príjemci nepotrebujú vedieť kto dáta produkuje.
		\item CEP systémy okrem predávania informácií medzi tvorcami a príjemcami vo forme udalostí, ale umožňujú detekciu vzťahov medzi udalosťami.
	\end{itemize}
	
	Príkladom takého vzťahu je dočasný vzťah definovaný pomocou korelačných pravidiel (event pattern). Pomocou agregácie a kompozície je možné generovať nové udalosti a z nich ďalej odvodiť ďalšie udalosti. Korelácia udalostí navyše pomáha redukovať množstvo dát a pomáha tak budovať škálovateľné systémy.
	
	\ac{CEP} sa podobne ako iné technológie vyvíjal v čase a je postavený na základe definovanom \ac{SEP} a \ac{ESP}.
	\begin{itemize}
		\item SEP je najjednoduchším prípadom, kde je udalosť spracovaná izolovane bez ohľadu na ostatné udalosti.
		\item ESP spracováva prúdy udalostí ako kolekcie, identifikuje typy udalostí použitím kontinuálnych výrazov a selektuje zaujímavé udalosti.
		\item CEP je rozšírenie ESP o mechanizmy \ac{ECA} umožňujúce vykonávanie definovaných príkazov v závislosti na spracovávanej udalosti vyhovujúcej stanoveným podmienkam.
	\end{itemize}
	
	Pre prácu s komplexnými udalosťami je potrebné najprv pochopiť čo sa pod týmto pojmom skrýva. V ďalšom texte aj pri samotnej práci v administračnom rozhraní sa stretneme s typmi udalostí a samotnými udalosťami. Tieto pojmy sú definované nasledovne \cite{etzion2011event}:
	
	\begin{description}
		\item[Udalosť] je definovaná ako výskyt v danom systéme alebo doméne. Je to niečo čo sa stalo, alebo je predpokladané že sa stalo v danej doméne. Slovo udalosť je tiež používané vo význame programátorského objektu, ktorý reprezentuje výskyt udalosti v počítačovom systéme.
		
		\item[Typ udalosti] je špecifikáciou pre skupinu objektov udalostí, ktoré majú rovnaký účel a štruktúru. Každý objekt udalosti je považovaný za inštanciu typu udalosti.
	\end{description}

	Pred samotným spracovaním udalostí je nutné definovať ich typ, pravidlá spracovania a spracovanie výstupu. Po vykonaní týchto úkonov stačí na vstup procesného engine posielať udalosti daného typu. Model fungovania takéhoto systému vidíme na obrázku \ref{image_cep-model}. Na rovnakom princípe funguje serverová časť programu, ktorou sa bude bližšie zaoberať kapitola \ref{chap:implementacia}.
	\myFigure{cep-model}{Model spracovania udalostí \cite{web:softwarearchitekturen}}{Model spracovania udalostí}
	
	Na rozdiel do databázových systémov sú dotazy na do prúdu udalostí vyhodnocované priebežne, v momente výskytu danej udalosti. Aj keď databázy zvyčajne pracujú s údajmi o udalostiach (napríklad históriou objednávok), dotazy v databázach sú jednorázové a ad-hoc, oproti konečnej množine dát. Naopak dotazy v CEP sú predom definované a množinou dát je nekonečný tok udalostí cez nich prechádzajúci \cite{web:ceptranslated}.
		
	Samotné spracovanie udalostí je postavené na logike, ktorú definujú pravidlá spracovania (statementy). Tieto pravidlá si môžeme predstaviť ako filter, cez ktorý prichádzajúce udalosti pretekajú. Ak udalosť filtru vyhovuje, prevedú sa vopred definované akcie a udalosť je odoslaná na výstup.	Tieto pravidlá je možné definovať pomocou jazyka EPL, ktorý je syntaxou veľmi podobný SQL.
	
	EPL umožňuje definovať dátové okná, ktoré predstavujú obdobu pohľadov (view) ako ich poznáme z databázových prostredí. CEP sa ale zaoberá prácou s tokmi dát, nie statickým pohľadom na nich prináša EPL možnosť rozšíriť tieto náhľady o definíciu rozsahu skúmaných dát. Príkladom použitia takejto definície je definícia dátového okna, ktoré bude udržiavať špecifický počet prijatých udalostí, prípadne udalosti prijaté v istom časovom rozsahu. Príklad takýchto obmedzení vidíme na obrázku \ref{image_cep-windows}. Bližšie príklady EPL syntaxe nájdeme v kapitole \ref{chap:technologie}.

	\myFigure{cep-windows}{Príklad dátového okna obmedzeného časom a počtom udalostí \cite{web:softwarearchitekturen}}{Príklad dátového okna obmedzeného časom a počtom udalostí}
	
	Detekcia udalostí však sama o sebe dostačujúca, systémy riadené udalosťami tiež automaticky reagujú na detekované udalosti podľa predom stanovených pravidiel. Tie zvyčajne pozostávajú z notifikácie (užívateľa alebo ďalšieho systému), jednoduchých akcií (automatický nákup akcií, aktivácia požiarneho systému) alebo interakciou s iným systémom (spustením nového procesu).

	Na trhu existuje viacero komerčných CEP riešení od firiem ako Tibco, Oracle či IBM. Zo zástupcov open-source produktov sú to napríklad JBoss Drools Fusion alebo Esper od firmy EsperTech, na ktorom bude postavená aj táto diplomová práca.

	%From \cite{web:cep-editorial}.
	%http://www.tibco.com/blog/2009/08/21/cep-versus-esp-an-essay-or-maybe-a-rant/

\section{NoSQL}
	Priekopníkom vzniku NoSQL databázy boli vedúce internetové spoločnosti ako Google, Facebook, Amazon a LinkedIn - prekonávali tak limitácie konceptu relačných databáz pri použití v moderných webových aplikáciách. Dnes používajú organizácie NoSQL na riešenie problémov, ktoré priniesli štyri trendy:
	\begin{description}
		\item[Big Users] Pred pár rokmi bolo 1000 užívateľov konkrétnej aplikácie veľa, 10 000 užívateľov v extrémnych prípadoch. Dnes sú pripojených k internetu 2 bilióny ľudí, ktorí strávia online okolo 35 biliónov hodín mesačne. Pre webové aplikácie teda nie je výnimočné mať milióny rôznych užívateľov denne.
		
		\item[Big Data] Obrovský nárast používania internetu vyúsťuje v nárast údajov, ktoré užívatelia a aplikácie produkujú. Podľa odhadu bola v roku 2013 veľkosť uložených údajov 4,4 zetabajtov s predpokladom exponenciálneho rastu. Do roku 2020 sa predpokladá desať násobný nárast \cite{web:idc-bigdata}. Užívateľské informácie, geografické dáta, sociálne grafy, údaje vyprodukované užívateľmi a aplikáciami a senzorové dáta sú príkladom nekonečných generátorov dát.
		
		\item[Internet of Things] je trend, ktorý je definovaný stále rastúcim množstvom prepojených zariadení - ktoré generujú dáta. Dnes je k internetu pripojených okolo 20 biliónov zariadení, vrátane telefónov, tabletov, zariadení v nemocniciach, autách či skladoch. Tieto zariadenia získavajú údaje o svojom okolí, pohybe alebo počasí z ich 50 biliónov senzorov.
		Napríklad telemetrické údaje, ktoré sú čiastočne štruktúrované a kontinuálne a pre SQL databázy s pevne definovanou schémou a štrukturovaným spôsobom ukladania dát predstavujú problém.
		
		\item[Cloud Computing] Množstvo aplikácií je dnes postavených na cloud infraštruktúre a využíva trojvrstvovú architektúru. V tej je k aplikáciám pristupované cez internet pomocou mobilnej aplikácie internetového prehliadača. Load balancery zodpovedajú za rozloženie záťaže a smerujú prichádzajúce požiadavky na jednotlivé servery, ktoré obsluhujú logiku aplikácie. Pri rastúcej záťaži nie je problémom pridať do konfigurácie load balancera ďalší server a takto rozložiť záťaž.
		Problém nastáva v databázovej vrstve. Relačné databázy boli zvyčajne pôvodným riešením, avšak ich použitie je čoraz viac problematické pretože databáza je centralizovaná a škálovateľná vertikálne, nie horizontálne. To predstavuje nevýhodu pre aplikácie rastúce dynamicky. NoSQL databázy sú distribuované a škálovateľné horizontálne, čo umožňuje podobne ako pri load balanceroch jednoducho pridať ďalší server a rozložiť záťaž.
	\end{description}
	
	NoSQL používa rozdielny dátový model ako relačné databázy. Relačný model rozprestrie dáta do viacerých vzájomne prepojených tabuliek obsahujúcich riadky a stĺpce. Pri získavaní informácií z relačnej databázy musia byť údaje z týchto tabuliek spojené. Podobne pri zápise musia byť údaje rozložené do jednotlivých tabuliek.

	Naproti tomu NoSQL dátový model agreguje ukladané dáta do jedného objektu - napríklad dokumentu v prípade objektových databáz. V NoSQL nie je definovaná operácia JOIN, čo vedie k duplikácií informácií vo viacerých tabuľkách. Tento problém však vyvažuje dnes lacný úložný priestor, flexibilita dátového modelu, zlepšený výkon operácií čítania a zápisu a škálovateľnosť tohoto systému.
	
	
	Pre vyriešenie problémov s rastúcim počtom užívateľov a údajov je nutné aplikácie škálovať - a to horizontálne alebo vertikálne.
	\begin{itemize}
		\item Vertikálne škálovanie predstavuje centralizovaný prístup založený na vylepšovaní konfigurácie serverov, prípadne ich nahradení výkonnejšími servermi.
		\item Horizontálne škálovanie predstavuje distribuovaný prístup, kde sú do konfigurácie pridané ďalšie servery, ktoré umožňujú rozloženie záťaže.
	\end{itemize}
	Pred príchodom NoSQL bolo obvyklé vertikálne škálovanie. Pri raste množstva dát boli potrebné výkonnejšie servery s väčšou RAM, výkonnejšími procesormi a väčším diskovým priestorom. Cena a komplikovanosť takýchto serverov však neúmerne stúpa so zvyšujúcimi sa požiadavkami, ako je možné vidieť na obrázku \ref{image_scale-up-out}. Po istej úrovni už nie je možné konfiguráciu serveru vylepšiť, je nutné zaobstarať ďalší a rozloženie zaťaženia databázy riešiť na aplikačnej úrovni, čo je veľmi náročná úloha pre programátorov a správcov databázy. 
	
	\myFigure{scale-up-out}{Škálovanie aplikácie vzhľadom na jej cenu}{Škálovanie aplikácie vzhľadom na jej cenu}
	
	\ac{NoSQL} zahrňuje širokú škálu rôznych databázových technológií v reakcii na prudký nárast množstva ukladaných dát, frekvencie prístupu k údajom a výkonnostným požiadavkom, ktoré z tohoto nárastu  vyplývajú.
	
	Relačné databázy neboli navrhnuté so zreteľom na požiadavky a nároky s ktorými sa dnešné aplikácie stretávajú. Tiež neumožňujú využiť výhody lacných úložísk dát a výpočtového výkonu.
	
%	http://www.mongodb.com/nosql-explained
%	http://www.couchbase.com/nosql-resources/what-is-no-sql
%	http://www.go-gulf.com/blog/online-time/

\chapter{Technológie}
\label{chap:technologie}

\section{Esper}
	Esper je komponenta, ktorá umožňuje spracovanie komplexných udalostí \ac{CEP}. Umožňuje vývoj aplikácií spracovávajúcich veľké množstvo udalostí - v reálnom čase ako aj historických. Tieto udalosti je možné filtrovať a analyzovať podľa potreby a reagovať v reálnom čase na predom definované stavy.  Esper je dostupný v troch verziách:
	\begin{description}
		\item[Esper] je open source s možnosťou komerčnej podpory. Táto verzia obsahuje základ potrebný pre realizáciu CEP, užívateľ však musí jednotlivé príkazy, schémy a nastavenia realizovať programovo. Je preto náročný na použitie pre ľudí, ktorí nevedia programovať. Riešenie je vhodné pre firmy, ktoré buď nevyužijú platenú verziu alebo majú špecifické požiadavky na výsledný produkt a sú schopné túto verziu podľa svojich potrieb upraviť.
		
		\item[Esper HA] je riešenie umožňujúce vysokú dostupnosť Esperu. Zabezpečuje že stav je po vypnutí alebo havárii obnoviteľný. Príkazy, schémy a iné nastavenia si EsperHA pri reštarte uchováva, čo je výhoda oproti open source verzii, kde je nutné tieto úkony riešiť programovo. Táto verzia je vhodná pre projekty závislé na vysokej dostupnosti Esperu a subjekty, pre ktoré je kritická neustála kontrola prichádzajúcich udalostí.
		EsperHA je spoplatnený, dostupná je trial len verzia, pre ktorej použitie je nutné identifikovať sa ako spoločnosť. Cena nie je na webových stránkach dostupná.
		
		\item[Esper Enterprise Edition] je kompletný produkt ``na kľúč'', obsahujúci všetky komponenty potrebné pre nasadenie do podniku. V jednom balíku je obsiahnuté GUI pre správu Esperu, restové služby poskytujúce prístup zvonku, \ac{EPL} editor, nástroje umožňujúce kontinuálne zobrazenie výsledkov v grafoch a tabuľkách. EsperEE je možné skombinovať s EsperEA pre dodatočné zabezpečenie vysokej dostupnosti. EsperEE je spoplatnený, rovnako ako pri EsperEA je dostupná trial verzia po splnení určitých podmienok. Cena nie je zverejnená a tieto dve riešenia sú určené predovšetkým pre podnikový sektor.
	\end{description}
	
	Pre tento projekt je použitá verzia Esper, ktorú som rozšíril o prístup k základným funkciám pomocou restovej api a persistenciu niektorých nastavení a nájdených výsledkov. Aktuálna verzia 5.1 je dostupná pod GNU General Public License (GPL) (GPL v2).

	\subsection{Typy udalostí}
	Každá udalosť spracovávaná Esperom je definovaná schémou, takzvaným typom udalosti. Tie môžu byť načítané pri štarte aplikácie, alebo nastavené programovo počas behu. EPL obsahuje klauzulu CREATE SCHEMA umožňujúcu definovanie typu udalosti. Prehľad základných typov udalostí je v tabuľke \ref{table:event-types}.

	\myTable{
	\begin{tabular}{ | l | p{10cm} | }
		\hline
		Trieda	&	Popis	\\ \hline
		java.lang.Object	&	Akýkoľvek Java \ac{POJO} s getter metódami. Takáto definícia je najjednoduchšia na úkor možnosti úprav počas behu programu.	\\ \hline
		java.util.Map	&	Udalosti definované ako implementácia java.util.Map interface, kde každá hodnota záznamu je vlastnosť udalosti.	\\ \hline
		Object[] (pole objektov)	&	Udalosti definované objektovým poľom, kde každá hodnota poľa je vlastnosť udalosti.	\\ \hline
		org.w3c.dom.Node	&	XML objektový model dokumentu popisujúci štruktúru udalosti.	\\ \hline
	\end{tabular}
	}{Možnosti definície typu udalosti}{table:event-types}
	
	Definície typu udalosti sú rozšíriteľné zásuvnými modulmi. Aplikácia môže používať kombináciu týchto typov, nemusí všetky typy definovať jedným spôsobom. Definície typov udalostí je možné reťaziť, kde typom udalosti môže byť iná komplexná udalosť.
	
	Z dôvodu nutnosti pridávania a mazania udalostí počas behu programu nemôže byť v tejto implementácii použitá definícia typu pomocou POJO. A pretože klient musí mať možnosť definovať typ, bola zvolená definícia pomocou XML schémy. Príklad jednoduchej schémy udalosti znázorňuje výpis \ref{lst:sample-schema}.
	
	\begin{lstlisting}[label=lst:sample-schema,caption=Príklad XML schémy udalosti]
	<?xml version="1.0" encoding="UTF-8"?>
	<xs:schema xmlns:xs="http://www.w3.org/2001/XMLSchema">
		<xs:element name="StockEvent">
			<xs:complexType>
				<xs:sequence>
			        <xs:element name="time" type="xs:string"></xs:element>
			        <xs:element name="open" type="xs:float"></xs:element>
			        <xs:element name="high" type="xs:float"></xs:element>
			        <xs:element name="low" type="xs:float"></xs:element>
			        <xs:element name="close" type="xs:float"></xs:element>
			        <xs:element name="volume" type="xs:float"></xs:element>
				</xs:sequence>
			</xs:complexType>
		</xs:element>
	</xs:schema>	
	\end{lstlisting}
	
	Po definovaní typu udalosti je Esper engine schopný prijímať udalosti v XML formáte. Príklad udalosti vyhovujúcej schéme \ref{lst:sample-schema} je vo výpise \ref{lst:sample-event}.
	\begin{lstlisting}[label=lst:sample-event,caption=Príklad XML udalosti]
	<?xml version="1.0"?>
	<events>
		<StockXsd>
			<time>2014-02-03 01:58:00.000</time>
			<open>1.34850</open>
			<high>1.34854</high>
			<low>1.34850</low>
			<close>1.34853</close>
			<volume>76.9400</volume>
		</StockXsd>
		<StockXsd>
			<time>2014-02-03 01:59:00.000</time>
			<open>1.34852</open>
			<high>1.34853</high>
			<low>1.34845</low>
			<close>1.34850</close>
			<volume>89.5800</volume>
		</StockXsd>
	</events>
	\end{lstlisting}
	Ako je z výpisu vidieť je možné udalosti zaobaliť do koreňového elementu events. Ten je vhodné použiť v prípade že na Esper engine odosielame súbory obsahujúce veľké množstvo udalostí, pretože tým obmedzíme počet HTTP volaní a predídeme možnému zahlteniu serveru. Táto funkcionalita je realizovaná v implementačnej časti aplikácie a nie je súčasťou Esperu.

	Po definovaní typu udalosti sa na ne môžeme odkazovať klauzulou FROM v EPL príkazoch. Tie sú bližšie popísané v nasledujúcej sekcii.

	\subsection{Event Processing Language}
		\ac{EPL} je jazyk umožňujúci definovanie príkazov a vzorov v CEP. Syntaxou je podobný SQL, pretože obsahuje klauzuly ako SELECT, FROM, WHERE, GROUP BY, HAVING alebo ORDER BY. Namiesto tabuliek však pracuje so tokmi udalostí, kde riadok tabuľky nahrádza prichádzajúca udalosť. Toky udalostí je možné spájať pomocou JOIN, filtrovať alebo agregovať.
		
		EPL definuje koncept pomenovaných okien (named windows), ktoré slúžia ako štruktúra uchovávajúca udalosti. Je možné do nej vkladať nové udalosti a mazať staré. Výhodou tejto štruktúry je možnosť jej použitia viacerými príkazmi, pretože je globálna, teda zdieľaná v rozsahu daného service providera.	
		
		Pomocou EPL môžeme tiež definovať premenné, ktoré sa dajú následne použiť napríklad na vkladanie parametrov do príkazov.
		
		\subsubsection{Syntax}
		Príkazy musia spĺňať pravidlá definované EPL syntaxou. Tá však nie je jednotná a jednotlivé CEP riešenia poskytujú svoje implementácie. Táto časť práce sa zaoberá pravidlami, ktoré používa jazyk EPL Esperu. Výpis \ref{lst:epl-syntax} zobrazuje štruktúru tejto syntaxe.
		
		\begin{lstlisting}[label=lst:epl-syntax,caption=Vzor EPL syntaxe \cite{web:esper-doc}]
		[annotations]
		[expression_declarations]
		[context context_name]
		[into table table_name]
		[insert into insert_into_def]
		select select_list
		from stream_def [as name] [, stream_def [as name]] [,...]
		[where search_conditions]
		[group by grouping_expression_list]
		[having grouping_search_conditions]
		[output output_specification]
		[order by order_by_expression_list]
		[limit num_rows]
		\end{lstlisting}

		Ako môžeme vidieť v tomto výpise, každý EPL príkaz musí obsahovať minimálne klauzuly SELECT a FROM. Ďalej je možné filtrovať pomocou klauzuly WHERE, spájať prúdy udalostí pomocou JOIN alebo využiť relačnú databázu ako zdroj udalostí. Nasledujúci popis rozoberá základné EPL klauzuly \cite{web:esper-doc}.
		
		\begin{description}
			\item[Select] Klauzula SELECT je povinná v každom EPL príkaze. Je v nej možné využiť náhradný znak * alebo vymenovať všetky požadované položky. Ak položka nemá unikátne meno, musí sa použiť predpona s názvom zdroja dát. 
			
			V prípade použitia znaku * v JOIN príkaze nebude výsledná udalosť obsahovať všetky položky oboch zdrojov. Namiesto toho bude pozostávať z položiek reprezentujúcich objekty daných udalostí pomenované podľa zdrojov.
			
			Syntax select klauzuly je znázornená vo výpise \ref{lst:select-syntax}. Môže obsahovať aj nepovinné parametre istream (input), irstream (input \& remove) a rstream (remove), ktoré definujú na ktoré udalosti príkaz reaguje. Prednastavené je použitie parametru istream.

			\begin{lstlisting}[label=lst:select-syntax,caption=Syntax SELECT klauzuly \cite{web:esper-doc}]
select [istream | irstream | rstream] [distinct] * | expression_list
			\end{lstlisting}
			Obsah klauzuly select tiež definuje typ udalostí vyprodukovaných daným príkazom.
			
			\item[FROM] FROM klauzula špecifikuje jeden alebo viac zdrojov, pomenovaných okien alebo tabuliek (od verzie Esper 5.1). Tie môžu byť pomenované klauzulou AS. Pre join je potrebné definovať viacero zdrojov dát. Syntax from klauzuly je vo výpise \ref{lst:from-syntax}.
			\begin{lstlisting}[label=lst:from-syntax,caption=Syntax FROM klauzuly \cite{web:esper-doc}]
from stream_def [as name] [unidirectional]
	[retain-union | retain-intersection] 
[, stream_def [as stream_name]] [, ...]
			\end{lstlisting}
			Podporovaný je tiež join s relačnou databázou ako zdrojom dát. To je možné využiť napríklad na prístup k historickým dátam.
			
			\item[WHERE] Where klauzula je nepovinná časť príkazu, ktorá špecifikuje vyhľadávacie parametre. Zvyčajne obsahuje výrazy pozostávajúce z porovnávacích operátorov =, \textless , \textgreater , \textgreater=, \textless=, !=, \textless\textgreater, exists, is null a ich kombinácie pomocou kľúčových slov AND a OR.
			\begin{lstlisting}[label=lst:where-syntax,caption=Syntax WHERE klauzuly \cite{web:esper-doc}]
where exists (
	select orderId from Settlement.win:time(1 min) 
		where settlement.orderId = order.orderId
)
			\end{lstlisting}
			Klauzula where môže obsahovať tiež vnorené výrazy, ako je to znázornené vo výpise \ref{lst:where-syntax}.
			
			\item[JOIN] Klauzula FROM môže obsahovať viacero zdrojov dát. V tom prípade sú dátové zdroje spojené pomocou JOIN. Predvolene je použitý inner join, ktorý produkuje udalosti len v prípade výskytu vyhovujúcej udalosti vo všetkých zdrojoch. V prípade použitia outer join sa chýbajúce udalosti nahradia hodnotou null.
	
			K dispozícii sú tiež varianty left outer join, right outer join a full outer join. Výpis \ref{lst:join-syntax} znázorňuje pravidlá syntaxe pri použití join.
			\begin{lstlisting}[label=lst:join-syntax,caption=Syntax JOIN klauzuly \cite{web:esper-doc}]
...from stream_def [as name] 
((left|right|full outer) | inner) join stream_def 
[on property = property [and property = property ...] ]
[ ((left|right|full outer) | inner) join stream_def [on ...]]...
			\end{lstlisting}
			Každý z prúdov dát definovaný pomocou join klauzuly obsahuje vstupný a výstupný stream. Join tak môže byť realizovaný pri prijatí udalosti v ktoromkoľvek z týchto prúdov. Join je teda viacsmerový, prípadne dvojsmerný pri použití dvoch prúdov dát.
			EPL definuje kľúčové slovo unidirectional, ktoré umožňuje identifikovať jediný prúd dát poskytujúci udalosti ktoré spustia join. Všetky ostatné prúdy sa stanú pasívnymi. Keď je prijatá udalosť pasívnym prúdom dát, negeneruje join novú udalosť.
	
			\item[OUTPUT] Output klauzula umožňuje kontrolovať rýchlosť ktorou sú produkované udalosti. Zvyčajne sa používa spolu s určením časového údaju. Tým môže byť napríklad výstup každých n sekúnd, n udalostí alebo v daný čas dňa. Časy výstupu je možné definovať tiež vo formáte cronu. Jeden zo spôsobov zápisu zobrazuje výpis \ref{lst:output-syntax}.
			\begin{lstlisting}[label=lst:output-syntax,caption=Syntax OUTPUT klauzuly \cite{web:esper-doc}]
output [after suppression_def] 
[[all | first | last | snapshot] every output_rate [seconds | events]]
			\end{lstlisting}
			V tomto výpise vidíme aj možnosť definovania toho čo sa má vyprodukovať. Je možné si vybrať produkovanie všetkých udalostí, prvej, poslednej alebo snímky. Snímka sa používa spolu s agregačnými funkciami a produkuje jedinú udalosť s hodnotou agregačnej funkcie.
		\end{description}

		Esper navyše umožňuje presmerovávať toku udalostí, prípadne ich za behu upravovať nasledujúcimi klauzulami:
		\begin{description}
			\item[INSERT INTO] Túto klauzulu je možné použiť pre vloženie výsledkov príkazu do pomenovaného okna alebo tabuľky. Tiež umožňuje presmerovať tieto výsledky ako vstupný tok pre iný príkaz. Syntax pre použitie klauzuly insert into je zobrazená vo výpise \ref{lst:insertinto-syntax}. Príklad použitia je dostupný v nasledujúcom texte.
			\begin{lstlisting}[label=lst:insertinto-syntax,caption=Syntax INSERT INTO klauzuly \cite{web:esper-doc}]
			insert [istream | irstream | rstream] into event_stream_name  [ (property_name [, property_name] ) ]
			\end{lstlisting}
			\item[UPDATE] Klauzula UPDATE slúži na úpravu vlastností udalosti a je aplikovaná pred spracovaním príkazu.
		\end{description}

		EPL tiež umožňuje definovať náhľady, ktoré predstavujú istú obdobu náhľadov (view) ako ich poznáme z databázových prostredí. CEP sa ale zaoberá prácou s tokmi dát a nie statickým pohľadom na ne, preto rozširuje tieto náhľady o viacero funkcií. Tými sú napríklad tvorenie štatistík z vlastností udalostí, ich zoskupovanie či funkcie pre umožnenie výberu udalostí, ktoré bude dátové okno obsahovať. Náhľady môžu byť reťazené. Názorný príklad fungovania náhľadov je na obrázku \ref{image_cep-windows}.
	
		\myFigure{cep-windows}{Príklad dátového okna obmedzeného časom a počtom udalostí \cite{web:softwarearchitekturen}}{Príklad dátového okna obmedzeného časom a počtom udalostí}

		Esper rozdeľuje náhľady do menných priestorov. V nasledujúcom zhrnutí sú predstavené tie základné \cite{web:esper-doc}.
		\begin{description}
			\item[Náhľady do dátových okien] definujú kĺzavé okná a nájdeme ich v mennom priestore win.
				\subitem win:length - náhľad rozširuje okno o definovaný počet udalostí do minulosti. Nové udalosti vytláčajú tie, ktoré sa do okna už nezmestia, čo vytvára štruktúru podobnú konceptu fifo.
				\subitem win:length\_batch - náhľad funguje podobne ako win:length, avšak udalosti odstraňuje nárazovo pri zaplnení okna o definovanej veľkosti.
				\subitem win:time - náhľad rozširuje dátové okno o minulé udalosti obmedzené časovou značkou.
				\subitem win:keepall - na rozdiel od predchádzajúcich náhľadov, ktoré udalosti nevyhovujúce podmienke z dátového okna odstránia, tento náhľad udržuje\ všetky prijaté udalosti. Keďže sú všetky udalosti udržiavané v pamäti je nutné dať si pozor aby náhľad nezabral všetku dostupnú operačnú pamäť.
				\subitem win:firstlength - je podobný win:length v tom že obmedzuje počet udalostí. Naproti nemu však v dátovom okne udržiava len prvých n udalostí.
				\subitem win:firsttime - je ekvivalentný s win:firstlength, avšak udalosti nie sú obmedzené počtom, ale časom.

			\item[Štandardné náhľady] Ostatné bežne používané náhľady sú dostupné s predponou std.
				\subitem std:unique - tento náhľad uchováva len najaktuálnejšiu udalosť v prípade prijatia duplicitnej udalosti.
				\subitem std:size - náhľad poskytuje prístup k premennej size, ktorá obsahuje počet udalostí prijatých daným príkazom. Náhľad vytvára novú udalosť len v prípade zmeny premennej size. 
				\subitem std:firstevent - náhľad udržiava len prvú prijatú udalosť. Všetky udalosti prijaté po nej sú ignorované. Toto správanie spôsobuje že je jeho forma podobná ako win:length o veľkosti 1.

			\item[Štatistické náhľady] Štatistické náhľady pokrýva menný priestor stat.
				\subitem stat:uni - umožňuje pristupovať k štatistickým hodnotám, napríklad priemeru, smerodajnej odchýlky, súčtu či rozptylu.
				\subitem stat:correl - počíta korelačnú hodnotu. Funkcia vyžaduje minimálne dva parametra, ktorých korelačnú hodnotu počíta.
				\subitem stat:weighted\_avg - ako názov napovedá, náhľad umožňuje vypočítať vážený priemer. Podobne ako stat:correl vyžaduje aspoň dva parametra, prvý udáva údaj z ktorého sa počíta priemer a druhý jeho váhu.
		\end{description}
		
		EPL jazyk tiež umožňuje definovať vzory. Patterny sú výrazy, ktoré hľadajú zhodu podľa definovaného vzoru. Je možné ich definovať ako samostatný výraz alebo ako súčasť príkazu. Môžu sa vyskytovať kdekoľvek v klauzule FROM, vrátane join. Vďaka tomu ich je možné použiť v kombinácii s klauzulami WHERE, GROUP BY, HAVING a INSERT INTO.
		
		V nasledujúcich výpisoch sú príklady príkazov, zobrazujúcich príklady syntaxe popisovanej v predchádzajúcom texte.
		\begin{lstlisting}[label=lst:epl-simple,caption=Jednoduchý EPL príkaz]
		select * from TweetEvent.win:time(60 sec) where message='happy'
		\end{lstlisting}
		
		\begin{lstlisting}[label=lst:output-example,caption=EPL príkaz s výstupom každých 60 sekúnd]
		select sum(price) from OrderEvent.win:time(30 min)
			output snapshot every 60 seconds
		\end{lstlisting}

		\begin{lstlisting}[label=lst:epl-join,caption=Jednoduchý EPL príkaz použitím join]
		select * from TickEvent.std:unique(symbol) as t,
			NewsEvent.std:unique(symbol) as n
		where t.symbol = n.symbol
		\end{lstlisting}

		\begin{lstlisting}[label=lst:epl-pattern,caption=EPL príkaz s použitím vzoru \cite{web:esper-doc}]
		select a.custId, sum(a.price + b.price)
		from pattern [every a=ServiceOrder -> 
			b=ProductOrder(custId = a.custId) where timer:within(1 min)].win:time(2 hour) 
		where a.name in ('Repair', b.name)
		group by a.custId
		having sum(a.price + b.price) > 100
		\end{lstlisting}
		
		\begin{lstlisting}[label=lst:insert-into,caption=Príklad použitia klauzuly INSERT INTO \cite{web:esper-doc}]
		insert into CombinedEvent
		select A.customerId as custId, A.timestamp - B.timestamp as latency
		from EventA.win:time(30 min) A, EventB.win:time(30 min) B
		where A.txnId = B.txnId
		\end{lstlisting}
		
		\begin{lstlisting}[label=lst:views,caption=Príklady použitia dátových náhľadov]
Počíta priemernú cenu akcie z udalostí prijatých v posledných 30 sekundách
select sum(price) from StockTickEvent(symbol='GE').win:time(30 sec)

Počíta počet udalostí StockTickEvent prijatých počas poslednej minúty
select size from StockTickEvent.win:time(1 min).std:size()

Počíta smerodajnú odchýlku z posledných 10 prijatých udalostí
select stddev from StockTickEvent.win:length(10).stat:uni(price)
		\end{lstlisting}
		
		\begin{lstlisting}[label=lst:update,caption=Príklady úpravy udalosti pred spracovaním \cite{web:esper-doc}]
update istream AlertEvent 
set severity = 'High'
where severity = 'Medium' and reason like '%withdrawal limit%'		
		\end{lstlisting}
		
		\subsubsection{Objektový model}	
		Objektový model je sada tried poskytujúcich objektovú reprezentáciu príkazu alebo vzoru. Tá umožňuje zostrojiť, zmeniť alebo získať údaje z EPL príkazov a vzorov na vyššom stupni ako pri práci s textovou reprezentáciou. Objektový model pozostáva z objektového grafu, ktorého prvky je jednoducho prístupné. Objektový model umožňuje plný export do textovej formy a naopak.
		
		Príkazy vo výpise \ref{lst:epl-nomodel} a \ref{lst:epl-model} sú ekvivalentné. Podobným spôsobom je možné vytvárať príkazy, vzory, definovať premenné premenné alebo vytvárať dátové okná.
		
		\begin{lstlisting}[label=lst:epl-nomodel,caption=EPL príkaz bez použitia objektového modelu \cite{web:esper-doc}]
	select line, avg(age) as avgAge 
	from ReadyEvent(line in (1, 8, 10)).win:time(10) as RE
	where RE.waverId != null
	group by line 
	having avg(age) < 0
	order by line
		\end{lstlisting}
		
		\begin{lstlisting}[label=lst:epl-model,caption=EPL príkaz s použitím objektového modelu \cite{web:esper-doc}]
	EPStatementObjectModel model = new EPStatementObjectModel();
	model.setSelectClause(SelectClause.create()
		.add("line")
		.add(Expressions.avg("age"), "avgAge"));
	Filter filter = Filter.create("com.chipmaker.ReadyEvent", Expressions.in("line", 1, 8, 10));
	model.setFromClause(FromClause.create(
		FilterStream.create(filter, "RE").addView("win", "time", 10)));
	model.setWhereClause(Expressions.isNotNull("RE.waverId"));
	model.setGroupByClause(GroupByClause.create("line"));
	model.setHavingClause(Expressions.lt(Expressions.avg("age"), Expressions.constant(0)));
	model.setOrderByClause(OrderByClause.create("line"));
		\end{lstlisting}
	
		Rovnako ako v textovej reprezentácii sú v objektovej reprezentácii klauzuly SELECT a FROM povinné. Pomocou objektového modelu je tiež možné skontrolovať syntax príkazu pred pridaním do Esper engine.

	\subsection{Api}
		Esper pre svoje ovládanie neposkytuje grafické rozhranie. Na komunikáciu používa api, ktorá definuje tieto primárne rozhrania:
		\begin{itemize}
			\item Rozhranie udalostí a ich typov
			\item Administrátorské rozhranie na vytváranie a správu EPL príkazov a vzorov a definovanie konfigurácie Esperu
			\item Runtime rozhranie, ktoré slúži na posielanie udalostí do Esperu, definovanie premenných a spúšťanie on-demand výrazov.
		\end{itemize}
		
		\subsubsection{EP Service Provider}
		EPServiceProvider reprezentuje konkrétnu inštanciu Esperu. Každá takáto inštancia je nezávislá od ostatných a má svoje vlastné administrátorské a runtime rozhranie. Pri prístupe umožňuje rozhranie voľbu ``getDefaultProvider'' bez parametrov, ktorá vráti predvolenú inštanciu, alebo ``getProvider'' s textovým parametrom URI identifikujúcim konkrétnu inštanciu. Tá je vytvorená ak ešte neexistuje. Opakované volania s rovnakým URI vracajú stále rovnakú inštanciu. Túto funkcionalitu je možné využiť napríklad pre oddelenie pracovného prostredia viacerých užívateľov.
		
		\subsubsection{EP Administrator}
		EPAdministrator umožňuje registrovanie EPL príkazov, vzorov alebo ich objektovej reprezentácie do Esperu a to metódami createPattern, createStatement a create pre objektový model. Tieto funkcie poskytujú voliteľné parametre umožňujúce definovať meno príkazu a užívateľský objekt, ktorý je v implementačnej časti tejto práce využitý na ukladanie dodatočných informácií o príkaze - napríklad definovanie TTL pri ukladaní výsledkov do databázy.
		
		Po registrácii nového EPL výrazu rozhranie vracia inštanciu vytvoreného EPStatement, pomocou ktorej môžeme ovládať už vytvorený príkaz alebo pristupovať k výsledkom. Praktická časť práce z ovládacích funkcií využíva stop() a start(), ktoré definujú, či je príkaz aktívny.
		
		Esper poskytuje tri možnosti ako pristupovať k výsledkom konkrétneho príkazu. Tieto je možné rôzne kombinovať. Možnosti sú predstavené v nasledujúcom výpise:
		\begin{description}
			\item[Listener] V prvom prípade aplikácia poskytuje implementáciu rozhraní UpdateListener alebo StatementAwareUpdateListener vytváranému príkazu. Takýto listener bude následne notifikovaný pri výskyte novej udalosti a metóde update bude predaná inštancia EventBean, ktorá obsahuje udalosť produkovanú niektorým z príkazov.
			\item[Subscriber] Týmto spôsobom Esper posiela výsledky na definovaný subscriber. Je to najrýchlejšia možnosť, pretože Esper predáva typované výsledky priamo do objektov aplikácie, nemusí teba zostavovať inštancie EventBean ako v predošlom prípade. Nevýhodou je že príkaz môže mať registrovaný maximálne jeden subscriber, naproti predošlému spôsobu, kde umožňoval definovať viacero listenerov.
			\item[Pull Api] Týmto spôsobom aplikácia pristupuje k výsledkom on-demand spôsobom, kde jednorázovou žiadosťou o výsledky daného príkazu získa zoznam EventBean prístupný pomocou iterátora. Toto je využiteľné v prípade kedy aplikácia nevyžaduje nepretržité spracovanie nových výsledkov v real-time.
		\end{description}
		V tomto projekte bol použitý prvý spôsob listenera. Aplikácia v tomto prípade použije implementáciu rozhrania StatementAwareUpdateListener, ktorá je registrovaná pri vytváraní nového príkazu metódou addListener. Vďaka použitiu rozhrania StatementAwareUpdateListener a nie UpdateListener získava aplikácia prístup k príkazu, ktorý konkrétnu udalosť vyprodukoval, pre všetky príkazy môže byť preto definovaný jediný spoločný listener.
		
		Esper podporuje tiež spracovanie udalostí, ktoré nevyhoveli žiadnemu statemenentu. Tieto výsledky získame registrovaním implementácie rozhrania UnmatchedListener. 
		
		\subsubsection{EP Runtime}
		EPRuntime rozhranie slúži na odosielanie nových udalostí do Esperu k spracovaniu, nastavenie a prístup k hodnotám premenných a spúšťanie on-demand EPL výrazov. Na odosielanie nových udalostí slúži metóda sendEvent, ktorá je preťažená. Typ parametra tejto metódy indikuje typ udalosti odosielanej do Esperu. Tieto typy boli bližšie popísané v predchádzajúcich sekciách.
		
		V prípade použitia XML definície typov udalostí sa pri spracovaní prichádzajúcej udalosti skontroluje že meno koreňového elementu prichádzajúcej udalosti je zhodné s menom typu udalosti definovanej XML schémou.
		
		Ak aplikácia nepozná EPL výrazy dopredu alebo nevyžaduje streamovanie výsledkov, je možné prostredníctvom EPRuntime spúšťať jednorázové výrazy. Tieto nie sú permanentné, po ich vykonaní je výsledok okamžite predaný aplikácii na spracovanie. Použitie nachádzajú napríklad v spojení s pomenovanými oknami a tabuľkami, ktoré je možné indexovať pre zrýchlenie prístupu.		

\section{Cassandra}
	Cassandra je databázový projekt, ktorý pôvodne vznikol vo firme Facebook. Neskôr bol zverejnený ako open-source a v roku 2009 bol prijatý do Apache inkubátora. V roku 2010 získal top prioritu a je naďalej vyvíjaný a dostupný pod Apache 2.0 licenciou \cite{what-is-Cassandra}.
	
	Cassandra je distribuovaná databáza, ktorá umožňuje spracovanie a uchovávanie veľkého množstva dát rozložených na veľkom počte menej výkonných serverov, ako je to znázornené na obrázku \ref{image_intro_Cassandra}. Táto architektúra zároveň poskytuje vysokú dostupnosť dát pri zabezpečení proti strate dát pri výpadku niektorého zo serverov. Cassandra je navrhnutá na použitie veľkého počtu počítačov (v ráde stoviek) podľa možností rozložených v rôznych častiach sveta.

	\myFigure{intro_Cassandra}{Škálovanie databázy Cassandra s rastúcou záťažou \cite{img:scaling}}{Škálovanie databázy Cassandra s rastúcou záťažou}
	
	Cassandra bola navrhnutá pre beh na cenovo dostupnom hardware a podporuje rýchly zápis veľkého množstva dát pri zachovaní efektívnosti prístupu k nim. Týmto pomáha znižovať firemné náklady na hardware.
	
	Vďaka týmto vlastnostiam je Cassandra využívaná množstvom známych firiem, medzi ktoré patrí napríklad CERN, eBay, GitHub, Netflix, Twitter alebo Cisco. Veľké produkčné nasadenia obsahujú stovky TB dát v klastroch zložených zo stoviek serverov. Pri porovnaní výkonnosti s ostatnými NoSQL databázami  Cassandra získava výborné výkonnostné výsledky aj vďaka svojej jednoduchej architektúre.
	
%TODO too much
% porovnanie s alternatívami vidíme na obrázku \ref{image_Cassandra-performance}
	
%	\myFigure{Cassandra-performance}{Porovnanie výkonnosti cassandry s alternatívnymi NoSQL databázami}{Výkonnosť Cassandra databázy}
	
	Cassandra je dostupná z dvoch zdrojov, prvým z nich stránka projektu Apache Cassandra. Tá je základnou verziou a obsahuje databázový engine a cqlsh nástroj slúžiaci ako vývojový shell. Pre jeho spustenie je nutné mať nainštalované interpretátor jazyka python.
	
	Druhou je nadstavba tretej strany DataStax Cassandra, ktorá odlišuje komerčnú a nekomerčnú verziu. V nekomerčnej verzii nájdeme rovnako ako v Apache balíku databázový engine a cqlsh. Navyše na svojich stránkach DataStax poskytuje zdarma rozšírenia, ktoré zjednodušujú prácu s databázou, a to:
	
	\begin{description}
		\item[OpsCenter] je grafický nástroj na správu databázy. Poskytuje prehľadné rozhranie pre administrátorov a vývojárov v ktorom je možné vidieť jednotlivé časti klastru. Umožňuje monitorovať stav, aktuálnu zátaž, pridávať a odoberať servery do konfigurácie klastra, nastaviť zálohovanie či generovať štatistiky. Pomocou neho je možné prehľadne spravovať klastre zložené zo stoviek serverov.

		\item[DevCenter] Pre úpravu štruktúry a údajov databázy slúži nástroj DevCenter. Grafické rozhranie umožňuje po pripojení na databázový engine vytvárať a spúšťať dotazy v CQL jazyku. Pri vytváraní dotazov je automaticky kontrolovaná syntax a sú zvýraznené chyby s popisom. 
		Obsahuje tiež interaktívne pomôcky na vytvorenie keyspace alebo tabuliek, export výsledkov alebo ukladanie dotazov.
		Tento nástroj sa dá zjednodušene vnímať ako grafická verzia cqlsh.

		\item[Java driver] Pre použitie databázy v programe napísanom v jave je potrebný ovládač, ktorý je možné stiahnuť práve na stránkach DataStax. Nájdeme tu tiež ovládače pre ďalšie vývojové platformy.
	\end{description}
	
	Aj keď v mnohom pripomína Cassandra relačnú databázu, nepodporuje plne relačný model. Namiesto toho poskytuje klientom jednoduchší dátový model a prináša návod ako niektoré chýbajúce funkcie nahradiť.
	Nasledujúci text rozoberá niektoré základné odlišnosti cassandry a relačných databáz.
	\subsection{Keyspace}
	Keyspace je možné prirovnať k schéme relačnej databázy. Slúži ako kontajner pre dáta, ktoré zdieľajú určité vlastnosti. Pri vytvorení keyspace je nutné určiť spôsob replikácie a počet kópií dát. Tieto kópie slúžia na zachovanie dátovej integrity v prípade výpadku niektorého zo serverov. Syntax pre vytvorenie nového keyspace je zobrazená vo výpise \ref{lst:create-keyspace}.

	\begin{lstlisting}[label=lst:create-keyspace,caption=Syntax pre vytvorenie nového keyspace]
	CREATE KEYSPACE <identifier> WITH <properties>
	\end{lstlisting}
	
	Po vytvorení keyspace je možné v ňom vytvárať tabuľky, ktoré už obsahujú samotné dáta. Tabuľka patrí do keyspace podobne ako v relačnom pojatí je tabuľka obsiahnutá v databáze. 
	
	\myFigure{keyspace-colfamily}{Porovnanie štruktúry SQL a NoSQL databázy \cite{img:colfamily}}{Porovnanie štruktúry SQL a NoSQL databázy}
	
	V predchádzajúcich verziách CQL bolo možné sa stretnúť s skupinami stĺpcov (column family). Tento pojem bol používaný v spojení s dynamickým modelom databázy, kedy stĺpce nebolo nutné definovať, ich definíciou sa vynucoval typ obsahu. V najnovšej verzii CQL sa však od tohoto prístupu upustilo, a dnes v aktuálnej verzii nájdeme tabuľky podobne ako ich poznáme z relačných databáz. Klauzula COLUMN FAMILY ostala ako synonym klauzuly TABLE. Porovnanie štruktúry SQL a NoSQL databázy je na obrázku \ref{image_keyspace-colfamily}.
	
	\subsection{Primárny kľúč}
	Podobne ako v relačnej databáze musí mať každý riadok unikátny identifikátor - primárny kľúč. Ten môže byť tvorený jediným údajom, alebo môže byť zložený z viacerých stĺpcov. Naproti relačnej databáze má primárny kľúč aj dodatočný význam.
	
	\begin{lstlisting}[label=lst:cql-pk,caption=Tvorenie primárneho kľúča v CQL]
	create table thesis (
		key_one text,
		key_two int,
		key_three int,
		data text,
		PRIMARY KEY(key_one, key_two, key_three)
	);
	\end{lstlisting}
	
	Vo výpise \ref{lst:cql-pk} je zobrazené názorné vytvorenie tabuľky so zloženým primárnym kľúčom. Tento kľúč má dve zložky:
	\begin{description}
		\item[Kľúč partície] (Partition key) určuje na ktorých uzloch sa uložia dáta. Je tiež zodpovedný za distribúciu naprieč jednotlivé uzly. V príklade je to key\_one.
		\item[Zoskupovací kľúč] (Clustering key) určuje zoskupovanie dát, čo je proces, ktorý usporadúva dáta v rámci daného úseku. V príklade je použitý zložený kľúč pozostávajúci z key\_two a key\_three.
	\end{description}
	Keďže Cassandra je distribuovaná databáza, tieto kľúče slúžia k vyhľadaniu miesta, kde sú uložená vyhľadávané záznamy. A práve toto vyhľadávanie je veľmi ovplyvnené architektúrou databázy.
	
	Práve tu vzniká jedna z najväčších odlišností od relačných databáz, ktorých užívatelia sú zvyknutí filtrovať údaje podľa ľubovoľného stĺpca tabuľky. Vyhľadávanie v klauzule WHERE je možné len podľa stĺpcov z ktorých sa primárny kľúč skladá a to v poradí v ktorom sú definované. Vo výpise \ref{lst:cql-pk-use} vidíme možnosti filtrovania v dotazoch do tabuľky definovanej výpisom \ref{lst:cql-pk}.
	
	\begin{lstlisting}[label=lst:cql-pk-use,caption=Príklad vyhľadávanie v tabuľke thesis]
Príklad validných dotazov:
	select * from thesis where key_one='kluc' and key_two=11
	select * from thesis where key_one='kluc' and key_two=11 and key_three=13
		
Príklad nevalidných dotazov:
	select * from thesis where key_two=11
	select * from thesis where key_three=11 and key_one='kluc'
	\end{lstlisting}
	
	Kvôli tomuto obmedzeniu je nutné prispôsobovať návrh tabuliek dotazom, ktoré ich budú využívať. To nutne vedie k duplicitám dát vo viacerých tabuľkách. Toto je ďalšia dôležitá odlišnosť a jeden z problémov v zmene myslenia pri prechode zo sveta relačných databáz, kde je správnym prístupom normalizácia dát. Pre dnešné dátové úložiská však nepredstavuje taký problém množstvo dát ako rýchlosť prístupu k nim. Preto sú duplicitné dáta prijateľným kompromisom v porovnaní s výhodami ktoré NoSQL databázy prinášajú.

	\subsection{Dátové typy}
	Ďalším z rozdielov oproti bežným relačným databázam sú dátové typy, ktoré je možné využívať. Nasledujúci zoznam zhŕňa tie, s ktorými sa v relačných databázach bežne nestretneme.
	\begin{description}
		\item[Kolekcie] Použitím dátového typu kolekcie môžeme definovať map, set a list. Pri definícii je nutné uviesť dátový typ, ktorý bude daná kolekcia obsahovať. Kolekcie sú vhodné na ukladanie relatívne malého množstva dát, napríklad telefónnych čísel daného užívateľa alebo popisov výrobku. 

		\item[TimeUUID] Typ UUID je používaný na predchádzanie kolíziám. Jeho rozšírenie TimeUUID obsahuje navyše časovú značku, čo je možné využiť v aplikáciách na vytvorenie jedinečného časového identifikátora.
		
		Na výpočet TimeUUID je použitá MAC adresa, časová značka a sekvenčné číslo. Z vygenerovaného TimeUUID je možné spätne získať časovú značku, takže funguje ako časový identifikátor, ktorý je zároveň jedinečný.
		
		\item[Tuple] umožňuje udržiavať stanovený počet vopred definovaných typov dát v jednom dátovom poli.
	\end{description}

	Výpis \ref{lst:dtypes} zobrazuje príklad použitia týchto dátových typov. V prvom kroku je vytvorená tabuľka incidentov, ktorá používa ako unikátny identifikátor riadku časovú značku a obsahuje dva stĺpce - data uchovávajúce informácie o incidente a notified, ktorý obsahuje zoznam osôb ktoré majú byť upozornené na incident.
	V druhom kroku je do tabuľky vložený vzorový incident.
	\begin{lstlisting}[label=lst:dtypes,caption=Príklad použitia dátových typov databázy Cassandra]
CREATE TABLE incident (
	tid timeuuid primary key,
	data <tuple<int, text, float>>,
	notified set<text>,
);

INSERT INTO incident (tid, data, notified) VALUES(
	now(),
	(31, 'Sunny', 77.5),
	{'f@baggins.com', 'baggins@gmail.com'}
);
	\end{lstlisting}

	\subsection{Offset, Join, Order, Count}
	V relačných databázach je samozrejmosťou výsledky radiť podľa potreby či spájať tabuľky. Cassandra má však rozdielnu architektúru, preto niektoré z typických klauzúl nenájdeme vôbec, prípadne je ich implementácia odlišná. Nasledujúci popis rozoberá hlavné rozdiely medzi Cassandrou a relačnoými databázami z pohľadu práce s týmito klauzulami.
	
	\begin{description}
	\item[OFFSET] Klauzuly limit a offset sú bežne používane v spojení so stránkovaním výsledkov vyhľadávania. Kvôli svojej distribuovanej architektúre však Cassandra neimplementuje klauzulu offset. Pri stránkovaní teda nejde jednoducho preskočiť na konkrétnu stranu. Implementácia teda zvyčajne spočíva v zobrazení výsledkov vo vzťahu ku konkrétnemu záznamu.

	\item[JOIN] Klauzula JOIN je nahradená ústupom od normalizácie. Údaje sú uchovávané v tabuľkách duplicitne a ich štruktúra je prispôsobená požiadavkám na operácie, ktoré s nimi budú vykonávané. Potrebné join sa teda musia plánovať už pri návrhu tabuľky, prípadne majú za následok vznik nových tabuliek.
	
	\item[ORDER BY] Klauzulu ORDER BY tiež nie je možné použiť ľubovoľne na každom definovanom stĺpci. V prípade že tabuľka definuje zoskupovací stĺpec, je možné ho použiť pre zoradenie výsledkov. V opačnom prípade je možné použiť len stĺpce definované v zoskupovacom kľúči.
	
	\item[COUNT] Klauzula COUNT je implementovaná rozdielne ako v relačných databázach. Operácia pre zápis v databáze Cassandra prebieha bez nutnosti čítania už existujúcich záznamov. To zvyšuje rýchlosť zápisu, avšak databáza neudržiava informáciu o aktuálnom počte záznamov. Operácia count preto musí pri zavolaní spočítať existujúce záznamy, čo je časovo náročná operácia. Preto sa táto klauzula zvyčajne používa spolu s klauzulou limit, ktorá zamedzí zbytočnému zahlteniu databázy \cite{cascount}.
	\end{description}
	
	Tieto rozdiely sú pri prechode z relačných databáz prekvapením, návrh databáz je totiž z veľkej časti obmedzujúci. Obmedzenia vyplývajú predovšetkým z architektúry dátového úložiska databázy, avšak práve táto architektúra ponúka aj mnoho výhod. Preto je použitie NoSQL databázy nutné dôkladne zvážiť a prispôsobiť potrebám konkrétnej aplikácie.
	
\section{Frameworky}
	Pre uľahčenie vývoja projektov je dnes bežné použitie frameworku. Rozsahom malé aplikácie často vyžadujú použitie funkcionality (prihlasovanie, odosielanie mailov, práca s databázou, zjednotenie grafického zobrazenia pre rôzne platformy, správu závislostí), ktorej implementácia je zdĺhavá a v málo prípadoch lepšia ako pri opätovnom použitý riešenia na to stavaného. Pri vývoji jednotlivých častí tohoto projektu boli použité voľne dostupné frameworky, ktorých krátkym popisom sa zaoberá táto sekcia.

	\subsection{Spring}
	Spring je framework napísaný v jave, distribuovaný pod Apache License verzie 2.0. Spring pozostáva z viacerých projektov zameraných na riešenie konkrétnych problémov vývoja aplikácie. V praktickej časti tejto aplikácie boli použité nasledujúce spring komponenty:
	\begin{description}
		\item[Spring Framework] svojou funkcionalitou rieši základné oblasti vývoja java aplikácií. Obsahuje základnú podporu pre injektovanie závislostí, správu transakcií, vývoj webových aplikácií alebo prístup k dátam. Táto funkcionalita je rozdelená do komponentov, z v praktickej časti sú použité spring-core, spring-jdbc a spring-webmvc.
		
		\item[Spring Boot] je spôsob ako urýchliť vývoj spring aplikácií. Rovnako ako Ruby On Rails sa prikláňa k prístupu convention over configuration. Z využitej funkcionality je dôležitý hlavne webový server Tomcat, ktorý tento komponent obsahuje. Vďaka nemu nie je nutné robiť deploy war súborov serverovej časti aplikácie, stačí jednoducho zdrojové kódy preložiť a spustiť. Je možné tiež využiť komponenty obsiahnuté v rodičovskom pom súbore. Spring boot automaticky nakonfiguruje spring aplikáciu bez nutnosti XML konfiguračných súborov s predvolenými nastaveniami.
	\end{description}
		
	\subsection{Ruby On Rails}
	Mottom frameworku Ruby On Rails je snaha o dosiahnutie spokojnosti programátora a udržateľnú produktivitu. Framework uprednostňuje prístup convention over configuration, čo znamená že konfigurácia je preddefinovaná a ak programátor používa predom dohodnuté konvencie postačujú minimálne úpravy na dosiahnutie požadovaného výsledku. Konfiguráciu je samozrejme možné podľa potreby upraviť. Framework je voľne dostupný a distribuovaný pod MIT licenciou.
	
	Vytvorenie a spustenie nového projektu pozostáva z jednoduchej série príkazov zobrazených vo výpsie \ref{lst:ror-create}. Všetky závislosti sú automaticky stiahnuté pri vytvorení projektu nástrojom bundler.
	\begin{lstlisting}[label=lst:ror-create,caption=Príklad vytvorenia a spustenia projektu v Ruby On Rails]
		rails new myweb
		cd myweb
		rails s
	\end{lstlisting}

	Adresárová štruktúra projektu je prispôsobená MVC architektúre. Jednotlivé komponenty nájdeme v adresároch model, view a controller.
	\begin{description}
		\item[Model] je vrstva reprezentujúca údaje a ich logiku. Umožňuje definovať objekty, ktorých dáta vyžadujú uloženie v databáze. Vlastnosti týchto objektov sú mapované na relačné dáta. Model je možné definovať pomocou migrácií, ktoré po spustení upravujú štruktúru databázy a umožňujú rollback k predchádzajúcemu stavu. V modeloch je možné definovať kontroly vstupných dát, asociácie, funkcie na rozšírenie prístupu k údajom alebo spätné volania v závislosti na vykonávanej akcii.
		\item[View] adresár je ďalej rozdelený do podadresárov reprezentujúcich jednotlivé controllery. Dôležitým je tiež adresár layout, ktorý ako názov napovedá obsahuje jednotný layout aplikácie. Prípona .erb súborov umožňuje použitie ruby kódu. V tomto adresári sa stretneme tiež so systémom vkladania čiastkových elementov stránky príkazom partial. Adresár helpers umožňuje definovať funkcie použiteľné v erb súboroch, ktoré sprehľadňujú štruktúru kódu.
		\item[Controller] je zodpovedný za obsluhu požiadavku a vyprodukovanie odpovedi. Jeho úloha zvyčajne pozostáva z prijatia požiadavky, získania alebo uloženia dát do databázy, definovania premenných pre zobrazenia potrebného view súboru. Controller poskytuje prístup k request a response objektom a definuje premennú flash, ktorá nesie správu o úspechu alebo chybe danej akcie.
	\end{description}
	
	Konfigurácia projektu je rozdelená do troch súborov podľa aktuálneho prostredia - test, development a production. Rovnako je možné podľa prostredia definovať rôzne databázy v súbore database.yml.
	V konfigurácii framework nájdeme tiež súbory initializers, ktoré sú spustené pri štarte projektu a inicializujú jednotlivé komponenty a súbor routes obsahujúci smerovanie prichádzajúcich požiadavkov na jednotlivé controllery. Závislosti a použité komponenty sú definované v súbore Gemfile.
	
	\subsection{Sinatra}
	Sinatra je jazyk použiteľný pre rýchly vývoj jednoduchých webových aplikácií postavených na ruby. Framework zapuzdruje jednoduchý webový server. Umožňuje definovať takzvané route, predstavujúce URL, ktoré aplikácia rozoznáva. Obsahom bloku definujúceho route je telo metódy, ktorá sa má vykonať pri zavolaní danej URL. Tieto bloky akceptujú vstupné parametre, predstavujúce GET a POST parametre. Príklad takejto routy zobrazuje nasledujúci výpis.
	\begin{lstlisting}[label=lst:sinatra-sampel,caption=Príklad definovania GET route vo frameworku Sinatra]
	get '/hello/:name' do |n|
		"Ahoj #{n}!"
	end
	\end{lstlisting}
	V tejto aplikácii bola Sinatra použitá pri demonštrácii jedného z usecase - webu kontaktov. Celá logika aplikácie pozostáva vďaka použitiu tohoto frameworku z 32 riadkov.

	\subsection{Bootstrap}
	Bootstrap framework je najpopulárnejším HTML a CSS frameworkom pre vývoj webových projektov. Umožňuje rýchly a jednoduchý vývoj front-end aplikácie. Je dostupný vo forme css súborov ako aj sass pre jednoduché použitie v rails projektoch. Distribuovaný je pod MIT licenciou.
	
	\myFigure{bootstrap-grid}{Ukážka Bootstrap grid systému \cite{bootstrap}}{Bootstrap grid systém}

	Bootstrap poskytuje triedy upravujúce zobrazovanie základných HTML komponentov. Aplikovaný je pomocou premennej class na jednotlivých komponentoch. Distribúcia tiež zahŕňa sadu ikon, písem a javascript, ktorý je zodpovedný napríklad za zobrazenie vyskakovacích okien alebo pomocných textov. 
	
	\begin{lstlisting}[label=lst:bootstrap-grid,caption=Príklad použitia Bootstrap grid systému \cite{bootstrap}]
	<div class="row">
		<div class="col-md-8">.col-md-8</div>
		<div class="col-md-4">.col-md-4</div>
	</div>
	<div class="row">
		<div class="col-md-4">.col-md-4</div>
		<div class="col-md-4">.col-md-4</div>
		<div class="col-md-4">.col-md-4</div>
	</div>
	\end{lstlisting}
	
	Jednou z jeho základných súčastí je mriežkový systém rozloženia stránky. Pred jeho použitím je potrebné pridať elementom, ktoré obaľujú stránku triedu container.	Mriežkový systém poskytuje responzívne rozhranie pozostávajúce z riadka obsahujúceho 12 stĺpcov, ktoré sa prispôsobujú podľa veľkosti obrazovky cieľového zariadenia. Tento systém umožňuje stránku rozdeliť na sekcie, ktoré je možné odsadiť, zarovnať, alebo rovnomerne rozložiť podľa potreby. Každý stĺpec funguje ako samostatná jednotka, ktorá môže obsahovať nový riadok s 12 stĺpcami. Stĺpce je možné spájať so skupín. Príklad použitia takéhoto rozloženia je vo výpise \ref{lst:bootstrap-grid}. Rozloženie stránky vytvorenej týmto kódom zobrazuje obrázok \ref{image_bootstrap-grid}.
					
	V tejto aplikácii bol bootstrap použitý pri vytváraní front-end administračného rozhrania ThesisWeb a usecase webu klientov.
	
	Využité frameworky a nástroje značne uľahčili vývoj projektu a v kapitole \ref{chap:instalacia} je bližšie popísaný spôsob ich inštalácie a použitia. Všetky použité nástroje sú voľne dostupné.
	
\chapter{Návrh a Implementácia}
Táto kapitola popisuje implementačné detaily projektu. 
%TODO img - use case
%TODO rozdiel v api export / get exportAll / getAll

\section{ThesisApi}
	Esper je voľne dostupný pre dve vývojové platformy - javu a .net. Pre serverovú časť aplikácie som zvolil javu s použitím spring frameworku.
	\subsection{Konfiguračné súbory}
		application-config.xml - jdbc config
		database.properties
		esper.cfg.xml
		logback.xml
	\subsection{Databáza}
		Aplikácia využíva dve databázy, jednu na ukladanie konfiguračných dát a druhú na udalosti nájdené esperom. Je to tak kvôli predpokladu že esper bude produkovať veľké množstvo dát, preto je vhodné použitie databázy optimalizovanej pre zápis.
	
		\subsubsection{Databáza konfigurácie}
		Ako databázu pre ukladanie stavu esperu a jednotlivých konfiguračných položiek som použil \ac{HSQLDB}. Táto databáza je napísaná v jave a pre ukladanie tabuliek ponúka pamäťový aj diskový mód. Použil som ju, pretože je jednoduchá, nenáročná na systém a je možné ju stiahnuť ako jednu zo závislostí aplikácie pomocou mavenu. Databáza je jednoducho nahraditeľná iným riešením, keďže k nej je pristupované pomocou jdbc. Pre zmenu je potrebné upraviť konfiguračný súbor application-config.xml a zmeniť položku dataSource.
		
		Databáza obsahuje tabuľku schém a statementov, ktoré sú načítavané pri štarte Api. Jej štruktúru vidíme na obrázku \ref{image_db-model-1}.
		\myFigure{db-model-1}{Model databázy konfigurácie}{Model databázy konfigurácie}
			
		\subsubsection{Databáza udalostí}
		Databáza udalostí slúži na ukladanie výsledkov vyhovujúcich niektorému z definovaných statementov. Ako konkrétne riešenie som použil databázu Cassandra. Cassandra je jedným z predstaviteľov NoSQL databáz. Jednou z jej výhod je možnosť rozšíriteľnosti v prípade veľkého objemu dát a optimalizácia pre zápis. Použitie tejto databázy je v rámci "proof of concept" princípu, kde by bolo jednoduchšie pracovať napríklad s HSQLDB ako v prípade konfigurácie, avšak kvôli možnosti generovania veľkého množstva udalostí esperom je použitá databáza na to vhodná.
		\myFigure{db-model-2}{Model databázy udalostí}{Model databázy udalostí}
	
	\subsection{Spring framework}
		Ako základ serverovej časti aplikácie som použil Spring framework. Jeho hlavnou úlohou je injektovanie závislostí vo väčšine tried, no využil som aj rozšírenia pre databázu, webový prístup a pribalený webový server tomcat.
		Spring Framework poskytuje v aplikácii podporu pre dependency injection, správu transakcií a prístup k dátam pomocou jdbc. Serverová časť aplikácie využíva nasledujúce knižnice:
		\begin{description}
			\item[spring-core:] Pomocou anotácie @Autowired sú v aplikácii riešené závislosti väčšiny komponent.
			\item[spring-jdbc:] Prístup do databázy je realizovaný pomocou spring triedy NamedParameterJdbcTemplate, ktorá oproti jdbc pridáva možnosť prístupu k vygenerovanému id nového záznamu.
			\item[spring-webmvc:] Prístup k dátam a ovládaniu serverovej časti aplikácie je umožnený pomocou restovej api.
			\item[spring-boot-starter-web:] Táto závislosť umožňuje použitie pribaleného tomcat serveru. Ten sa spustí pri štarte aplikácie a sprístupní restovú api.
		\end{description}

	\myFigure{components}{Diagram komponentov aplikácie}{Diagram komponentov aplikácie}

	\subsection{Esper engine}
		Esper je v aplikácii implementovaný triedou EsperManager, ktorá poskytuje prístup ku konfiguračným nástrojom a sprístupňuje handler prichádzajúcich udalostí. Pri spustení aplikácie načítava konfiguračné údaje z databázy a inicializuje esper. Najdôležitejšou úlohou tejto triedy je správa schém a statementov.
		\begin{description}
			\item[Schémy] sú reprezentované XML (org.w3c.dom.Node) dokumentom. Oproti POJO reprezentácii umožňuje tento formát vytváranie schém počas behu čo je z pohľadu užívateľa nutnosťou. V prípade potreby by bolo možné použiť udalosti reprezentované pomocou java.util.Map alebo objektovým poľom, avšak to vyžaduje definovanie formátu prenosu a následné spracovanie do požadovaného formátu klientom, čo nie je veľmi intuitívne. Použitie XML reprezentácie klientovi umožní komunikovať priamo s esperom.
			
			\item[Udalosti] ktoré engine spracováva musia byť v XML formáte. Toto obmedzenie je zapríčinené XML reprezentáciou schém. Ako zdroje udalostí týmto eliminujeme adaptéry CSV a HTTP z esperio knižnice. Na prijímanie XML udalostí slúži rest služba. Tá spracováva vstupný stream dvoma spôsobmi:
			\begin{itemize}
				\item Ako jednotlivé udalosti, kde koreňový element udalosti reprezentuje názov schémy reprezentujúcej udalosť.
				\item Ako stream udalostí, kde je koreňový element nazvaný "events" a obaľuje jednotlivé udalosti definované v predošlom bode. Koreňový element je v tomto prípade ignorovaný.
			\end{itemize}
			
			\item[Statementy] pridávané do esper engine môžu obsahovať len meno a epl výraz. Preto je ku každému statementu priradený aj užívateľský objekt - statementBean s dodatočnými informáciami:
			\begin{itemize}
				\item ID ktoré unikátne identifikuje statement v rámci celej aplikácie, nie len daného esper provideru
				\item TTL hovoriaci ako dlho má byť nájdená udalosť perzistentná.
				\item STATE udávajúci či je konkrétny statement spustený alebo zastavený
			\end{itemize}
				
			
			Pri pridávaní statementu do esper engine je kontrolovaná unikátnosť mena v rámci daného esper providera. V prípade že existuje statement s rovnakým menom je meno doplnené o "--N", kde N značí najbližšie voľné celé číslo. Takto upravený statement je následne uložený.
		\end{description}
		
		Aplikácia definuje jeden globálny listener, ktorý je priradený všetkým statementom. Ten v prípade výskytu udalosti vyhovujúcej niektorému z definovaných statementov uloží udalosť vo forme obsahu json objektu do databázy.
		
		Esper umožňuje export výsledkov pomocou JSONRenderer a XMLRenderer, avšak tieto triedy produkujú formátovaný výstup, čo je v serverovej časti aplikácie neželané. Preto je konverzia realizovaná upravenými verziami týchto tried, ktoré nepridávajú formátovacie znaky a produkujú validné XML a JDON dokumenty. Predvolene je použitý upravený JSONRenderer.
		
		Pri ukladaní týchto týchto výsledkov do databázy je nastavený TTL, ktorý bol definovaný pri vytvorení statementu. Ak TTL pri vytvorení statementu nebolo definované použije sa predvolené nastavenie, kde sú výsledky persistentné až kým ich užívateľ manuálne neodstráni.

	\subsection{Maven}
		Zostavenie projektu je jednou z nevyhnutných súčastí tvorby java aplikácií. Na uľahčenie tohoto procesu je možné použiť viacero nástrojov, ktorých hlavnými predstaviteľmi sú maven, gradle a ant. Api komponent tejto aplikácie je zostavený pomocou nástroja maven. 
		
		Maven uľahčuje prácu vo viacerých oblastiach, a to \cite{web:maven-doc}:
		\begin{itemize}
			\item Uľahčenie prekladu aplikácie
			\item Poskytnutie jednotného riešenia pre zostavenie aplikácie
			\item Poskytnutie informácií o projekte
			\item Poskytnutie vzorov pre vývoj aplikácií
			\item Umožnenie transparentnej migrácie nových vlastností programu
		\end{itemize}
		Pre túto prácu je najdôležitejšia prvá oblasť. Vďaka mavenu nemusí distribúcia projektu obsahovať všetky knižnice závislostí, ani nemusia byť jednotlivo sťahované pri zostavovaní projektu. Všetky potrebné závislosti sú definované v súbore pom.xml a pri preklade automaticky stiahnuté z internetu. Tento súbor zároveň definuje vlastnosti projektu ako názov, verziu, verziu javy použitú pre zostavenie, spôsob generovania dokumentácie a iné detaily.
		
		Maven tiež umožňuje vytvárať v pom.xml súboroch závislosti a odkazovať sa na externé konfigurácie. Táto funkcionalita je v projekte využitá pri spring závislostiach, kde je verzia niektorých komponent definovaná v externom rodičovskom súbore. Rodičom spring závislostí projektu je spring-boot-starter-parent.
		
		%Väčšina funkcionality mavenu sa skladá z pluginov, k

\section{ThesisWeb}
	Web časť projektu slúži ako administračné rozhranie. Je riešené formou webovej aplikácie, ktorej základ je framework Ruby On Rails. Ten umožňuje rýchle vytváranie stránok, kde prevažuje prístup "convention over configuration". Dôležitou súčasťou frameworku sú gemy, ktoré reprezentujú závislosti projektu. Jedným z najpodstatnejších v tomto projekte je gem "her", ktorý zabezpečuje komunikáciu s Api časťou projektu.
	
	Grafická stránka tohoto projektu sa opiera o framework Bootstrap. Ten v základnej konfigurácii poskytuje html komponenty s vylepšeným UI. Tiež opravuje niektoré chyby kompatibility pri zobrazovaní stránok v rôznych prehliadačoch.
	
	Web nevyužíva žiadne persistentné úložisko dát. Všetky realizované zmeny sú posielané na restovú api, ktorá zmeny spracuje a uloží. Zobrazované údaje sú tiež získavané zo vzdialeného zdroja.
		
\section{UseCase}
	Na predvedenie možností použitia aplikácie som vytvoril 2 jednoduché programy. Prvý Twitter Stream demonštruje možnosť zasielania udalostí na server a Contacts Web predstavuje možnosť alternatívneho použitia esperu. 
	
	\subsection{Twitter Stream}
	Twitter Stream predstavuje možnosť zasielanie streamu udalostí na REST rozhranie serveru. Skript je napísaný v Ruby a demonštruje ako sa je možné pomocou necelých 30 riadkov kódu napojiť na Twitter Api a presmerovať vzorku tweetov na server. Prijatý tweet je vo forme objektu, ktorý je nutné pred preposlaním formátovať do XML dokumentu. Skript pred spustením odošle na server požiadavku pre registrovanie schémy tweetu, ktorá sa nachádza v samostatnom súbore.
	
	Podobne ako je predvedené v Twitter Stream je možné naprogramovať ďalšie generátory udalostí. Jedinou podmienkou je definovanie schémy (čo je možné aj pomocou administračného rozhrania) pred spustením generátora a streamovanie udalostí vo forme XML dokumentov na URL definovanú v administračnom rozhraní.
	
	\subsection{Contacts Web}
	Contacts Web je webová aplikácia, zostavená pomocou frameworku Sinatra a Bootstrap. Aplikácia obsahuje prezentačnú a aplikačnú vrstvu, databázová vrstva je riešená pomocou REST volaní na server. Vďaka tomu je možné logiku aplikácie zmestiť do necelých 30 riadkov kódu.
	
	Logika aplikácie pozostáva z obsluhy požiadavkov na vytvorenie nového kontaktu, zmazanie existujúceho a výpis existujúcich kontaktov. Všetky tieto požiadavky sú posielané na server.
	
	Možnosť filtrovania údajov pomocou preddefinovaných pravidiel.
	Deklaratívne programovanie.
	
\chapter{Inštalácia \& Použitie}
	Táto kapitola sa bude zaoberať názorným postupom, ktorý je potrebný k spusteniu aplikácie. Postup zahŕňa kroky od prípravy systému, inštaláciu aplikácie až po popis administračného rozhrania a názorné príklady použitia.
	
	\myTable{
		\begin{tabular}{ | l | r | r | }
			\hline
			Program & Verzia & Zdroj	\\
			\hline
			Windows	8	&	Professional N 64-bit	&	Microsoft DreamSpark	\\ \hline
			Java JDK	&	1.8.0\_31	&	www.oracle.com	\\ \hline
			Maven	&	3.2.3	&	maven.apache.org	\\ \hline
			Cassandra	&	2.1.2	&	cassandra.apache.org	\\ \hline
			RailsInstaller	&	3.1.0	&	railsinstaller.org	\\ \hline
			Node.js	&	0.10.31	&	nodejs.org	\\ \hline
		\end{tabular}
	}{Verzie programov použitých pri inštalácii}{table:program-verzia}
		
\section{Príprava systému}
	Aplikácia je k dispozícii vo forme zdrojových kódov. Pred použitím ju preto musíme skompilovať, nastaviť databázu a premenné prostredia a stiahnuť závislé knižnice. Táto sekcia sa zaoberá prípravou systému a prostredia ku spusteniu aplikácie.

\subsubsection{Operačný systém}
	Ako operačný systém bol použitý Windows 8 Professional N 64-bit. Táto verzia je pre študentov FIS dostupná prostredníctvom projektu DreamSpark - software spoločnosti Microsoft licencovaného pre akademické inštitúcie. Windows 8 bol nainštalovaný s predvolenými nastaveniami. Vzhľadom na minimálne požiadavky databázového systému Cassandra 2GB RAM by bolo vhodné použiť systém s minimálne 4GB RAM.

\subsubsection{Java}	% http://download.oracle.com/otn-pub/java/jdk/8u31-b13/jdk-8u31-windows-x64.exe
	Ako programovací jazyk serverovej časti aplikácie bola použitá java. Demonštračná kompilácia využíva aktuálnu verziu jdk1.8.0\_31. Je samozrejme možné použitie iných verzií, avšak knižnica esper správne pracuje len s niektorými z nich. Konkrétne nastáva problém pri pridaní schémy definovanej XML dokumentom do esper konfigurácie, kde vzniká výnimka:
	\begin{lstlisting}[label=lst:exjava,caption=Výnimka pri definovaní XML schémy v niektorých verziách javy]
	com.espertech.esper.client.ConfigurationException: Failed to read schema via URL 'null'
	\end{lstlisting}
	
	Testované verzie javy a ich kompatibilita je v nasledujúcej tabuľke:
	\myTable{
	\begin{tabular}{ | l | c | }
		\hline
		Verzia	&	Stav	\\
		\hline
		jdk1.7.0\_25	&	kompatibilná	\\ \hline
		jdk1.7.0\_71	&	nekompatibilná	\\ \hline
		jdk1.7.0\_72	&	kompatibilná	\\ \hline
		jdk1.8.0\_25	&	nekompatibilná	\\ \hline
		jdk1.8.0\_31	&	kompatibilná	\\ \hline
	\end{tabular}
	}{Kompatibilita verzií javy s esper xml schémami}{table:java-kompatibilita}
	
	Java bola nainštalovaná použitím predvolených nastavení, súčasťou ktorých je aj inštalácia JRE. Po nainštalovaní definujeme premennú prostredia JAVA\_HOME nasledovne:
	\begin{lstlisting}
	JAVA_HOME - C:\Program Files\Java\jdk1.8.0_31
	\end{lstlisting}
	Toto nastavenie je vyžadované nástrojom Maven, ktorý popisuje nasledujúca sekcia.
	
\subsubsection{Maven}	% http://maven.apache.org/
	Na zostavenie projektu bol použitý Maven vo verzii 3.2.3. Inštalácia pozostáva zo stiahnutia zip archívu (binárnej distribúcie) zo stránky projektu a rozbalenia do cieľového adresára. Pre jednoduchšie použitie je vhodné rozšíriť PATH o cestu k binárnym súborom nástroja, v tomto prípade:
	\begin{lstlisting}
	PATH = %PATH%;C:\Apache\apache-maven-3.2.3\bin
	\end{lstlisting}

\subsubsection{Cassandra}	% http://cassandra.apache.org/
	Cassandra je okrem binárnej verzie distribuovaná aj ako spustiteľný MSI inštalátor vďaka DataStax komunite. Tento spôsob inštalácie je pohodlnejší, avšak aktuálna verzia 2.1.2 pre operačný systém Windows po inštalácii nepracovala správne. Prejavili sa problémy ako chyby pri spustení spôsobené chybnou verziou knižnice jamm použitej pri kompilácii či poruchy Windows agenta.
		
	Pre projekt som kvôli týmto dôvodom použil binárnu distribúciu. Inštalácia tejto verzie je veľmi podobná inštalácii Mavenu, preberanému v predošlej sekcii. Na stránkach projektu stiahneme archív a rozbalíme ho do cieľového adresára. Tiež rozšírime systémovú premennú PATH o cestu k spúšťacím súborom takto:
	\begin{lstlisting}
	PATH = %PATH%;C:\Apache\apache-cassandra-2.1.2\bin
	\end{lstlisting}
	
	%TODO:CQL
	Zo stránok projektu je CQL

\subsubsection{RubyOnRails}	% http://railsinstaller.org/ 3.1.0
	Inštalácia frameworku RubyOnRails na windows platforme ja náročnejšia, pretože obsahuje viacero komponent, preto použijeme program RailsInstaller. Ten v použitej verzii obsahuje nasledujúce komponenty použité v projekte:
	\begin{itemize}
		\item Ruby 2.1.5 - interpretátor jazyka Ruby
		\item Rails 4.1 - webový framework pre administračné rozhranie
		\item Bundler - manažér závislostí projektu, funkciou podobný nástroju Maven
		\item Git - verzovací nástroj použitý prestiahnutie projektu z verejného repozitára
		\item DevKit - nástroj pre zostrojenie (build) natívnych C/C++ rozšírení na Windowse
	\end{itemize}

	Pri inštalácii použijeme predvolené nastavenia. Pri prvom použití nástroja bundler však narazíme na problém s certifikátmi:
	\begin{lstlisting}
	Error:SSL\_connect returned=1 errno=0 state=SSLv3 read server certificate B: certificate verify failed.
	\end{lstlisting}
	Riešenie pozostáva zo stiahnutia súboru obsahujúceho certifikáty certifikačných autorít. Ten následne sprístupníme pre ruby definovaním premennej prostredia \cite{web:certificate-fix}.
	\begin{lstlisting}
	ruby "%USERPROFILE%\Desktop\win_fetch_cacerts.rb"
	SSL_CERT_FILE = C:\RailsInstaller\cacert.pem
	\end{lstlisting}
		
	Problém s certifikátmi sa týmto vyriešil. Pri vytvorení a spustení prvej aplikácie ale narazíme na ďalšu výnimku:
	\begin{lstlisting}
	ExecJS::ProgramError - TypeError: Object doesn't support this property or method
	\end{lstlisting}
	Táto je spôsobená nekompatibilitou predvoleného windows javascript runtime a rails prostredia pri spracovaní assets (css a js) súborov. Jedným z riešení tohoto problému je inštalácia platformy Node.js. Pre potreby projektu postačuje predvolená inštalácia.
	
	Týmto krokmi sme pripravili prostredie pre spustenie aplikácie. Podrobný postup pre spustenie je popísaný v nasledujúcej sekcii.

\section{Spustenie aplikácie}
	Pred samotným spustením aplikácie je potrebné ju stiahnuť a skompilovať. Ak bol dodržaný postup inštalácie popísaný v predchádzajúcej sekcii tak sú všetky potrebné nástroje k dispozícii. Pokračujeme teda stiahnutím zdrojových kódov kompletného projektu z git repozitára. V príkazovom riadku:
	\begin{lstlisting}
	git clone https://github.com/kravciak/thesis-diploma-code.git
	\end{lstlisting}
	Kompletný projekt obsahuje 4 samostatné komponenty. Každá z nich sa spúšťa samostatne, avšak všetky sú závislé na Api, a pred použitím je nutné nakonfigurovať schémy a statementy - na čo slúži komponenta Web. Postup spustenia je preto nasledovný:
	
\subsubsection{Api}
	Predtým než je spustená Api je nutné spustiť cassandra databázu. Tá je spustiteľná z bin priečinka inštalácie súborom cassandra.bat. Pre spustenie vyžaduje administrátorské oprávnenia. Cassandru je tiež možné nainštalovať ako service pridaním parametra "install". Pri správnom spustení sa na konzole zobrazí text:
	\begin{lstlisting}
	Starting listening for CQL clients on localhost/127.0.0.1:9042...
	Binding thrift service to localhost/127.0.0.1:9160
	Listening for thrift clients...
	\end{lstlisting}
	
	Api nájdeme v priečinku ThesisApi. Je závislá na viacerých knižniciach, ktoré sú automaticky stiahnuté nástrojom Maven pri zostavovaní. Po úspešnom zostavení aplikáciu spustíme.
	\begin{lstlisting}
	mvn install
	mvn exec:java
	\end{lstlisting}
	Pri úspešnom spustení sa Api napojí na cassandru, spustí webový server apache a načíta konfiguráciu z databázy.
	
\subsubsection{Web}
	Administračné rozhranie sa nachádza v priečinku ThesisWeb. Podobne ako Api aj Web je závislý na viacerých gemoch. Tie sú stiahnuteľné pomocou bundleru, ktorého inštalácia je popísaná v predchádzajúcej sekcii. Po stiahnutí závislostí môžeme web spustiť.
	\begin{lstlisting}
	bundle install
	rails s
	\end{lstlisting}
	Po spustení je webové rozhranie dostupné na adrese \url{http://localhost:3000/}.
		
\subsubsection{Twitter Stream}
	Twitter stream závisí na dvoch gemoch, builder a twitter gem. Twitter gem slúži na prijatie vzorky tweetov a builder na transformáciu tweetu do formy xml. Obe nainštalujeme pomocou príkazu gem install, ktorý je dostupný ako súčasť ruby.
	\begin{lstlisting}
	gem install twitter builder
	ruby generator.rb
	\end{lstlisting}
	Keďže twitter stream sa pred spustením preposielania tweetov pripojí na Api (a ak je Api offline vyhlási výnimku) je nutné ich spustiť v tomto poradí.
	
\subsubsection{Contacts Web}
	Závislosti contacts webu nainštalujeme rovnako ako v prípade twitter streamu.		
	\begin{lstlisting}
	gem install sinatra faraday gyoku json
	ruby client.rb
	\end{lstlisting}
	Pre používanie je potrebné mať spustené Api.

\section{Použitie}

	\subsection{ThesisApi}
	Serverová časť aplikácie je prístupná prostredníctvom restovej api počúvajúcej na porte 80. Jej použitie teda spočíva z zavolaní správnej URL so správnymi parametrami. Po spustení je aplikácia predvolene prístupná na adrese http://localhost:8080/. Vo výpise \ref{lst:rest-api} sú zobrazené príklady volaní, ktoré server akceptuje.

	\begin{lstlisting}[label=lst:rest-api,caption=Príklad volaní REST API]
	Výpis prvých 100 schém
	http://localhost:8080/schemas?offset=0&limit=100
	
	Výpis prvých 100 statementov
	http://localhost:8080/statements?offset=0&limit=100
	
	Zobrazenie výsledkov
	http://localhost:8080/statements/53/results
	\end{lstlisting}
	
	Každá metóda má definovaný spôsob prístupu, teda jedinečnú kombináciu URL a metódy volania (GET, POST, DELETE). Tabuľka \ref{table:rest-urls} poskytuje kompletný prehľad mapovania http metódy a url na funkciu, ktorá obsluhuje dané volanie.	Funkcie sú rozdelené do štyroch hlavných skupín podľa oblasti, ktorú obsluhujú. Takisto ich mapovanie zodpovedá konkrétnej oblasti.
	
	\myTable{
		\begin{tabular}{ | l | l | l | }
			\hline
			HTTP metóda & Funkcia & URL	\\
			\hline
			
			GET    & ping()          & /esper/ping                                           \\
			POST   & processStream() & /esper/events                                         \\
			&&\\
			GET    & get()           & /schemas/\{id\}                                       \\
			GET    & getAll()        & /schemas                                              \\
			POST   & create()        & /schemas                                              \\
			DELETE & delete()        & /schemas/\{id\}                                       \\
			&&\\
			GET    & control()       & /statements/\{id\}/control                            \\
			GET    & get()           & /statements/\{id\}                                    \\
			GET    & getAll()        & /statements                                           \\
			POST   & create()        & /statements                                           \\
			DELETE & delete()        & /statements/\{id\}                                    \\
			&&\\
			DELETE & deleteAll()   & /statements/\{statement\_id\}/results                 \\
			DELETE & deleteOne()   & /statements/\{statement\_id\}/results/\{uuid\}        \\
			GET    & exportOne()   & /statements/\{statement\_id\}/results/\{uuid\}/export \\
			GET    & exportAll()   & /statements/\{statement\_id\}/results/export          \\
			GET    & getOne()      & /statements/\{statement\_id\}/results/\{uuid\}        \\
			GET    & getAll()      & /statements/\{statement\_id\}/results                 \\
			GET    & count()       & /statements/\{statement\_id\}/results/count           \\
			GET    & replayAll()   & /statements/\{statement\_id\}/results/replay          \\
			GET    & replayOne()   & /statements/\{statement\_id\}/results/\{uuid\}/replay \\
			\hline
		\end{tabular}
	}{Mapovanie funkcií na REST URL}{table:rest-urls}
	
	Obrázok \ref{image_api-funkcie} predstavuje parametre funkcií REST rozhrania. Rovnako ako tabuľka \ref{table:rest-urls} je rozdelený do štyroch skupín predstavujúcich triedy obsahujúcu konkrétnu funkciu. Ich vzájomnou kombináciou sme schopní identifikovať všetky REST volania, ktoré umožňuje serverová časť aplikácia spracovať.

	\myFigure{api-funkcie}{Parametre funkcií REST rozhrania}{Parametre funkcií REST rozhrania}

	%TODO Formát prijímaných udalostí udalosti, spracovanie streamu umožňuje veľké súbory
	%TODO Streamovanie je realizované pomocou Faraday::UploadIO komponentu aplikácie.

	\subsection{ThesisWeb}
	Hlavná obrazovka zobrazuje tri oblasti esperu, ktoré je možné pomocou aplikácie obsluhovať - správu schém, správu statementov a formulár pre odosielanie udalostí. Po kliknutí na jednu z nich sa zobrazí obrazovka s rozšíreným menu \ref{image_menu}, ktoré navyše obsahuje možnosť prechodu na zobrazenie výsledkov nájdených konkrétnym statementom.
	\myFigure{gui-menu}{Menu aplikácie rozšírené o zobrazenie výsledkov}{Menu aplikácie}
	
	Pod týmto menu sa nachádza navigačný panel s informáciou o ceste k práve zobrazovanej stránke. Po jeho pravej strane sú ovládacie tlačidlá na pridávanie, mazanie, prípadne export elementov na stránke. Súčasťou navigačného panelu je na stránke schém a statementov vyhľadávanie, ktoré umožňuje filtrovať pomocou mena schémy alebo statementu.
	
	
	Pod navigačným menu sa nachádza samotný obsah stránky. Nasledujúci text popisuje každú zo štyroch hlavných sekcií.
	\begin{description}
		\item[Schémy:] Na hlavnej obrazovke je možné vidieť zoznam schém a pri niektorých z nich číslo. Toto číslo reprezentuje počet statementov závislých na konkrétnej schéme. Pro rozkliknutí schémy sa zobrazí jej detail. Na ľavej strane sú zobrazené dodatočné informácie vrátane zoznamu statementov závislých na danej schéme. Na pravej strane obrazovky je konkrétna XML schéma. Pomocou ovládacích tlačidiel je možné túto schému exportovať alebo zmazať v prípade že neobsahuje žiaden na nej závislý statement.

		Pri pridávaní je nutné vyplniť meno schémy, koreňový element umožňujúci esper engine jedinečne identifikovať schému (zhodný s koreňovým elementu definície schémy) a XML definíciu schémy.

		\item[Statementy:] Podobne ako v prípade schém nájdeme na hlavnej obrazovke zoznam statementov a pri každom z nich počet udalostí, ktoré mu vyhovujú. Po rozkliknutí sa zobrazí detail statementu. Na tejto stránke je zaujímavá z popisu statementu na ľavej strane URL adresa pomocou ktorej sú výsledky dostupné priamo zo serverovej API a odkaz pre zobrazenie výsledkov.
		Na pravej strane je zobrazená schéma statementu s menami polí ich dátovými typmi. Nad ňou je panel pre opätovné preposlanie uložených výsledkov do esper engine. Pri preposlaní je možné zvoliť si počiatočný a konečný dátum ohraničujúci ktoré výsledky sa majú preposlať a meno udalosti pod ktorým sa majú odoslať. Tlačidlo medzi poliami ohraničujúcimi časový úsek umožňuje spočítať vyhovujúce udalosti.
		Ovládací panel statementu umožňuje spustiť, pozastaviť alebo zmazať daný statement. Pozastavené statementy nevyhľadávajú nové udalosti
		\myFigure{gui-replay-form}{Formulár preposielania historických udalostí na esper engine}{Formulár preposielania historických udalostí}
		
		Pri vytváraní novej schémy je potrebné zadať jej meno, umožňujúce jedinečne identifikovať statement v esper engine. V prípade že už existuje statement s rovnakým menom je na jeho koniec automaticky pridané číslo vo forme "--n". Voliteľne je možné zadať TTL, ktoré špecifikuje ako dlho majú byť výsledky statementu držané v databáze. Posledné textové pole obsahuje text vytváraného statementu. Novo vytvorený statement je predvolene v spustenom stave.
		
		\item[Streamer:] Pre posielanie nových udalostí existuje viacero spôsobov, jedným z nich je odoslanie udalostí pomocou administračného rozhrania na tejto stránke. V hornej časti je URL, ktorú je možné použiť na odosielanie udalostí priamo na API serverovej časti. Štruktúra odosielanej udalosti je bližšie popísaná v použití serverovej časti aplikácie a esper dokumentácii.
		Pre odoslanie udalostí je možné použiť formulár alebo udalosti načítať zo súboru. Obe možnosti sa dajú použiť súčasne. Pri načítaní zo súboru sú odosielané udalosti streamované, čo umožňuje posielať veľké objemy dát.

		V ovládacích prvkoch stránky nájdeme tlačidlo pre odoslanie vzorky dát. Tieto sú načítané zo súboru a vyhovujú prednastaveným schémam a statementom. Ich použitie uľahčuje pochopenie fungovania aplikácie.
		
		\item[Výsledky] Pre zobrazenie výsledkov musíme najprv vybrať statement, ktorého výsledky chceme vidieť. Na stránku výsledkov je možné sa dostať z detailu statementu kliknutím na odkaz s počtom výsledkov alebo na poslednú položku z menu.
		Na základnej stránke je tabuľka so zoznamom výsledkov. Tieto je možné stránkovať, avšak z dôvodu použitia NoSQL databázy nie je možné preskočiť na konkrétnu stránku. Je to zapríčinené tým, že použitá databáza nepozná klauzulu OFFSET a k výsledkom je tak možné pristupovať len špecifikáciou ich primárneho kľúča, alebo vzťahu k inému primárnemu kľúču.
		Po rozkliknutí konkrétneho výsledku sa na ľavej strane zobrazí jeho detailný popis, v ktorom je možné vidieť napríklad statement ktorému výsledok patrí alebo čas za ktorý expiruje. Pod týmito metadátami je daný výsledok zobrazený v JSON formáte.
		
		Podobne ako na stránke statementu, kde je možné preposlať všetky výsledky je možné preposlať jeden konkrétny výsledok pomocou panelu v pravej časti tejto stránky.

		Vďaka ovládacím prvkom je možné exportovať alebo zmazať všetky výsledky na hlavnej stránke, alebo konkrétny výsledok pri zobrazení jeho detailu. V navigačnom paneli je zobrazené aj UUID výsledku.
	\end{description}

	
	\myFigure{gui-breadcrumbs}{Menu aplikácie rozšírené o zobrazenie výsledkov}{Menu aplikácie}
	
	\myFigure{gui-schema}{Menu aplikácie rozšírené o zobrazenie výsledkov}{Menu aplikácie}
	

názorné priklady
- 3 sposoby vkladania dat - zo suboru / handleru / webova aplikacia
- moznost prehravania historickych dat (vysledokv) ako vstup
\chapter*{Záver}
\addcontentsline{toc}{chapter}{\protect\numberline{}Záver}

kedze esper koncept nie je vseobecne znamy a pri spusteni som narazil na problemy - nazorna instalacia a priklady pouzitia


\emptydoublepage

% =========================================================
% ==================== Zoznam skratiek ====================
\begin{acronym}
	\acro{CEP}{Complex event processing}
	\acro{JDBC}{Java database connectivity technology}
	\acro{CQL}{Cassandra Query Language}
	%\acro{}{}
	%\acro{}{}
	%\acro{}{}
	%\acro{}{}
	%\acro{}{}
	%\acro{}{}
	%\acro{}{}
	%\acro{}{}
	%\acro{}{}
	%\acro{}{}
	%\acro{}{}
	%\acro{}{}
\end{acronym}


%--------------------------------------------------------------
\backmatter
%Bibliografia
%\phantomsection
%\nocite{*}
\bibliographystyle{ieeetr}
\bibliography{bibliografia}	\emptydoublepage

\listoffigures \emptydoublepage
\listoftables

\end{document}

% Dlhe url v bibliografii
%http://tex.stackexchange.com/questions/10924/underfull-hbox-in-bibliography
