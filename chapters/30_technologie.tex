\chapter{Technológie}
\section{Esper}
	Esper je komponenta, ktorá umožňuje spracovanie komplexných udalostí \ac{CEP}. Umožňuje vývoj aplikácií spracovávajúcich veľké množstvo udalostí - v reálnom čase ako aj historických. Tieto udalosti je možné filtrovať a analyzovať podľa potreby a reagovať v reálnom čase na predom definované stavy.  Esper je dostupný v troch verziách:
	\begin{description}
		\item[Esper] je dostupné ako open source s možnosťou komerčnej podpory. Táto verzia obsahuje základ potrebný pre použitie esperu, užívateľ však musí jednotlivé statementy, schémy a nastavenia realizovať programovo. Je preto náročný na použitie pre ľudí, ktorí nevedia programovať. Riešenie je vhodné pre firmy, ktoré buď nevyužijú platenú verziu alebo majú špecifické požiadavky na výsledný produkt a sú schopné túto verziu podľa svojich potrieb upraviť.
		
		\item[Esper HA] je riešenie umožňujúce vysokú dostupnosť Esperu. Zabezpečuje že stav je po vypnutí alebo havárii obnoviteľný. Statementy, schémy a iné nastavenia si EsperHA pri reštarte uchováva, čo je výhoda oproti predchádzajúcej verzii - kde je nutné tieto úkony riešiť programovo. Táto verzia je vhodná pre projekty závislé na vysokej dostupnosti esperu a subjekty, pre ktoré je kritická neustála kontrola prichádzajúcich udalostí.
		EsperHA je spoplatnený, dostupná je trial len verzia, pre ktorej použitie je nutné identifikovať sa ako spoločnosť. Cena nie je na webových stránkach dostupná.
		
		\item[Esper Enterprise Edition] je kompletný produkt "na kľúč", obsahujúci všetky komponenty potrebné pre nasadenie do podniku. V jednom balíku je obsiahnuté GUI pre správu esperu, restové služby poskytujúce prístup zvonku, \ac{EPL} editor, zobrazovacie nástroje umožňujúce kontinuálne zobrazenie výsledkov v grafoch a tabuľkách. EsperEE je možné skombinovať s EsperEA pre dodatočné zabezpečenie vysokej dostupnosti. EsperEE je spoplatnený, rovnako ako pri EsperEA je dostupná trial verzia po splnení určitých podmienok. Cena nie je dostupná, tieto dve dve riešenia sú určené predovšetkým pre podnikový sektor.
	\end{description}
	
	Pre tento projekt je použitá verzia Esper, ktorú som rozšíril o prístup k základným funkciám pomocou restovej api a persistenciu niektorých nastavení a nájdených výsledkov. Táto verzia je dostupná pod GNU General Public License (GPL) (GPL v2).

	\subsection{Procesný model}

	\subsection{Typy udalostí}
	Každá udalosť spracovávaná esperom je definovaná schémou, takzvaným typom udalosti. Tie môžu byť definované pri štarte aplikácie, alebo programovo počas behu. EPL tiež obsahuje klauzulu CREATE SCHEMA umožňujúcu definovanie typu udalosti pomocou EPL. Prehľad základných typov udalostí je v nasledujúcej tabuľke.

	\myTable{
	\begin{tabular}{ | l | p{10cm} | }
		\hline
		Trieda	&	Popis	\\ \hline
		java.lang.Object	&	Akýkoľvek Java POJO (plain-old java object) s getter metódami. Takáto definícia je najjednoduchšia na úkor možnosti úprav počas behu programu.	\\ \hline
		java.util.Map	&	Udalosti definované ako implementácia java.util.Map interface, kde každá hodnota záznamu je vlastnosť udalosti.	\\ \hline
		Object[] (pole objektov)	&	Udalosti definované objektovým poľom, kde každá hodnota poľa je vlastnosť udalosti.	\\ \hline
		org.w3c.dom.Node	&	XML objektový model dokumentu popisujúci štruktúru udalosti.	\\ \hline
	\end{tabular}
	}{Základné typy udalostí}{table:event-types}
	
	Definície typu udalosti sú rozšíriteľné zásuvnými modulmi. Aplikácia môže používať kombináciu týchto typov, nemusí všetky typy definovať jedným spôsobom. Definície typov udalostí je možné reťaziť, kedy typom udalosti môže byť iná komplexná udalosť.
	
	Z dôvodu nutnosti pridávania a mazania udalostí počas behu programu nemôže byť v mojej implementácii použitá definícia typu pomocou POJO. A pretože klient musí mať možnosť definovať typ, bola zvolená definícia pomocou XML schémy. Jednoduchý príklad schémy udalosti znázorňuje nasledujúci výpis. 
	
	\begin{lstlisting}[label=lst:sample-schema,caption=Jednoduchý príklad XML schémy udalosti]
<?xml version="1.0" encoding="UTF-8"?>
<xs:schema xmlns:xs="http://www.w3.org/2001/XMLSchema">
	<xs:element name="TweetEvent">
		<xs:complexType>
			<xs:sequence>
				<xs:element name="username" type="xs:string"></xs:element>
				<xs:element name="message" type="xs:string"></xs:element>
			</xs:sequence>
		</xs:complexType>
	</xs:element>
</xs:schema>		
	\end{lstlisting}

	Po definovaní typu udalosti sa na ne môžeme odkazovať klauzulou FROM v EPL statementoch. Tie sú bližšie popísané v nasledujúcej sekcii.

	\subsection{Event Processing Language}
		\ac{EPL} je jazyk umožňujúci definovanie statementov a patternov v CEP. Syntaxou je podobný SQL. Obsahuje klauzuly SELECT, FROM, WHERE, GROUP BY, HAVING a ORDER BY. Namiesto tabuliek však pracuje so streamami, kde riadok tabuľky nahrádza prichádzajúca udalosť. Streamy udalostí je možné spájať pomocou joinov, filtrovať a agregovať.
		
		Klauzula INSERT INTO je použiteľná na presmerovanie udalostí na iný stream pre dodatočné spracovanie. Klauzula UPDATE slúži na úpravu vlastností udalosti a je aplikovaná pred spracovaním statementu. 
		
		EPL statementy môžu obsahovať definíciu náhľadov (view). Tieto majú viacero funkcií, ako okná náhľadu na stream udalostí, tvorenie štatistík v vlastností udalosti alebo zoskupovanie udalostí. Náhľady môžu byť reťazené. Zabudované náhľady sú napríklad win:length - aplikuje statement na definovaný počet udalostí, win:time - aplikuje statement na udalosti obmedzené časovo alebo std:lastevent - ktorý obsahuje poslednú prijatú udalosť.
		
		EPL definuje koncept pomenovaných okien (named windows), ktoré slúžia ako štruktúra uchovávajúca udalosti. Je možné do nej vkladať nové udalosti a mazať staré. Výhodou tejto štruktúry je možnosť jej použitia viacerými statementami, pretože je globálna, teda zdieľaná v rozsahu daného service providera.	
		
		Pomocou EPL môžeme tiež definovať premenné, ktoré slúžia na vkladanie parametrov do statementu a definovanie schém udalostí. Tými sa zaoberá nasledujúca sekcia.
		
		\subsubsection{Syntax}
		Každý EPL statement musí obsahovať minimálne klauzulu SELECT a FROM.

		SELECT klauzula môže obsahovať náhradný znak * alebo vymenovať požadované vlastnosti udalosti. Tiež definuje typ výslednej udalosti publikovanej statementom. SELECT poskytuje aj nepovinné klauzuly istream (input), irstream (input \& remove) a rstream (remove), ktoré definujú streamy, ktorých udalosti sa majú poslať na UpdateListener a observery statementu. Prednastavené je použitie nastavenia istream.
		
		FROM klauzula špecifikuje jeden alebo viac streamov, pomenovaných okien alebo tabuliek (od verzie esper 5.1). Tie môžu byť pomenované klauzulou AS. Pre join je potrebné definovať viacero streamov. Podporovaný je tiež join so relačnou databázou ako zdrojom dát. To je možné využiť napríklad na prístup k historickým dátam.

		Patterny sú výrazy, ktoré hľadajú zhodu podľa definovaného vzoru. EPL umožňuje definovať pattern ako samostatný výraz alebo ako súčasť statementu. Pattern sa môže vyskytovať kdekoľvek v klauzule FROM, vrátane join. Môžu preto byť použité v kombinácii s klauzulami WHERE, GROUP BY, HAVING a INSERT INTO.

		V nasledujúcich výpisoch sú príklady statementov, zobrazujúcich príklady syntaxe popisovanej v predchádzajúcom texte.
				
		\begin{lstlisting}[label=lst:epl-simple,caption=Jednoduchý EPL statement]
		select * from TweetEvent.win:time(60 sec) where message='happy'
		\end{lstlisting}
		
		\begin{lstlisting}[label=lst:epl-join,caption=Jednoduchý EPL statements použitím join]
		select * from TickEvent.std:unique(symbol) as t, NewsEvent.std:unique(symbol) as n
		where t.symbol = n.symbol
		\end{lstlisting}
		
		\begin{lstlisting}[label=lst:epl-pattern,caption=EPL statement s použitím patternu]
		select a.custId, sum(a.price + b.price)
		from pattern [every a=ServiceOrder -> 
			b=ProductOrder(custId = a.custId) where timer:within(1 min)].win:time(2 hour) 
		where a.name in ('Repair', b.name)
		group by a.custId
		having sum(a.price + b.price) > 100
		\end{lstlisting}
				
		\subsubsection{Objektový model}	
		Objektový model je sada tried poskytujúcich objektovú reprezentáciu statementu alebo patternu. Tá umožňuje zostrojiť, zmeniť alebo získať údaje z EPL statementov a patternov na vyššom stupni ako pri práci s textovou reprezentáciou. Objektový model pozostáva z objektového grafu, ktorého prvky je jednoducho prístupné. Táto forma tiež umožňuje kontrolu EPL syntaxe pred pridaním statementu alebo patternu do esperu. Objektový model umožňuje plný export do textovej formy a naopak.
		
		Rovnako ako v textovej reprezentácii sú v objektovej reprezentácii klauzuly SELECT a FROM povinné.
		
	\subsection{Api}
		Esper pre svoje ovládanie neposkytuje grafické rozhranie. Na komunikáciu používa api, ktorá definuje tieto primárne rozhrania:
		\begin{itemize}
			\item Rozhranie udalostí a ich typov
			\item Administrátorské rozhranie na vytváranie a správu EPL statementov a patternov a definovanie konfigurácie esperu
			\item Runtime rozhranie, ktoré slúži na posielanie udalostí do esperu, definovanie premenných a spúšťanie on-demand výrazov.
		\end{itemize}
		
		\subsubsection{EP Service Provider}
		EPServiceProvider reprezentuje konkrétnu inštanciu esperu. Každá takáto inštancia je nezávislá od ostatných a má svoje vlastné administrátorské a runtime rozhranie. Pri prístupe umožňuje rozhranie voľbu "getDefaultProvider" bez parametrov, ktorá vráti predvolenú inštanciu, alebo "getProvider" s textovým parametrom URI identifikujúcim konkrétnu inštanciu. Tá je vytvorená ak ešte neexistuje. Opakované volania s rovnakým URI vracajú stále rovnakú inštanciu.
		
		\subsubsection{EP Administrator}
		EPAdministrator umožňuje registrovanie EPL statementov, patternov alebo ich objektovej reprezentácie do esperu a to metódami createPattern, createStatement a create pre objektový model. Tieto funkcie poskytujú voliteľné parametre umožňujúce definovať meno statementu a užívateľský objekt, ktorý je v tejto aplikácii využitý na ukladanie dodatočných informácií o statemente - napríklad definovanie TTL pri ukladaní výsledkov do databázy.
		
		Po registrácii nového EPL výrazu rozhranie vracia inštanciu vytvoreného EPStatement, pomocou ktorej môžeme ovládať už vytvorený statement alebo pristupovať k výsledkom. V tejto aplikácii je z ovládacích funkcií implementovaná stop() a start(), ktoré definujú, či je statement aktívny.
		
		Esper poskytuje tri možnosti ako pristupovať k výsledkom konkrétneho statementu. Tieto je možné rôzne kombinovať. Možnosti sú predstavené v nasledujúcom výpise:
		\begin{description}
			\item[Listener] V prvom prípade aplikácia poskytuje implementáciu rozhraní UpdateListener alebo StatementAwareUpdateListener vytváranému statementu. Takýto listener bude následne notifikovaný pri výskyte novej udalosti a metóde update bude predaná inštancia EventBean, ktorá obsahuje výsledok statementu.
			\item[Subscriber] Týmto spôsobom esper posiela výsledky na definovaný subscriber. Je to najrýchlejšia možnosť, pretože esper predáva typované výsledky priamo do objektov aplikácie, nemusí teba zostavovať inštancie EventBean ako v predošlom prípade. Nevýhodou je že statement môže mať registrovaný maximálne jeden subscriber, naproti predošlému spôsobu, kde umožňoval definovať viacero listenerov.
			\item[Pull Api] Týmto spôsobom aplikácia pristupuje k výsledkom on-demand spôsobom, kde si jednorázovo žiadosťou o výsledky daného statementu získa zoznam EventBean prístupný pomocou iterátora. Toto je využiteľné v prípade kedy aplikácia nevyžaduje nepretržité predávanie nových výsledkov v real-time.
		\end{description}
		V tomto projekte bol použitý prvý spôsob listenera. Aplikácia v tomto prípade použije implementáciu rozhrania StatementAwareUpdateListener, ktorá je registrovaná pro vytváraní nového statementu metódou addListener. Vďaka použitiu rozhrania StatementAwareUpdateListener a nie UpdateListener získava aplikácia prístup k statementu, ktorý konkrétnu udalosť vyprodukoval, pre všetky statementy môže byť preto definovaný jediný spoločný listener.
		
		Esper podporuje tiež spracovanie udalostí, ktoré nevyhoveli žiadnemu statemenentu. Tieto výsledky získame registrovaním implementácie rozhrania UnmatchedListener. 
		
		\subsubsection{EP Runtime}
		EPRuntime rozhranie slúži na odosielanie nových udalostí do esperu k spracovaniu, nastavenie a prístup k hodnotám premenných a spúšťanie on-demand EPL výrazov. Na odosielanie nových udalostí slúži metóda sendEvent, ktorá je preťažená. Typ parametra tejto metódy indikuje typ udalosti odosielanej do esperu. Tieto typy boli bližšie popísané v predchádzajúcich sekciách.
		
		V prípade použitia XML definície typov udalostí sa pri spracovaní prichádzajúcej udalosti skontroluje že meno koreňového elementu prichádzajúcej udalosti je zhodné s menom typu udalosti definovanej XML schémou.
		
		Ak aplikácia nepozná EPL výrazy dopredu alebo nevyžaduje streamovanie výsledkov, je možné prostredníctvom EPRuntime spúšťať jednorázové výrazy. Tieto nie sú permanentné, po ich vykonaní je výsledok okamžite predaný aplikácii pre spracovanie. Použitie nachádzajú napríklad s pomenovanými oknami a tabuľkami, ktoré je možné indexovať pre zrýchlenie prístupu.		

\section{Cassandra}
	http://planetcassandra.org/what-is-apache-cassandra/
	
	Cassandra je databázový projekt, ktorý pôvodne vznikol vo firme Facebook. Neskôr bol zverejnený ako open-source a v roku 2009 bol prijatý do Apache inkubátora. V roku 2010 získal top prioritu a je naďalej vyvíjaný a dostupný pod Apache 2.0 licenciou.
	
	Databáza je distribuovaná databáza, ktorá umožňuje spracovanie a uchovávanie veľkého množstva dát rozložených na veľkom počte menej výkonných serverov. Táto architektúra zároveň poskytuje vysokú dostupnosť dát pri zabezpečení proti strate dát pri výpadku niektorého zo serverov. Cassandra je navrhnutá na použitie veľkého počtu počítačov (v ráde stoviek) podľa možností rozložených v rôznych častiach sveta.
	
	Aj keď v mnohom pripomína Cassandra relačnú databázu, nepodporuje plne relačný model. Namiesto toho poskytuje klientom jednoduchý dátový model, ktorý je možné dynamicky upravovať.
	
	Cassandra bola navrhnutá pre beh na cenovo dostupnom hardware a podporuje rýchly zápis veľkého množstva dát pri zachovaní efektívnosti prístupu k nim. Týmto pomáha znižovať firemné náklady na hardware.
	
	Vďaka týmto vlastnostiam je cassandra využívaná množstvom známych firiem, medzi ktoré patrí napríklad CERN, eBay, GitHub, Netflix, Twitter alebo Cisco. Veľké produkčné nasadenia obsahujú stovky TB dát v klastroch obsahujúcich stovky serverov. Pri porovnaní výkonnosti s ostatnými NoSQL databázami  Cassandra získava výborné výkonnostné výsledky aj vďaka svojej jednoduchej architektúre, porovnanie s alternatívami vidíme na obrázku \ref{image_cassandra-performance}
	
	\myFigure{cassandra-performance}{Porovnanie výkonnosti cassandry s alternatívnymi NoSQL databázami}{Výkonnosť Cassandra databázy}
	
\section{Frameworky}
	Pre uľahčenie vývoja projektov je dnes bežné použitie frameworku. Rozsahom malé aplikácie často vyžadujú použitie funkcionality (prihlasovanie, odosielanie mailov, práca s databázou, zjednotenie grafického zobrazenia pre rôzne platformy, správu závislostí), ktorej implementácia je zdĺhavá a v málo prípadoch lepšia ako pri opätovnom použitý riešenia na to stavaného. Pri vývoji jednotlivých častí tohoto projektu boli použité voľne dostupné frameworky, ktorých bližším popisom sa zaoberá táto sekcia.

	\subsection{Spring}
	Spring je framework napísaný v jave, distribuovaný pod Apache License verzie 2.0. Spring pozostáva z viacerých projektov zameraných na riešenie konkrétnych problémov vývoja aplikácie. V ThesisApi časti tejto aplikácie boli použité nasledujúce spring komponenty:
	\begin{description}
		\item[Spring Framework] svojou funkcionalitou rieši základné oblasti vývoja java aplikácií. Obsahuje základnú podporu pre injektovanie závislostí, správu transakcií, vývoj webových aplikácií alebo prístup k dátam. Táto funkcionalita je rozdelená do komponentov, z menovaných funcionalít to sú spring-core, spring-jdbc a spring-webmvc.
		
		\item[Spring Boot] je spôsob ako urýchliť vývoj spring aplikácií. Rovnako ako Ruby On Rails sa prikláňa k prístupu convention over configuration. Z využitej funkcionality je dôležitý hlavne webový server Tomcat, ktorý tento komponent obsahuje. Vďaka nemu nie je nutné robiť deploy war súborov serverovej časti aplikácie, stačí jednoducho zdrojové kódy preložiť a spustiť. Je možné tiež využiť komponenty obsiahnuté v rodičovskom pom súbore. Spring boot automaticky nakonfiguruje spring aplikáciu bez nutnosti XML konfiguračných súborov s predvolenými nastaveniami.
	\end{description}
		
	\subsection{Ruby On Rails}
	Mottom frameworku Ruby On Rails je snaha o dosiahnutie spokojnosti programátora a udržateľnú produktivitu. Framework uprednostňuje prístup convention over configuration, čo znamená že konfigurácia je preddefinovaná a ak programátor používa predom dohodnuté konvencie postačujú minimálne úpravy na dosiahnutie požadovaného výsledku. Konfiguráciu je samozrejme možné podľa potreby upraviť. Framework je voľne dostupný a distribuovaný pod MIT licenciou.
	
	Vytvorenie a spustenie nového projektu pozostáva z jednoduchej série príkazov zobrazených vo výpsie \ref{lst:ror-create}. Všetky závislosti sú automaticky stiahnuté pri vytvorení projektu nástrojom bundler.
	\begin{lstlisting}[label=lst:ror-create,caption=Príklad vytvorenia a spustenia projektu v Ruby On Rails]
		rails new myweb
		cd myweb
		rails s
	\end{lstlisting}

	Adresárová štruktúra projektu je prispôsobená MVC architektúre. Jednotlivé komponenty nájdeme v adresároch model, view a controller.
	\begin{description}
		\item[Model] je vrstva reprezentujúca údaje a ich logiku. Umožňuje definovať objekty, ktorých dáta vyžadujú uloženie v databáze. Vlastnosti týchto objektov sú mapované na relačné dáta. Model je možné definovať pomocou migrácií, ktoré po spustení upravujú štruktúru databázy a umožňujú rollback k predchádzajúcemu stavu. V modeloch je možné definovať kontroly vstupných dát, asociácie, funkcie na rozšírenie prístupu k údajom alebo spätné volania v závislosti na vykonávanej akcii.
		\item[View] adresár je ďalej rozdelený do podadresárov reprezentujúcich jednotlivé controllery. Dôležitým je tiež adresár layout, ktorý ako názov napovedá obsahuje jednotný layout aplikácie. Prípona .erb súborov umožňuje použitie ruby kódu. V tomto adresári sa stretneme tiež so systémom vkladania čiastkových elementov stránky príkazom partial. Adresár helpers umožňuje definovať funkcie použiteľné v erb súboroch, ktoré sprehľadňujú štruktúru kódu.
		\item[Controller] je zodpovedný za obsluhu požiadavku a vyprodukovanie odpovedi. Jeho úloha zvyčajne pozostáva z prijatia požiadavky, získania alebo uloženia dát do databázy, definovania premenných pre zobrazenia potrebného view súboru. Controller poskytuje prístup k request a response objektom a definuje premennú flash, ktorá môže obsahovať správu o úspechu alebo chybe danej akcie.
	\end{description}
	
	Konfigurácia projektu je rozdelená do troch súborov podľa aktuálneho prostredia - test, development a production. Rovnako je možné podľa prostredia definovať rôzne databázy v súbore database.yml.
	V konfigurácii framework nájdeme tiež súbory initializers, ktoré sú spustené pri štarte projektu a inicializujú jednotlivé komponenty a súbor routes obsahujúci smerovanie prichádzajúcich požiadavkov na jednotlivé controllery. Závislosti a použité komponenty sú definované v súbore Gemfile.
	
	\subsection{Sinatra}
	Sinatra je jazyk použiteľný pre rýchly vývoj jednoduchých webových aplikácií postavených na ruby. Framework zapuzdruje jednoduchý webový server. Umožňuje definovať takzvané route, predstavujúce url, ktoré aplikácia rozoznáva. Obsahom bloku definujúceho route je telo metódy, ktorá sa má vykonať pri zavolaní danej url. Tieto bloky akceptujú vstupné parametre, predstavujúce GET a POST parametre. Príklad takejto routy zobrazuje nasledujúci výpis.
	\begin{lstlisting}[label=lst:sinatra-sampel,caption=Príklad definovania GET route vo frameworku Sinatra]
	get '/hello/:name' do |n|
		"Ahoj #{n}!"
	end
	\end{lstlisting}
	V tejto aplikácii bola Sinatra použitá pri demonštrácii jedného z usecase - webu kontaktov. Celá logika aplikácie pozostáva vďaka použitiu tohoto frameworku z 32 riadkov.

	\subsection{Bootstrap}
	Bootstrap framework je najpopulárnejším HTML a CSS frameworkom pre vývoj webových projektov. Umožňuje rýchly a jednoduchý vývoj front-end aplikácie. Je dostupný vo forme css súborov ako aj sass pre jednoduché použitie v rails projektoch. Distribuovaný je pod MIT licenciou.
	
	Bootstrap poskytuje triedy upravujúce zobrazovanie základných HTML komponentov. Aplikovaný je pomocou premennej class na jednotlivých komponentoch. Distribúcia tiež zahŕňa sadu ikon, písem a javascript, ktorý je zodpovedný napríklad za zobrazenie vyskakovacích okien alebo pomocných textov. 
	
	Jednou z jeho najznámejších súčastí je mriežkový systém rozloženia stránky. Pred jeho použitím je potrebné pridať elementom, ktoré obaľujú stránku triedu container.	Mriežkový systém poskytuje responzívne rozhranie pozostávajúce z riadka obsahujúceho 12 stĺpcov, ktoré sa prispôsobujú podľa veľkosti obrazovky cieľového zariadenia. Tento systém umožňuje stránku rozdeliť na sekcie, ktoré je možné odsadiť, zarovnať, alebo rovnomerne rozložiť podľa potreby. Každý stĺpec funguje ako samostatná jednotka, ktorá môže obsahovať nový riadok s 12 stĺpcami. Stĺpce je možné spájať so skupín. Príklad použitia takéhoto rozloženia je vo výpise \ref{lst:bootstrap-grid}.
	\begin{lstlisting}[label=lst:bootstrap-grid,caption=Príklad použitia Bootstrap grid systému]
	<div class="row">
		<div class="col-md-8">.col-md-8</div>
		<div class="col-md-4">.col-md-4</div>
	</div>
	<div class="row">
		<div class="col-md-4">.col-md-4</div>
		<div class="col-md-4">.col-md-4</div>
		<div class="col-md-4">.col-md-4</div>
	</div>
	\end{lstlisting}
	
	Rozloženie stránky vytvorenej týmto kódom zobrazuje obrázok \ref{image_bootstrap-grid}.
		
	\myFigure{bootstrap-grid}{Ukážka Bootstrap grid systému}{Bootstrap grid systém}
			
	V tejto aplikácii bol bootstrap použitý pri vytváraní front-end administračného rozhrania ThesisWeb a usecase webu klientov.
	