\chapter{Inštalácia \& Použitie}
	Táto kapitola sa bude zaoberať názorným postupom, ktorý je potrebný k spusteniu aplikácie. Postup zahŕňa kroky od prípravy systému, inštaláciu aplikácie až po popis administračného rozhrania a názorné príklady použitia.
	
	\myTable{
		\begin{tabular}{ | l | r | r | }
			\hline
			Program & Verzia & Zdroj	\\
			\hline
			Windows	8	&	Professional N 64-bit	&	Microsoft DreamSpark	\\ \hline
			Java JDK	&	1.8.0\_31	&	www.oracle.com	\\ \hline
			Maven	&	3.2.3	&	maven.apache.org	\\ \hline
			Cassandra	&	2.1.2	&	cassandra.apache.org	\\ \hline
			RailsInstaller	&	3.1.0	&	railsinstaller.org	\\ \hline
			Node.js	&	0.10.31	&	nodejs.org	\\ \hline
		\end{tabular}
	}{Verzie programov použitých pri inštalácii}{table:program-verzia}
		
\section{Príprava systému}
	Aplikácia je k dispozícii vo forme zdrojových kódov. Pred použitím ju preto musíme skompilovať, nastaviť databázu a premenné prostredia a stiahnuť závislé knižnice. Táto sekcia sa zaoberá prípravou systému a prostredia ku spusteniu aplikácie.

\subsubsection{Operačný systém}
	Ako operačný systém bol použitý Windows 8 Professional N 64-bit. Táto verzia je pre študentov FIS dostupná prostredníctvom projektu DreamSpark - software spoločnosti Microsoft licencovaného pre akademické inštitúcie. Windows 8 bol nainštalovaný s predvolenými nastaveniami. Vzhľadom na minimálne požiadavky databázového systému Cassandra 2GB RAM by bolo vhodné použiť systém s minimálne 4GB RAM.

\subsubsection{Java}	% http://download.oracle.com/otn-pub/java/jdk/8u31-b13/jdk-8u31-windows-x64.exe
	Ako programovací jazyk serverovej časti aplikácie bola použitá java. Demonštračná kompilácia využíva aktuálnu verziu jdk1.8.0\_31. Je samozrejme možné použitie iných verzií, avšak knižnica esper správne pracuje len s niektorými z nich. Konkrétne nastáva problém pri pridaní schémy definovanej XML dokumentom do esper konfigurácie, kde vzniká výnimka:
	\begin{lstlisting}[label=lst:exjava,caption=Výnimka pri definovaní XML schémy v niektorých verziách javy]
	com.espertech.esper.client.ConfigurationException: Failed to read schema via URL 'null'
	\end{lstlisting}
	
	Testované verzie javy a ich kompatibilita je v nasledujúcej tabuľke:
	\myTable{
	\begin{tabular}{ | l | c | }
		\hline
		Verzia	&	Stav	\\
		\hline
		jdk1.7.0\_25	&	kompatibilná	\\ \hline
		jdk1.7.0\_71	&	nekompatibilná	\\ \hline
		jdk1.7.0\_72	&	kompatibilná	\\ \hline
		jdk1.8.0\_25	&	nekompatibilná	\\ \hline
		jdk1.8.0\_31	&	kompatibilná	\\ \hline
	\end{tabular}
	}{Kompatibilita verzií javy s esper xml schémami}{table:java-kompatibilita}
	
	Java bola nainštalovaná použitím predvolených nastavení, súčasťou ktorých je aj inštalácia JRE. Po nainštalovaní definujeme premennú prostredia JAVA\_HOME nasledovne:
	\begin{lstlisting}
	JAVA_HOME - C:\Program Files\Java\jdk1.8.0_31
	\end{lstlisting}
	Toto nastavenie je vyžadované nástrojom Maven, ktorý popisuje nasledujúca sekcia.
	
\subsubsection{Maven}	% http://maven.apache.org/
	Na zostavenie projektu bol použitý Maven vo verzii 3.2.3. Inštalácia pozostáva zo stiahnutia zip archívu (binárnej distribúcie) zo stránky projektu a rozbalenia do cieľového adresára. Pre jednoduchšie použitie je vhodné rozšíriť PATH o cestu k binárnym súborom nástroja, v tomto prípade:
	\begin{lstlisting}
	PATH = %PATH%;C:\Apache\apache-maven-3.2.3\bin
	\end{lstlisting}

\subsubsection{Cassandra}	% http://cassandra.apache.org/
	Cassandra je okrem binárnej verzie distribuovaná aj ako spustiteľný MSI inštalátor vďaka DataStax komunite. Tento spôsob inštalácie je pohodlnejší, avšak aktuálna verzia 2.1.2 pre operačný systém Windows po inštalácii nepracovala správne. Prejavili sa problémy ako chyby pri spustení spôsobené chybnou verziou knižnice jamm použitej pri kompilácii či poruchy Windows agenta.
		
	Pre projekt som kvôli týmto dôvodom použil binárnu distribúciu. Inštalácia tejto verzie je veľmi podobná inštalácii Mavenu, preberanému v predošlej sekcii. Na stránkach projektu stiahneme archív a rozbalíme ho do cieľového adresára. Tiež rozšírime systémovú premennú PATH o cestu k spúšťacím súborom takto:
	\begin{lstlisting}
	PATH = %PATH%;C:\Apache\apache-cassandra-2.1.2\bin
	\end{lstlisting}
	
	%TODO:CQL
	Zo stránok projektu je CQL

\subsubsection{RubyOnRails}	% http://railsinstaller.org/ 3.1.0
	Inštalácia frameworku RubyOnRails na windows platforme ja náročnejšia, pretože obsahuje viacero komponent, preto použijeme program RailsInstaller. Ten v použitej verzii obsahuje nasledujúce komponenty použité v projekte:
	\begin{itemize}
		\item Ruby 2.1.5 - interpretátor jazyka Ruby
		\item Rails 4.1 - webový framework pre administračné rozhranie
		\item Bundler - manažér závislostí projektu, funkciou podobný nástroju Maven
		\item Git - verzovací nástroj použitý prestiahnutie projektu z verejného repozitára
		\item DevKit - nástroj pre zostrojenie (build) natívnych C/C++ rozšírení na Windowse
	\end{itemize}

	Pri inštalácii použijeme predvolené nastavenia. Pri prvom použití nástroja bundler však narazíme na problém s certifikátmi:
	\begin{lstlisting}
	Error:SSL\_connect returned=1 errno=0 state=SSLv3 read server certificate B: certificate verify failed.
	\end{lstlisting}
	Riešenie pozostáva zo stiahnutia súboru obsahujúceho certifikáty certifikačných autorít. Ten následne sprístupníme pre ruby definovaním premennej prostredia \cite{web:certificate-fix}.
	\begin{lstlisting}
	ruby "%USERPROFILE%\Desktop\win_fetch_cacerts.rb"
	SSL_CERT_FILE = C:\RailsInstaller\cacert.pem
	\end{lstlisting}
		
	Problém s certifikátmi sa týmto vyriešil. Pri vytvorení a spustení prvej aplikácie ale narazíme na ďalšu výnimku:
	\begin{lstlisting}
	ExecJS::ProgramError - TypeError: Object doesn't support this property or method
	\end{lstlisting}
	Táto je spôsobená nekompatibilitou predvoleného windows javascript runtime a rails prostredia pri spracovaní assets (css a js) súborov. Jedným z riešení tohoto problému je inštalácia platformy Node.js. Pre potreby projektu postačuje predvolená inštalácia.
	
	Týmto krokmi sme pripravili prostredie pre spustenie aplikácie. Podrobný postup pre spustenie je popísaný v nasledujúcej sekcii.

\section{Spustenie aplikácie}
	Pred samotným spustením aplikácie je potrebné ju stiahnuť a skompilovať. Ak bol dodržaný postup inštalácie popísaný v predchádzajúcej sekcii tak sú všetky potrebné nástroje k dispozícii. Pokračujeme teda stiahnutím zdrojových kódov kompletného projektu z git repozitára. V príkazovom riadku:
	\begin{lstlisting}
	git clone https://github.com/kravciak/thesis-diploma-code.git
	\end{lstlisting}
	Kompletný projekt obsahuje 4 samostatné komponenty. Každá z nich sa spúšťa samostatne, avšak všetky sú závislé na Api, a pred použitím je nutné nakonfigurovať schémy a statementy - na čo slúži komponenta Web. Postup spustenia je preto nasledovný:
	
\subsubsection{Api}
	Predtým než je spustená Api je nutné spustiť cassandra databázu. Tá je spustiteľná z bin priečinka inštalácie súborom cassandra.bat. Pre spustenie vyžaduje administrátorské oprávnenia. Cassandru je tiež možné nainštalovať ako service pridaním parametra "install". Pri správnom spustení sa na konzole zobrazí text:
	\begin{lstlisting}
	Starting listening for CQL clients on localhost/127.0.0.1:9042...
	Binding thrift service to localhost/127.0.0.1:9160
	Listening for thrift clients...
	\end{lstlisting}
	
	Api nájdeme v priečinku ThesisApi. Je závislá na viacerých knižniciach, ktoré sú automaticky stiahnuté nástrojom Maven pri zostavovaní. Po úspešnom zostavení aplikáciu spustíme.
	\begin{lstlisting}
	mvn install
	mvn exec:java
	\end{lstlisting}
	Pri úspešnom spustení sa Api napojí na cassandru, spustí webový server apache a načíta konfiguráciu z databázy.
	
\subsubsection{Web}
	Administračné rozhranie sa nachádza v priečinku ThesisWeb. Podobne ako Api aj Web je závislý na viacerých gemoch. Tie sú stiahnuteľné pomocou bundleru, ktorého inštalácia je popísaná v predchádzajúcej sekcii. Po stiahnutí závislostí môžeme web spustiť.
	\begin{lstlisting}
	bundle install
	rails s
	\end{lstlisting}
	Po spustení je webové rozhranie dostupné na adrese \url{http://localhost:3000/}.
		
\subsubsection{Twitter Stream}
	Twitter stream závisí na dvoch gemoch, builder a twitter gem. Twitter gem slúži na prijatie vzorky tweetov a builder na transformáciu tweetu do formy xml. Obe nainštalujeme pomocou príkazu gem install, ktorý je dostupný ako súčasť ruby.
	\begin{lstlisting}
	gem install twitter builder
	ruby generator.rb
	\end{lstlisting}
	Keďže twitter stream sa pred spustením preposielania tweetov pripojí na Api (a ak je Api offline vyhlási výnimku) je nutné ich spustiť v tomto poradí.
	
\subsubsection{Contacts Web}
	Závislosti contacts webu nainštalujeme rovnako ako v prípade twitter streamu.		
	\begin{lstlisting}
	gem install sinatra faraday gyoku json
	ruby client.rb
	\end{lstlisting}
	Pre používanie je potrebné mať spustené Api.

\section{Použitie}
popis GUI..
názorné priklady
rozpisat sa o usecase