\chapter{Inštalácia \& Použitie}
	Táto kapitola sa bude zaoberať názorným postupom, ktorý je potrebný k spusteniu aplikácie. Postup zahŕňa kroky od prípravy systému, inštaláciu aplikácie až po popis administračného rozhrania a názorné príklady použitia.
	
	\myTable{
		\begin{tabular}{ | l | r | r | }
			\hline
			Program & Verzia & Zdroj	\\
			\hline
			Windows	8	&	Professional N 64-bit	&	Microsoft DreamSpark	\\ \hline
			Java JDK	&	1.8.0\_31	&	www.oracle.com	\\ \hline
			Maven	&	3.2.3	&	maven.apache.org	\\ \hline
			Cassandra	&	2.1.2	&	cassandra.apache.org	\\ \hline
			RailsInstaller	&	3.1.0	&	railsinstaller.org	\\ \hline
			Node.js	&	0.10.31	&	nodejs.org	\\ \hline
		\end{tabular}
	}{Verzie programov použitých pri inštalácii}{table:program-verzia}
		
\section{Príprava systému}
	Aplikácia je k dispozícii vo forme zdrojových kódov. Pred použitím ju preto musíme skompilovať, nastaviť databázu a premenné prostredia a stiahnuť závislé knižnice. Táto sekcia sa zaoberá prípravou systému a prostredia ku spusteniu aplikácie.

\subsubsection{Operačný systém}
	Ako operačný systém bol použitý Windows 8 Professional N 64-bit. Táto verzia je pre študentov FIS dostupná prostredníctvom projektu DreamSpark - software spoločnosti Microsoft licencovaného pre akademické inštitúcie. Windows 8 bol nainštalovaný s predvolenými nastaveniami. Vzhľadom na minimálne požiadavky databázového systému Cassandra 2GB RAM by bolo vhodné použiť systém s minimálne 4GB RAM.

\subsubsection{Java}	% http://download.oracle.com/otn-pub/java/jdk/8u31-b13/jdk-8u31-windows-x64.exe
	Ako programovací jazyk serverovej časti aplikácie bola použitá java. Demonštračná kompilácia využíva aktuálnu verziu jdk1.8.0\_31. Je samozrejme možné použitie iných verzií, avšak knižnica esper správne pracuje len s niektorými z nich. Konkrétne nastáva problém pri pridaní schémy definovanej XML dokumentom do esper konfigurácie, kde vzniká výnimka:
	\begin{lstlisting}[label=lst:exjava,caption=Výnimka pri definovaní XML schémy v niektorých verziách javy]
	com.espertech.esper.client.ConfigurationException: Failed to read schema via URL 'null'
	\end{lstlisting}
	
	Testované verzie javy a ich kompatibilita je v nasledujúcej tabuľke:
	\myTable{
	\begin{tabular}{ | l | c | }
		\hline
		Verzia	&	Stav	\\
		\hline
		jdk1.7.0\_25	&	kompatibilná	\\ \hline
		jdk1.7.0\_71	&	nekompatibilná	\\ \hline
		jdk1.7.0\_72	&	kompatibilná	\\ \hline
		jdk1.8.0\_25	&	nekompatibilná	\\ \hline
		jdk1.8.0\_31	&	kompatibilná	\\ \hline
	\end{tabular}
	}{Kompatibilita verzií javy s esper xml schémami}{table:java-kompatibilita}
	
	Java bola nainštalovaná použitím predvolených nastavení, súčasťou ktorých je aj inštalácia JRE. Po nainštalovaní definujeme premennú prostredia JAVA\_HOME nasledovne:
	\begin{lstlisting}
	JAVA_HOME - C:\Program Files\Java\jdk1.8.0_31
	\end{lstlisting}
	Toto nastavenie je vyžadované nástrojom Maven, ktorý popisuje nasledujúca sekcia.
	
\subsubsection{Maven}	% http://maven.apache.org/
	Na zostavenie projektu bol použitý Maven vo verzii 3.2.3. Inštalácia pozostáva zo stiahnutia zip archívu (binárnej distribúcie) zo stránky projektu a rozbalenia do cieľového adresára. Pre jednoduchšie použitie je vhodné rozšíriť PATH o cestu k binárnym súborom nástroja, v tomto prípade:
	\begin{lstlisting}
	PATH = %PATH%;C:\Apache\apache-maven-3.2.3\bin
	\end{lstlisting}

\subsubsection{Cassandra}	% http://cassandra.apache.org/
	Cassandra je okrem binárnej verzie distribuovaná aj ako spustiteľný MSI inštalátor vďaka DataStax komunite. Tento spôsob inštalácie je pohodlnejší, avšak aktuálna verzia 2.1.2 pre operačný systém Windows po inštalácii nepracovala správne. Prejavili sa problémy ako chyby pri spustení spôsobené chybnou verziou knižnice jamm použitej pri kompilácii či poruchy Windows agenta.
		
	Pre projekt som kvôli týmto dôvodom použil binárnu distribúciu. Inštalácia tejto verzie je veľmi podobná inštalácii Mavenu, preberanému v predošlej sekcii. Na stránkach projektu stiahneme archív a rozbalíme ho do cieľového adresára. Tiež rozšírime systémovú premennú PATH o cestu k spúšťacím súborom takto:
	\begin{lstlisting}
	PATH = %PATH%;C:\Apache\apache-cassandra-2.1.2\bin
	\end{lstlisting}
	
	%TODO:CQL
	Zo stránok projektu je CQL

\subsubsection{RubyOnRails}	% http://railsinstaller.org/ 3.1.0
	Inštalácia frameworku RubyOnRails na windows platforme ja náročnejšia, pretože obsahuje viacero komponent, preto použijeme program RailsInstaller. Ten v použitej verzii obsahuje nasledujúce komponenty použité v projekte:
	\begin{itemize}
		\item Ruby 2.1.5 - interpretátor jazyka Ruby
		\item Rails 4.1 - webový framework pre administračné rozhranie
		\item Bundler - manažér závislostí projektu, funkciou podobný nástroju Maven
		\item Git - verzovací nástroj použitý prestiahnutie projektu z verejného repozitára
		\item DevKit - nástroj pre zostrojenie (build) natívnych C/C++ rozšírení na Windowse
	\end{itemize}

	Pri inštalácii použijeme predvolené nastavenia. Pri prvom použití nástroja bundler však narazíme na problém s certifikátmi:
	\begin{lstlisting}
	Error:SSL\_connect returned=1 errno=0 state=SSLv3 read server certificate B: certificate verify failed.
	\end{lstlisting}
	Riešenie pozostáva zo stiahnutia súboru obsahujúceho certifikáty certifikačných autorít. Ten následne sprístupníme pre ruby definovaním premennej prostredia \cite{web:certificate-fix}.
	\begin{lstlisting}
	ruby "%USERPROFILE%\Desktop\win_fetch_cacerts.rb"
	SSL_CERT_FILE = C:\RailsInstaller\cacert.pem
	\end{lstlisting}
		
	Problém s certifikátmi sa týmto vyriešil. Pri vytvorení a spustení prvej aplikácie ale narazíme na ďalšu výnimku:
	\begin{lstlisting}
	ExecJS::ProgramError - TypeError: Object doesn't support this property or method
	\end{lstlisting}
	Táto je spôsobená nekompatibilitou predvoleného windows javascript runtime a rails prostredia pri spracovaní assets (css a js) súborov. Jedným z riešení tohoto problému je inštalácia platformy Node.js. Pre potreby projektu postačuje predvolená inštalácia.
	
	Týmto krokmi sme pripravili prostredie pre spustenie aplikácie. Podrobný postup pre spustenie je popísaný v nasledujúcej sekcii.

\section{Spustenie aplikácie}
	Pred samotným spustením aplikácie je potrebné ju stiahnuť a skompilovať. Ak bol dodržaný postup inštalácie popísaný v predchádzajúcej sekcii tak sú všetky potrebné nástroje k dispozícii. Pokračujeme teda stiahnutím zdrojových kódov kompletného projektu z git repozitára. V príkazovom riadku:
	\begin{lstlisting}
	git clone https://github.com/kravciak/thesis-diploma-code.git
	\end{lstlisting}
	Kompletný projekt obsahuje 4 samostatné komponenty. Každá z nich sa spúšťa samostatne, avšak všetky sú závislé na Api, a pred použitím je nutné nakonfigurovať schémy a príkazy - na čo slúži komponenta Web. Postup spustenia je preto nasledovný:
	
\subsubsection{Api}
	Predtým než je spustená Api je nutné spustiť cassandra databázu. Tá je spustiteľná z bin priečinka inštalácie súborom cassandra.bat. Pre spustenie vyžaduje administrátorské oprávnenia. Cassandru je tiež možné nainštalovať ako service pridaním parametra "install". Pri správnom spustení sa na konzole zobrazí text:
	\begin{lstlisting}
	Starting listening for CQL clients on localhost/127.0.0.1:9042...
	Binding thrift service to localhost/127.0.0.1:9160
	Listening for thrift clients...
	\end{lstlisting}
	
	Api nájdeme v priečinku ThesisApi. Je závislá na viacerých knižniciach, ktoré sú automaticky stiahnuté nástrojom Maven pri zostavovaní. Po úspešnom zostavení aplikáciu spustíme.
	\begin{lstlisting}
	mvn install
	mvn exec:java
	\end{lstlisting}
	Pri úspešnom spustení sa Api napojí na cassandru, spustí webový server apache a načíta konfiguráciu z databázy.
	
\subsubsection{Web}
	Administračné rozhranie sa nachádza v priečinku ThesisWeb. Podobne ako Api aj Web je závislý na viacerých gemoch. Tie sú stiahnuteľné pomocou bundleru, ktorého inštalácia je popísaná v predchádzajúcej sekcii. Po stiahnutí závislostí môžeme web spustiť.
	\begin{lstlisting}
	bundle install
	rails s
	\end{lstlisting}
	Po spustení je webové rozhranie dostupné na adrese \url{http://localhost:3000/}.
		
\subsubsection{Twitter Stream}
	Twitter stream závisí na dvoch gemoch, builder a twitter gem. Twitter gem slúži na prijatie vzorky tweetov a builder na transformáciu tweetu do formy xml. Obe nainštalujeme pomocou príkazu gem install, ktorý je dostupný ako súčasť ruby.
	\begin{lstlisting}
	gem install twitter builder
	ruby generator.rb
	\end{lstlisting}
	Keďže twitter stream sa pred spustením preposielania tweetov pripojí na Api (a ak je Api offline vyhlási výnimku) je nutné ich spustiť v tomto poradí.
	
\subsubsection{Contacts Web}
	Závislosti contacts webu nainštalujeme rovnako ako v prípade twitter streamu.		
	\begin{lstlisting}
	gem install sinatra faraday gyoku json
	ruby client.rb
	\end{lstlisting}
	Pre používanie je potrebné mať spustené Api.

\section{Použitie}

	\subsection{ThesisApi}
	Serverová časť aplikácie je prístupná prostredníctvom restovej api počúvajúcej na porte 80. Jej použitie teda spočíva z zavolaní správnej URL so správnymi parametrami. Po spustení je aplikácia predvolene prístupná na adrese http://localhost:8080/. Vo výpise \ref{lst:rest-api} sú zobrazené príklady volaní, ktoré server akceptuje.

	\begin{lstlisting}[label=lst:rest-api,caption=Príklad volaní REST API]
	Výpis prvých 100 schém
	http://localhost:8080/schemas?offset=0&limit=100
	
	Výpis prvých 100 príkazov
	http://localhost:8080/statements?offset=0&limit=100
	
	Zobrazenie výsledkov
	http://localhost:8080/statements/53/results
	\end{lstlisting}
	
	Niektoré z volaní vracajú súčasne s dátami aj metadáta poskytujúce dodatočné informácie, napríklad o stránkovaní, schéme udalostí či počte výsledkov. Tieto informácie sú spracované na strane grafického rozhrania a zjednodušujú užívateľovi prácu so systémom.
	
	Každá metóda má definovaný spôsob prístupu, teda jedinečnú kombináciu URL a metódy volania (GET, POST, DELETE). Tabuľka \ref{table:rest-urls} poskytuje kompletný prehľad mapovania http metódy a url na funkciu, ktorá obsluhuje dané volanie.	Funkcie sú rozdelené do štyroch hlavných skupín podľa oblasti, ktorú obsluhujú. Takisto ich mapovanie zodpovedá konkrétnej oblasti.
	
	\myTable{
		\begin{tabular}{ | l | l | l | }
			\hline
			HTTP metóda & Funkcia & URL	\\
			\hline
			
			GET    & ping()          & /esper/ping                                           \\
			POST   & processStream() & /esper/events                                         \\
			&&\\
			GET    & get()           & /schemas/\{id\}                                       \\
			GET    & getAll()        & /schemas                                              \\
			POST   & create()        & /schemas                                              \\
			DELETE & delete()        & /schemas/\{id\}                                       \\
			&&\\
			GET    & control()       & /statements/\{id\}/control                            \\
			GET    & get()           & /statements/\{id\}                                    \\
			GET    & getAll()        & /statements                                           \\
			POST   & create()        & /statements                                           \\
			DELETE & delete()        & /statements/\{id\}                                    \\
			&&\\
			DELETE & deleteAll()   & /statements/\{statement\_id\}/results                 \\
			DELETE & deleteOne()   & /statements/\{statement\_id\}/results/\{uuid\}        \\
			GET    & exportOne()   & /statements/\{statement\_id\}/results/\{uuid\}/export \\
			GET    & exportAll()   & /statements/\{statement\_id\}/results/export          \\
			GET    & getOne()      & /statements/\{statement\_id\}/results/\{uuid\}        \\
			GET    & getAll()      & /statements/\{statement\_id\}/results                 \\
			GET    & count()       & /statements/\{statement\_id\}/results/count           \\
			GET    & replayAll()   & /statements/\{statement\_id\}/results/replay          \\
			GET    & replayOne()   & /statements/\{statement\_id\}/results/\{uuid\}/replay \\
			\hline
		\end{tabular}
	}{Mapovanie funkcií na REST URL}{table:rest-urls}
	
	Obrázok \ref{image_api-funkcie} predstavuje parametre funkcií REST rozhrania. Rovnako ako tabuľka \ref{table:rest-urls} je rozdelený do štyroch skupín predstavujúcich triedy obsahujúcu konkrétnu funkciu. Ich vzájomnou kombináciou sme schopní identifikovať všetky REST volania, ktoré umožňuje serverová časť aplikácia spracovať.

	\myFigure{api-funkcie}{Parametre funkcií REST rozhrania}{Parametre funkcií REST rozhrania}

	%TODO Formát prijímaných udalostí udalosti, spracovanie streamu umožňuje veľké súbory
	%TODO Streamovanie je realizované pomocou Faraday::UploadIO komponentu aplikácie.

	\subsection{ThesisWeb}
	Hlavná obrazovka zobrazuje tri oblasti esperu, ktoré je možné pomocou aplikácie obsluhovať - správu schém a príkazov a formulár pre odosielanie udalostí. Po kliknutí na jednu z nich sa zobrazí obrazovka s rozšíreným menu \ref{image_menu}, ktoré navyše obsahuje možnosť prechodu na zobrazenie výsledkov nájdených konkrétnym príkazom.
	\myFigure{gui-menu}{Menu aplikácie rozšírené o zobrazenie výsledkov}{Menu aplikácie}
	
	Pod týmto menu sa nachádza navigačný panel s informáciou o ceste k práve zobrazovanej stránke. Po jeho pravej strane sú ovládacie tlačidlá na pridávanie, mazanie, prípadne export elementov na stránke. Súčasťou navigačného panelu je na stránke schém a príkazov vyhľadávanie, ktoré umožňuje filtrovať pomocou mena schémy alebo príkazu.
	
	
	Pod navigačným menu sa nachádza samotný obsah stránky. Nasledujúci text popisuje každú zo štyroch hlavných sekcií.
	\begin{description}
		\item[Schémy:] Na hlavnej obrazovke je možné vidieť zoznam schém a pri niektorých z nich číslo. Toto číslo reprezentuje počet príkazov závislých na konkrétnej schéme. Pro rozkliknutí schémy sa zobrazí jej detail. Na ľavej strane sú zobrazené dodatočné informácie vrátane zoznamu príkazov závislých na danej schéme. Na pravej strane obrazovky je konkrétna XML schéma. Pomocou ovládacích tlačidiel je možné túto schému exportovať alebo zmazať v prípade že neobsahuje žiaden na nej závislý príkaz.

		Pri pridávaní je nutné vyplniť meno schémy, koreňový element umožňujúci esper engine jedinečne identifikovať schému (zhodný s koreňovým elementu definície schémy) a XML definíciu schémy.

		\item[Statementy:] Podobne ako v prípade schém nájdeme na hlavnej obrazovke zoznam príkazov a pri každom z nich počet udalostí, ktoré mu vyhovujú. Po rozkliknutí sa zobrazí detail príkazu. Na tejto stránke je zaujímavá z popisu príkazu na ľavej strane URL adresa pomocou ktorej sú výsledky dostupné priamo zo serverovej API a odkaz pre zobrazenie výsledkov.
		Na pravej strane je zobrazená schéma príkazu s menami polí ich dátovými typmi. Nad ňou je panel pre opätovné preposlanie uložených výsledkov do esper engine. Pri preposlaní je možné zvoliť si počiatočný a konečný dátum ohraničujúci ktoré výsledky sa majú preposlať a meno udalosti pod ktorým sa majú odoslať. Tlačidlo medzi poliami ohraničujúcimi časový úsek umožňuje spočítať vyhovujúce udalosti.
		Ovládací panel príkazu umožňuje spustiť, pozastaviť alebo zmazať daný príkaz. Pozastavené príkazy nevyhľadávajú nové udalosti
		\myFigure{gui-replay-form}{Formulár preposielania historických udalostí na esper engine}{Formulár preposielania historických udalostí}
		
		Pri vytváraní novej schémy je potrebné zadať jej meno, umožňujúce jedinečne identifikovať príkaz v esper engine. V prípade že už existuje príkaz s rovnakým menom je na jeho koniec automaticky pridané číslo vo forme "--n". Voliteľne je možné zadať TTL, ktoré špecifikuje ako dlho majú byť výsledky príkazu držané v databáze. Posledné textové pole obsahuje text vytváraného príkazu. Novo vytvorený príkaz je predvolene v spustenom stave.
		
		\item[Streamer:] Pre posielanie nových udalostí existuje viacero spôsobov, jedným z nich je odoslanie udalostí pomocou administračného rozhrania na tejto stránke. V hornej časti je URL, ktorú je možné použiť na odosielanie udalostí priamo na API serverovej časti. Štruktúra odosielanej udalosti je bližšie popísaná v použití serverovej časti aplikácie a esper dokumentácii.
		Pre odoslanie udalostí je možné použiť formulár alebo udalosti načítať zo súboru. Obe možnosti sa dajú použiť súčasne. Pri načítaní zo súboru sú odosielané udalosti streamované, čo umožňuje posielať veľké objemy dát.

		V ovládacích prvkoch stránky nájdeme tlačidlo pre odoslanie vzorky dát. Tieto sú načítané zo súboru a vyhovujú prednastaveným schémam a príkazom. Ich použitie uľahčuje pochopenie fungovania aplikácie.
		
		\item[Výsledky] Pre zobrazenie výsledkov musíme najprv vybrať príkaz, ktorého výsledky chceme vidieť. Na stránku výsledkov je možné sa dostať z detailu príkazu kliknutím na odkaz s počtom výsledkov alebo na poslednú položku z menu.
		Na základnej stránke je tabuľka so zoznamom výsledkov. Tieto je možné stránkovať, avšak z dôvodu použitia NoSQL databázy nie je možné preskočiť na konkrétnu stránku. Je to zapríčinené tým, že použitá databáza nepozná klauzulu OFFSET a k výsledkom je tak možné pristupovať len špecifikáciou ich primárneho kľúča, alebo vzťahu k inému primárnemu kľúču.
		Po rozkliknutí konkrétneho výsledku sa na ľavej strane zobrazí jeho detailný popis, v ktorom je možné vidieť napríklad príkaz ktorému výsledok patrí alebo čas za ktorý expiruje. Pod týmito metadátami je daný výsledok zobrazený v JSON formáte.
		
		Podobne ako na stránke príkazu, kde je možné preposlať všetky výsledky je možné preposlať jeden konkrétny výsledok pomocou panelu v pravej časti tejto stránky.

		Vďaka ovládacím prvkom je možné exportovať alebo zmazať všetky výsledky na hlavnej stránke, alebo konkrétny výsledok pri zobrazení jeho detailu. V navigačnom paneli je zobrazené aj UUID výsledku.
	\end{description}
	
	\myFigure{gui-breadcrumbs}{Menu aplikácie rozšírené o zobrazenie výsledkov}{Menu aplikácie}
	
	\myFigure{gui-schema}{Menu aplikácie rozšírené o zobrazenie výsledkov}{Menu aplikácie}
	