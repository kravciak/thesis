\chapter*{Záver}
\addcontentsline{toc}{chapter}{\protect\numberline{}Záver}

V úvode tejto práce boli stanovené tri ciele. Pre ich zdokumentovanie bolo nutné čitateľa najprv uviesť do problematiky spracovania komplexných udalostí a práce s nosql databázami, ktorými sa zaoberajú kapitoly \ref{chap:pojmy} a \ref{chap:technologie}.

Po oboznámení čitateľa s použitými technológiami nasleduje kapitola \ref{chap:implementacia}, v ktorej je popísaný návrh a implementácia konkrétneho riešenia. Práve v tejto časti sú realizované ciele diplomovej práce.

Prvým z nich bolo vytvorenie administračného rozhrania k Esper engine. Táto čast bola implementovaná pomocou frameworku Ruby On Rails. Administračné rozhranie pre prístup k Esper engine využíva serverovú časť, na ktorú sa napája pomocou restovej api. Toto oddelenie spĺňa podmienku druhého cieľa, a umožňuje vývoj ďalších aplikácií využívajúcich prístup k Esper engine. Serverová časť bola implementovaná v jave za pomoci spring frameworku.

Tretím cieľom práce bolo umožnenie preposielania historických udalostí. Táto časť zadania bola tiež splnená, keďže administračné rozhrania obsahuje v detaile príkazu a udalosti formulár, umožňujúci presmerovať uložené výsledky na vstup Esper engine. Rovnako ako všetky operácie je preposielanie udalostí možné realizovať aj bez administračného rozhrania a to volaním definovanej URL v restovej api.

Databázová časť aplikácie bola rozdelená na dve samostatné časti, jednu SQL databázu, ktorá slúži na uchovávanie konfigurácie serverovej časti aplikácie a druhú NoSQL databázu na ukladanie výsledkov Esperu. Administračné rozhranie nevyužíva žiadne úložisko dát, všetky údaje sú dynamicky získavané pomocou restovej api.

Keďže pri implementácii a inštalácia celkového riešenia som narazil na viacero problémov kompatibility či už s java knižnicami, certifikátmi alebo javascript engine a ich riešenie nepovažujem za triviálne je súčasťou práce aj kapitola zaoberajúca sa inštaláciou, spustením a použitím aplikácie.

Zdrojový kód aplikácie je voľne dostupný z úložiska github, čo umožňuje ľubovoľné úpravy a rozšírenia, napríklad v rámci ďalšej diplomovej práce. Možné rozšírenia vidím v pridaní správy premenných, vzorov, pomenovaných okien, tabuliek či vylepšení grafického rozhrania.

\emptydoublepage
