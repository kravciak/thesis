\chapter{Implementácia}
Táto kapitola popisuje implementačné detaily projektu. 

\section{Api}
	Esper je voľne dostupný pre dve vývojové platformy - javu a .net. Pre serverovú časť aplikácie som zvolil javu s použitím spring frameworku.
	\subsection{Konfiguračné súbory}
		application-config.xml
		database.properties
		esper.cfg.xml
		logback.xml
	\subsection{Databáza}
		Aplikácia využíva dve databázy, jednu na ukladanie konfiguračných dát a druhú na udalosti nájdené esperom. Je to tak kvôli predpokladu že esper bude produkovať veľké množstvo dát, preto je vhodné použitie databázy optimalizovanej pre zápis.
	
		\subsubsection{Databáza konfigurácie}
		Ako databázu pre ukladanie stavu esperu a jednotlivých konfiguračných položiek som použil \ac{HSQLDB}. Táto databáza je napísaná v jave a pre ukladanie tabuliek ponúka pamäťový aj diskový mód. Použil som ju, pretože je jednoduchá, nenáročná na systém a je možné ju stiahnuť ako jednu zo závislostí aplikácie pomocou mavenu. Databáza je jednoducho nahraditeľná iným riešením, keďže k nej je pristupované pomocou jdbc. Pre zmenu je potrebné upraviť konfiguračný súbor application-config.xml a zmeniť položku dataSource.
		
		Databáza obsahuje tabuľku schém a statementov, ktoré sú načítavané pri štarte Api. Jej štruktúru vidíme na nasledujúcom obrázku.
		%TODO model databazy
			
		\subsubsection{Databáza udalostí}
		Databáza udalostí slúži na ukladanie výsledkov vyhovujúcich niektorému z definovaných statementov. Ako konkrétne riešenie som použil databázu Cassandra. Cassandra je jedným z predstaviteľov NoSQL databáz. Jednou z jej výhod je možnosť rozšíriteľnosti v prípade veľkého objemu dát a optimalizácia pre zápis. Použitie tejto databázy je v rámci "proof of concept" princípu, kde by bolo jednoduchšie pracovať napríklad s HSQLDB ako v prípade konfigurácie, avšak kvôli možnosti generovania veľkého množstva udalostí esperom je použitá databáza na to vhodná.
	
	\subsection{Spring framework}
		Ako základ serverovej časti aplikácie som použil Spring framework. Jeho hlavnou úlohou je injektovanie závislostí vo väčšine tried, no využil som aj rozšírenia pre databázu, webový prístup a pribalený webový server tomcat.
		Spring Framework poskytuje v aplikácii podporu pre dependency injection, správu transakcií a prístup k dátam pomocou jdbc. Serverová časť aplikácie využíva nasledujúce knižnice:
		\begin{description}
			\item[spring-core:] Pomocou anotácie @Autowired sú v aplikácii riešené závislosti väčšiny komponent.
			\item[spring-jdbc:] Prístup do databázy je realizovaný pomocou spring triedy NamedParameterJdbcTemplate, ktorá oproti jdbc pridáva možnosť prístupu k vygenerovanému id nového záznamu.
			\item[spring-webmvc:] Prístup k dátam a ovládaniu serverovej časti aplikácie je umožnený pomocou restovej api.
			\item[spring-boot-starter-web:] Táto závislosť umožňuje použitie pribaleného tomcat serveru. Ten sa spustí pri štarte aplikácie a sprístupní restovú api.
		\end{description}

	\subsection{Esper engine}
		Esper je v aplikácii implementovaný triedou EsperManager, ktorá poskytuje prístup ku konfiguračným nástrojom a sprístupňuje handler prichádzajúcich udalostí. Pri spustení aplikácie načítava konfiguračné údaje z databázy a inicializuje esper. Najdôležitejšou úlohou tejto triedy je správa schém a statementov.
		\begin{description}
			\item[Schémy] sú reprezentované XML (org.w3c.dom.Node) dokumentom. Oproti POJO reprezentácii umožňuje tento formát vytváranie schém počas behu čo je z pohľadu užívateľa nutnosťou. V prípade potreby by bolo možné použiť udalosti reprezentované pomocou java.util.Map alebo objektovým poľom, avšak to vyžaduje definovanie formátu prenosu a následné spracovanie do požadovaného formátu klientom, čo nie je veľmi intuitívne. Použitie XML reprezentácie klientovi umožní komunikovať priamo s esperom.
			
			\item[Udalosti] ktoré engine spracováva musia byť v XML formáte. Toto obmedzenie je zapríčinené XML reprezentáciou schém. Ako zdroje udalostí týmto eliminujeme adaptéry CSV a HTTP z esperio knižnice. Na prijímanie XML udalostí slúži rest služba. Tá spracováva vstupný stream dvoma spôsobmi:
			\begin{itemize}
				\item Ako jednotlivé udalosti, kde koreňový element udalosti reprezentuje názov schémy reprezentujúcej udalosť.
				\item Ako stream udalostí, kde je koreňový element nazvaný "events" a obaľuje jednotlivé udalosti definované v predošlom bode. Koreňový element je v tomto prípade ignorovaný.
			\end{itemize}
			
			\item[Statementy] pridávané do esper engine môžu obsahovať len meno a epl výraz. Preto je ku každému statementu priradený aj užívateľský objekt - statementBean s dodatočnými informáciami:
			\begin{itemize}
				\item ID ktoré unikátne identifikuje statement v rámci celej aplikácie, nie len daného esper provideru
				\item TTL hovoriaci ako dlho má byť nájdená udalosť perzistentná.
				\item STATE udávajúci či je konkrétny statement spustený alebo zastavený
			\end{itemize}
				
			
			Pri pridávaní statementu do esper engine je kontrolovaná unikátnosť mena v rámci daného esper providera. V prípade že existuje statement s rovnakým menom je meno doplnené o "--N", kde N značí najbližšie voľné celé číslo. Takto upravený statement je následne uložený.
		\end{description}
		
		Aplikácia definuje jeden globálny listener, ktorý je priradený všetkým statementom. Ten v prípade výskytu udalosti vyhovujúcej niektorému z definovaných statementov uloží udalosť vo forme obsahu json objektu do databázy.
		
		Esper umožňuje export výsledkov pomocou JSONRenderer a XMLRenderer, avšak tieto triedy produkujú formátovaný výstup, čo je v serverovej časti aplikácie neželané. Preto je konverzia realizovaná upravenými verziami týchto tried, ktoré nepridávajú formátovacie znaky a produkujú validné XML a JDON dokumenty. Predvolene je použitý upravený JSONRenderer.
		
		Pri ukladaní týchto týchto výsledkov do databázy je nastavený TTL, ktorý bol definovaný pri vytvorení statementu. Ak TTL pri vytvorení statementu nebolo definované použije sa predvolené nastavenie, kde sú výsledky persistentné až kým ich užívateľ manuálne neodstráni.

	\subsection{Maven}
		Zostavenie projektu je jednou z nevyhnutných súčastí tvorby java aplikácií. Na uľahčenie tohoto procesu je možné použiť viacero nástrojov, ktorých hlavnými predstaviteľmi sú maven, gradle a ant. Api komponent tejto aplikácie je zostavený pomocou nástroja maven. 
		
		Maven uľahčuje prácu vo viacerých oblastiach, a to \cite{web:maven-doc}:
		\begin{itemize}
			\item Uľahčenie prekladu aplikácie
			\item Poskytnutie jednotného riešenia pre zostavenie aplikácie
			\item Poskytnutie informácií o projekte
			\item Poskytnutie vzorov pre vývoj aplikácií
			\item Umožnenie transparentnej migrácie nových vlastností programu
		\end{itemize}
		Pre túto prácu je najdôležitejšia prvá oblasť. Vďaka mavenu nemusí distribúcia projektu obsahovať všetky knižnice závislostí, ani nemusia byť jednotlivo sťahované pri zostavovaní projektu. Všetky potrebné závislosti sú definované v súbore pom.xml a pri preklade automaticky stiahnuté z internetu. Tento súbor zároveň definuje vlastnosti projektu ako názov, verziu, verziu javy použitú pre zostavenie, spôsob generovania dokumentácie a iné detaily.
		
		Maven tiež umožňuje vytvárať v pom.xml súboroch závislosti a odkazovať sa na externé konfigurácie. Táto funkcionalita je v projekte využitá pri spring závislostiach, kde je verzia niektorých komponent definovaná v externom rodičovskom súbore. Rodičom spring závislostí projektu je spring-boot-starter-parent.
		
		%Väčšina funkcionality mavenu sa skladá z pluginov, k

\section{Web}
	Web časť projektu slúži ako administračné rozhranie. Je riešené formou webovej aplikácie, ktorej základ je framework Ruby On Rails. Ten umožňuje rýchle vytváranie stránok, kde prevažuje prístup "convention over configuration". Dôležitou súčasťou frameworku sú gemy, ktoré reprezentujú závislosti projektu. Jedným z najpodstatnejších v tomto projekte je gem "her", ktorý zabezpečuje komunikáciu s Api časťou projektu.
	
	Grafická stránka tohoto projektu sa opiera o framework Bootstrap. Ten v základnej konfigurácii poskytuje html komponenty s vylepšeným UI. Tiež opravuje niektoré chyby kompatibility pri zobrazovaní stránok v rôznych prehliadačoch.
	
	Web nevyužíva žiadne persistentné úložisko dát. Všetky realizované zmeny sú posielané na restovú api, ktorá zmeny spracuje a uloží. Zobrazované údaje sú tiež získavané zo vzdialeného zdroja.
		
\section{UseCase}
	
	\subsection{Twitter Stream}
	\subsection{Contacts Web}
		
- 3 sposoby vkladania dat - zo suboru / handleru / webova aplikacia
- moznost prehravania historickych dat (vysledokv) ako vstup