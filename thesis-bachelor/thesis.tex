\documentclass[12pt, a4paper]{book}
%\documentclass[12pt, a4paper, oneside]{book} %isis

% ----- Nastavenie pisma -----
\usepackage[slovak]{babel}
\usepackage[IL2]{fontenc}
\usepackage[utf8]{inputenc}

% ----- Vkladanie obrazkov -----
\usepackage{color, graphicx}
\graphicspath{{./images/}}

% ----- URL -----
\usepackage{url}
\usepackage[unicode]{hyperref} %odkazy vnutri dokumentu

% ----- Zdrojové kódy -----
\usepackage{listings}
% info - http://en.wikibooks.org/wiki/LaTeX/Packages/Listings
% comments - http://lenaherrmann.net/2010/05/20/javascript-syntax-highlighting-in-the-latex-listings-package
\lstset{
%  language=bash,                  % the language of the code
  basicstyle=\footnotesize,       % the size of the fonts that are used for the code
  numbers=none,                   % where to put the line-numbers
  numberstyle=\tiny\color{gray},  % the style that is used for the line-numbers
  stepnumber=2,                   % the step between two line-numbers. If it's 1, each line 
                                  % will be numbered
  numbersep=5pt,                  % how far the line-numbers are from the code
  backgroundcolor=\color{white},  % choose the background color. You must add \usepackage{color}
  showspaces=false,               % show spaces adding particular underscores
  showstringspaces=false,         % underline spaces within strings
  showtabs=false,                 % show tabs within strings adding particular underscores
  frame=lines,                    % adds a frame around the code
  rulecolor=\color{black},        % if not set, the frame-color may be changed on line-breaks within not-black text (e.g. commens (green here))
  tabsize=2,                      % sets default tabsize to 2 spaces
  captionpos=b,                   % sets the caption-position to bottom
  breaklines=true,                % sets automatic line breaking
  breakatwhitespace=false,        % sets if automatic breaks should only happen at whitespace
  title=\lstname,                 % show the filename of files included with \lstinputlisting;
                                  % also try caption instead of title
  keywordstyle=\color{blue},      % keyword style
  stringstyle=\color{mauve},      % string literal style
  escapeinside={\%*}{*)},         % if you want to add a comment within your code
  morekeywords={*,...},            % if you want to add more keywords to the set
	extendedchars=true,
	literate={á}{{\'a}}1						%http://www.latexsearch.com/sandbox.do
	{é}{{\'e}}1
	{í}{{\'i}}1
	{ó}{{\'o}}1
	{ú}{{\'u}}1
	{ľ}{{\v{l}}}1
	{š}{{\v{s}}}1
	{ť}{{\v{t}}}1
%	comment=[l]{\#},
%  commentstyle=\color{blue}
}
\renewcommand\lstlistingname{Výpis}
	
% ----- Zoznam skratiek -----
\usepackage[nolist,footnote]{acronym}

% ----- Okraje -----
\usepackage[top=2.5cm,left=2cm,right=3cm,bottom=2.5cm]{geometry}
%\usepackage[top=2.5cm,left=3cm,right=2cm,bottom=2.5cm]{geometry} %isis

\linespread{1.5}

%-----------------------------------------------------------------------------------------------------------------------------------------
%\textwidth100mm
%\marginparsep10mm
%\marginparwidth30mm


%\newlength{\fullwidth} % Width of text plus margin notes
%\setlength{\fullwidth}{\textwidth}
%\addtolength{\fullwidth}{\marginparsep}
%\addtolength{\fullwidth}{\marginparwidth}

%----------------------------------------------------------------------------------
% \myBigFigure	[ LABEL_PREFIX (optional) ]
%				{ FILENAME (without extension) }
%				{ CAPTION TEXT }
%				{ SHORT VERSION OF CAPTION TEXT }
%
%picture using full width of the page
\newcommand{\myBigFigure}[4][image]
{
\begin{figure}[t!bp]
	\checkoddpage
	\ifcpoddpage
		%nothing
	\else
		\hspace{-\marginparsep}\hspace{-\marginparwidth}
	\fi
	%use minipage to center the label beneath the figure
	\begin{minipage}{\fullwidth}
		\includegraphics[width= \fullwidth]{#2}
		\caption[#4]{#3}
		\label{#1_#2}
	\end{minipage}
\end{figure}
}


%----------------------------------------------------------------------------------
% \myFrameBigFigure	[ LABEL_PREFIX (optional) ]
%					{ FILENAME (without extension) }
%					{ CAPTION TEXT }
%					{ SHORT VERSION OF CAPTION TEXT }
%
%picture with frame using the full width of the page
\newcommand{\myFrameBigFigure}[4][image]
{
\begin{figure}[t!bp]
	\checkoddpage
	\ifcpoddpage
		%nothing
	\else
		\hspace{-\marginparsep}\hspace{-\marginparwidth}
	\fi
	%use minipage to center the label beneath the figure
	\begin{minipage}{\fullwidth}
	\frame{%
		\includegraphics[width= \fullwidth]{#2}%
		}
		\caption[#4]{#3}
		\label{#1_#2}
	\end{minipage}
\end{figure}
}

%----------------------------------------------------------------------------------
% \myHUGEFigure	[ LABEL_PREFIX (optional) ]
%				{ FILENAME (without extension) }
%				{ CAPTION TEXT }
%				{ SHORT VERSION OF CAPTION TEXT }
%
%landscape picture using the full width of the rotated page
\newcommand{\myHugeFigure}[4][image]
{
\begin{sidewaysfigure}[t!bp]
	
		\includegraphics[width= \textheight]{#2}
		\caption[#4]{#3}
		\label{#1_#2}
	
\end{sidewaysfigure}
}

%----------------------------------------------------------------------------------
% \myFigure	[ LABEL_PREFIX (optional) ]
%			{ FILENAME (without extension) }
%			{ CAPTION TEXT }
%			{ SHORT VERSION OF CAPTION TEXT }
%
%picture using the width of the text column
\newcommand{\myFigure}[4][image]%
{%
\begin{figure}[ht!bp]%
	\begin{center}%
		\includegraphics[width= \textwidth]{#2}%
		\caption[#4]{#3}
		\label{#1_#2}%
	\end{center}%
\end{figure}%
}%

%----------------------------------------------------------------------------------
% \myImgRef	[ LABEL_PREFIX (optional) ]
%			{ LABEL OF THE IMAGE }
%
%reference to an image
\newcommand{\myImgRef}[2][image]%
{%
	\ref{#1_#2}%
}%

%----------------------------------------------------------------------------------
% \myBigTable	{ YOUR TABULAR DEFINITION }
%			{ CAPTION TEXT }
%			{ TABLE_LABLE }
%
%table using the full width of the page
\newcommand{\myBigTable}[3]%
{%
\begin{table}[htdp]%
	\checkoddpage%
	\ifcpoddpage%
		%nothing
	\else%
		\hspace{-\marginparsep}\hspace{-\marginparwidth}%
	\fi%
	\begin{minipage}{\fullwidth}%
		\begin{center}%
			#1%
			\caption{#2}%
			\label{#3}%
		\end{center}%	
	\end{minipage}%
\end{table}%
}%

%----------------------------------------------------------------------------------
% \myTable	{ YOUR TABULAR DEFINITION }
%			{ CAPTION TEXT }
%			{ TABLE_LABLE }
%
%table using the width of the text column
\newcommand{\myTable}[3]%
{%
\begin{table}[htdp]%
	\begin{center}%
		#1%
		\caption{#2}%
		\label{#3}%
	\end{center}%	
\end{table}%
}%

%----------------------------------------------------------------------------------
% \myTxtRef	{ LABLE }
%
%references chapters or sections, outputs number and title, e.g., 5.3---"Yaddahyaddah"
\newcommand{\myTxtRef}[1]
{%
	\ref{#1}---``\nameref{#1}''%
}

%----------------------------------------------------------------------------------
% \myUnderscore
%
%typesets a 'nice' underscore for URLs
\newcommand{\myUnderscore}{$\underline{\hspace{0.5em}}$}

%----------------------------------------------------------------------------------
%\myTilde
%
%typesets a 'nice' tilde for URLs
\newcommand{\myTilde}{$\sim$}

%----------------------------------------------------------------------------------
% \myURL	{ TYPESET VERSION OF ANCHOR }
%			{ PRISTINE URL }
%			{ TYPESET VERSION OF URL }
%
%typesets a URL
%the typographically correct version appears as a footnote,
%the anchor appears in the text, the link points to the pristine URL
\newcommand{\myURL}[3]%
{%
	\textcolor{blue}{%
		\href{#2}{#1}%
	}%
	\footnote{#3}
}

%----------------------------------------------------------------------------------
% \mySimpleURL	{ TYPESET VERSION OF ANCHOR }
%				{ PRISTINE URL }
%
%typesets a URL
%the URL appears as a footnote,
%the anchor appears in the text, the link points to the URL
\newcommand{\mySimpleURL}[2]%
{%
	\textcolor{blue}{%
		\href{#2}{#1}%
	}%
	\footnote{#2}
}

%----------------------------------------------------------------------------------
% \myProjectURL	{ TYPESET VERSION OF ANCHOR }
%				{ PRISTINE URL INSIDE PROJECT DIRECTORY }
%				{ TYPESET VERSION OF URL INSIDE PROJECT DIRECTORY }
%
%typesets a URL to hci/public from where the contents of the WebServer folder from oliver can be accessed
%the typographically correct version appears as a footnote,
%the anchor appears in the text, the link points to the pristine URL
\newcommand{\myProjectURL}[3]%
{%
	\textcolor{blue}{%
		\href{http://hci.rwth-aachen.de/public/#2}{#1}%
	}%
	\footnote{http://hci.rwth-aachen.de/public/#3}%
}

%----------------------------------------------------------------------------------
% \mnote	{ MARGIN NOTE }
%
%puts a comment into the margin in small sans-serif font
\newcommand{\mnote}[1]{\marginpar{\raggedright\textsf{{\footnotesize{#1}}}}}

%----------------------------------------------------------------------------------
% \todo	{ TODO MARGIN NOTE }
%
%puts a 'todo' comment into the margin in red
\definecolor{red}{rgb}{1,0,0}
\newcommand{\todo}[1]{\mnote{\textcolor{red}{ToDo: #1}}}

%----------------------------------------------------------------------------------
% \chapterquote	{ QUOTATION }
%				{ SOURCE }
%
%outputs a quote with its source, can be used as an introduction to chapters
\newcommand{\chapterquote}[2]{
\begin{quotation}
    \begin{flushright}
	\noindent\emph{``{#1}''\\[1.5ex]---{#2}}
    \end{flushright}
\end{quotation}
}

%----------------------------------------------------------------------------------
% \myDefBox	{ TERM }
%			{ DEFINITION }
%
%outputs a margin note and a colored box (width of the text column) containing a term and its definition
\newcommand{\myDefBox}[2]
{%
	\setlength{\fboxrule}{1mm}%
	\fcolorbox{orange_med}{orange_light}%
	{%
		\parbox{\myDefBoxWidth}{{\bfseries\scshape#1:}\\#2}%
	}%
	\mnote{Definition:\\\emph{#1}}
}

%----------------------------------------------------------------------------------
% \myBigDefBox	{ TERM }
%				{ DEFINITION }
%
%outputs a colored box (width of the page) containing a term and its definition
\newcommand{\myBigDefBox}[2]
{%
	\begin{figure}[h!]
	\setlength{\fboxrule}{1mm}%
	\checkoddpage%
	\ifcpoddpage%
		%nothing
	\else%
		\hspace{-\marginparsep}\hspace{-\marginparwidth}%
	\fi%
	\fcolorbox{orange_med}{orange_light}%
	{%
		\parbox{\myBigDefBoxWidth}{{\bfseries\scshape#1:}\\#2}%
	}%
	\end{figure}
}

%----------------------------------------------------------------------------------
% \myDownloadURL	{ TYPESET DOWNLOAD NAME }
%					{ PRISTINE VERSION OF FILENAME }
%					{ TYPESET VERSION OF FILENAME }
%
%outputs a colored box containing a download link
\newcommand{\myDownloadURL}[3]{%
\checkoddpage%
	\ifcpoddpage%
		%nothing
	\else%
		\hspace{-\marginparsep}\hspace{-\marginparwidth}%
	\fi%
\setlength{\fboxrule}{1mm}%
\fcolorbox{green_med}{green_light}{%
\begin{minipage}{\myBigDefBoxWidth}%
\begin{center}%
\myProjectURL{#1}{folder/#2}{folder/#3}%
\end{center}%
\end{minipage}%
}%
}

%----------------------------------------------------------------------------------
% \emptydoublepage
%
% Clear double page without any header or footer at end of chapters
\newcommand{\emptydoublepage}{\clearpage\thispagestyle{empty}\cleardoublepage}

%----------------------------------------------------------------------------------
% \pagebreak	[ SOME STRANGE LATEX VALUE ]
%
%pagebreaks for the final print version (last resort weapon against wrong pagebreaks by LaTeX)
\newcommand{\PB}[1][3]
{%
	\pagebreak[#1]%
}

\begin{document}

% ----- Info o práci -----
\title{Řešení vysoké dostupnosti pro síťové filesystémy v~Linuxu}
\author{Martin Kra}
\date{Máj 9, 2012}

%--------------------------------------------------------------
\frontmatter

\maketitle \thispagestyle{empty} \emptydoublepage

\chapter*{Prehlásenie}
Prehlaem, že som bakalársku prácu spracoval samostatne, a že som uviedol všetky použité premene a literatúru, z ktorej som čerpal.

\emptydoublepage

\chapter*{Poďakovanie}
Chcem poďakovať všetkým, ktorí ma podporovali a pomáhali mi, najmä svojim rodičom a starým rodičom za trpezlivosť a podporu pri štúdiu a Anke za motiváciu. V neposlednom rade ďakujem vedúcemu mojej práce, Ing. Lubošovi Pavlíčkovi za pomoc pri tvorbe práce a Tobimu za cenné rady, ktoré mi dal.

\emptydoublepage

%\chapter*{Abstract\markboth{Abstract}{Abstract}}
%\addcontentsline{toc}{chapter}{\protect\numberline{}Abstract}
%\label{abstract}

\chapter*{Abstrakt}
Cieľom tejto práce je oboznámenie čitateľov s problematikou vysokej dostupnosti a poskytnutie základného prehľadu o voľne dostupných technológiách, pomocou ktorých sa dá dosiahnuť. Zameriavam sa predovšetkým na riešenie problémov vyplývajúcich z hardwarových chýb systémov. Dokument bude možné použiť na získanie základných vedomostí potrebných pre konfiguráciu vlastného riešenia.

V praktickej časti realizujem vysoko dostupné klastrové riešenie pomocou voľne dostupných technológií a zároveň porovnám výkonnosť rôznych súborových systémov na tejto platforme. Práca môže poslúžiť aj ako návod ako pomocou vlastnej konfigurácie otestovať a čo najvhodnejšie zvoliť súborový systém pre použitie v produkčnom prostredí.

\section*{Kľúčové slová}
vysoká dostupnosť, pacemaker, drbd, linux, súborový systém \emptydoublepage

\chapter*{Abstract}
The goal of this paper is to inform readers about high availability problematics and provide basic overview of free technologies you can use to achieve high availability. I focus mainly on solution of problems caused by hardware failures. Document can by used to gain basic knowledge needed for configuration of your own solution.

In practical part I will try to realize high available cluster using free technologies and compare performance of different filesystems on this platform. Document will provide instructions for choosing and testing appropriate filesystem for specific configuration that can be used in production environment.

\section*{Keywords}
high availability, pacemaker, heartbeat, drbd, linux, filesystem \emptydoublepage

\tableofcontents \emptydoublepage

%--------------------------------------------------------------
\mainmatter

\chapter*{Úvod}
\addcontentsline{toc}{chapter}{\protect\numberline{}Úvod}

Rýchly nárast množstva dát produkovaných užívateľmi a aplikáciami prináša problémy s ich spracovaním a vyhodnocovaním. Pri analýze statických dát z bežnej databázy narážame na obmedzenia spôsobené dávkovým spracovaním dotazov. Technológia spracovania \ac{CEP} prináša možnosť spracovania prúdov informácií a ich komplexných závislostí v reálnom čase. Definuje pojem udalosť, ktorá predstavuje jednotku informácie s ktorou systém pracuje. Bližšie sa touto technológiou zaoberá kapitola \ref{chap:pojmy}.

Prvotným cieľom tejto diplomovej práce je vytvorenie grafického nástroja, ktorý umožní vykonávanie základných operácií potrebných pre prácu s esper engine. Tými sú správa schém a príkazov, odosielanie udalostí a zobrazenie nájdených výsledkov. Použitím tohoto nástroja odpadá užívateľovi potreba znalosti programovacích platforiem java a net, pre ktoré je Esper oficiálne dostupný.

Druhotným cieľom je rozdelenie väzby medzi administračným rozhraním a esper engine a umožnenie prístupu k základným operáciám definovaným v prvom cieli pomocou restovej api. Toto rozdelenie umožní vytváranie ďalších aplikácií, ktoré budú využívať esper engine taktiež bez nutnosti znalosti programovania v jave alebo NET platforme.

So splnením druhého cieľa je tiež viazaný koncept deklaratívnej tvorby webových aplikácií. Bežne sa stretneme s imperatívnym programovacím prístupom, kde je funkcionalita riešená presne definovanými algoritmami, teda postupom ako danú úlohu vyriešiť. Deklaratívne programovanie naproti tomu hovorí čo sa má vykonať, nie ako sa to má vykonať.
Využitie restovej api pre komunikáciu s esper engine umožňuje vytvárať jednoduché webové aplikácie, ktorých dátová časť je riešená deklaratívne. Pred použitím je nutné definovať schému dát napríklad pomocou administračného rozhrania a nastaviť príkazy, ktoré budú zachytávať a filtrovať prichádzajúce dáta. Následne sa stačí odosielať nové záznamy na definovanú URL. Tie vyhovujúce definovaným filtrom budú prístupné vo výsledkoch daného príkazu.

Tretím cieľom práce je umožnenie práce s historickými dátami. Jedným spôsobom realizácie bude možnosť odosielania súborov obsahujúcich dáta vo formáte XML. Druhým zaujímavejším riešením bude možnosť presmerovania výsledkov konkrétneho príkazu na vstup esper engine ako nový zdroj dát.

Súčasťou práce je úvod do problematiky NoSQL databáz, ktorých použitie je vhodné hlavne pri spracovaní veľkého množstva dát. Keďže esper engine je stavaný na takéto úlohy bude pre ukladanie výsledkov použitá práve NoSQL databáza.

Použitie diplomovej práce vyžaduje základnú znalosť esper syntaxe a je koncipovaná ako pre začiatočníkov, ktorí môžu k esperu pristupovať bez znalosti javy tak pre pokročilých užívateľov, ktorým môže uľahčiť a sprehľadniť prácu. Súčasťou práce je tiež postup prípravy systému a inštalácia, keďže tieto úkony nie sú triviálne.

Pri písaní práce som čerpal prevažne z online dokumentácie esperu a ruby on rails frameworku. Tiež som využil bakalársku prácu Štefana Repčeka \cite{bp-repcek} a diplomovú prácu Jána Dema \cite{dp-demo} ako množstvo iných dostupných materiálov dostupných prevažne online.

\emptydoublepage
\chapter{Vysoká dostupnosť}
V tomto oddiele sa budem venovať priblíženiu pojmu vysokej dostupnosti a bežným riešeniam, ktoré sa používajú na riešenie tejto problematiky. Predvediem spôsob počítania dostupnosti z manažérskeho pohľadu a problém v rozlíšení nedostupnosti systému a služby.

\section{Čo je vysoká dostupnosť}
Vysoká dostupnosť, tiež nazývaná \ac{RAS} označuje počítačový systém, ktorý sa dokáže rýchlo obnoviť z výpadku. Môže nastať minúta alebo dva nedostupnosti, počas ktorých systém migruje, potom ale bude ďalej pokračovať v činnosti bez zásahu administrátora. Neznamená to to isté ako fault-tolerant systém, kde je nepretržitá prevádzka navrhovaná tak, aby bolo možné pokračovať pri výpadku bez jeho zaznamenania. Systémy vysokej dostupnosti tiež umožňujú upravovať jednotlivé komponenty systému, alebo reštartovať jednotlivé servery bez nutnosti celé riešenie dočasne odstaviť\cite{web:pcmag.com-ha}.

Vysoká dostupnosť môže byť pochopená viacerými spôsobmi. Musíme rozlišovať, či hovoríme o plánovanej alebo neplánovanej nedostupnosti. Plánovaná nedostupnosť zahŕňa prípady kedy systém vypneme zámerne, či už v dôsledku reštartu po inštalácii, aktualizácií, alebo zmene konfigurácie systému. Tomu zvyčajne predchádzajú prípravy takejto udalosti na úrovni managementu a taktiež komunikácie so zákazníkmi. Neplánované výpadky sú tažko predvídateľné a ich príčiny môžu byť rôzneho pôvodu - od hardwarových a softwarových problémov až po chyby ľudí a prírodné katastrofy. Príkladom môžu byť výpadky RAM pamätí, pevných diskov, procesorov, sieťovej infraštruktúry, alebo útoky z internetu. Plánované výpadky nemajú veľký vplyv na obavy o dostupnosť systémov, keďže sú realizované prevažne v časoch, kedy tieto systémy nie sú využívané a zákazníci sa na nich môžu dopredu pripraviť.

Ako najjednoduchšie riešenie sa v praxi používa pravidelné zálohovanie. Zálohovanie síce rieši problém obnovy dát a všetky seriózne firmy dôležité dáta zálohujú, ale to je len prvý a krátkozraký krok, pretože obnova záloh je zdĺhavý proces sprevádzaný výpadkom. Trvanie nedostupnosti môže tvať niekoľko hodín, až dní v závislosti od závažnosti a času poruchy. Pri obnove zo záloh je totiž často potrebné zaobstarať hardware podobný tomu predchádzajúcemu, nainštalovať pôvodný operačný systém, potrebné aplikácie a značnú dobu tiež trvá kopírovanie veľkého množstva dát. Tiež sa líši reakčná doba v pracovnom čase a počas víkendu. Trvanie výpadku v rádu hodín alebo dní je však často neakceptovateľné.

V mnohých prípadoch je výpadok systému tolerovateľný len počas niekoľkých sekúnd. Ak na pár minút prestane fungovať elektronická pošta alebo konferenčné hovory tak si väčšina ľudí ide uvariť kávu. Avšak v prípade, že sa v letovej veži namiesto polohy lietadiel ukáže čierna obrazovka, je tento stav tolerovateľný len na pár sekúnd. Software použitý v praktickej časti (Pacemaker) slúži okrem iného práve na udržanie stálej prevádzky letovej spoločnosti DFS.

Systémy vysokej dostupnosti síce nie sú životne dôležité v každej situácii, avšak väčšina poskytovateľov elektronických služieb k nim smeruje už len z dôvodu prilákania zákazníkov. Či už ako marketingový plán alebo spôsob ako predísť migrácii klientov ku konkurencii v dôsledku príležitostnej nedostupnosti ich služieb. Napríklad hostingy internetových stránok často ponúkajú priestor zdarma. Od plateného sa väčšinou líši okrem iného aj garantovanou dostupnosťou.

Pri počítaní dostupnosti systému treba brať do úvahy aj rozhodnutie ako pristupovať k nedostupnosti konkrétnej služby. Nie sú výnimočné prípady kedy je operačný systém v poriadku, ale konkrétna aplikácia neodpovedá. Stáva sa to napríklad pri preťažení aplikácie väčším množstvom požiadavkov ako je schopná spracovať, aj keď je systém v poriadku. Vidíme, že v tomto prípade sa bude aplikácia užívateľom javiť ako nedostupná a zároveň administrátor môže tvrdiť 100\% dostupnosť.

Dopady systémových výpadkov sa dajú rozdeliť na krátkodobé a dlhodobé. Medzi tie krátkodobé patria stratený zisk a produktivita. Ich cena sa vyčísľuje oveľa ľahšie, avšak často tvoria len špičku ľadovca. Z dlhodobého hľadiska firma prichádza o svoje dobré meno, lojalitu zákazníkov, investorov. Výnimkou nie sú zmluvné pokuty, počítať treba aj s cenou obnovenia dát. Plný dopad trojdňového výpadku dobre znázorňuje príklad firmy RIM (BlackBerry), ktorá vyčíslila straty vo výške 54 miliónov dolárov. Na obrázku \ref{image_downtime-cost} je vidieť odhadovaný dopad hodinového výpadku pre rôzne odvetvia.

\myFigure {downtime-cost} {Cena hodinového výpadku v rôznych odvetviach \cite{pdf:managing-the-cost-of-downtime}} {Cena hodinovej nedostupnosti}

V praxi je často spomínaná percentuálna dostupnosť systémov, definovaná aj ako "`počet deviatok"'. Na prvý pohľad sa môže zdať číslo 99\% veľa, z obrázku \ref{image_availability} však možno vyčítať, že tomu odpovedajúca doba výpadku môže byť nepríjemná a to najmä v závislosti na čase kedy sa vyskytne. Vysoká dostupnosť zvyčajne znamená 3 a viac deviatok a počítajú sa do nej aj plánované výpadky \cite{pdf:ha-and-disaster-recovery}.

\myFigure {availability} {Percentná dostupnosť v závislosti na dobe dostupnosti systému \cite{pdf:ha-blueprints}} {Dostupnosť v percentách}

Konfigurácie využívajúce vysokú dostupnosť sú využívané predovšetkým v odvetviach, kde by výpadok mal veľký dopad na užívateľov a tých ktoré vyžadujú 24-hodinovú prevádzku, napríklad nemocnice, banky, letecké spoločnosti alebo nukleárne zariadenia.

Pri predstavení pojmu vysoká dostupnosť je potrebné vysvetliť význam ďalšej dôležitej skratky \ac{SPOF}. Tento všeobecný pojem sa používa pre komponent systému ktorý v prípade chyby spôsobí znefunkčnenie celého systému. Pritom nezáleží či je komponent hardwarový, softwarový alebo elektrický. Príkladom môže byť napájanie vysoko dostupného serverového klastra z jedného elektrického zdroja. V prípade chyby zásuvky je celá redundantná infraštruktúra zbytočná \cite{pdf:ha-blueprints}.

\section{Test desiateho poschodia}
Tento termín vymyslel Steve Traugott a týka sa situácie kedy potrebujeme obnoviť prevádzku systému do pôvodného stavu a to v čo najkratšom čase po výpadku. Test je založený na jednoduchej otázke "`Môžem zobrať akýkoľvek bežiaci stroj ktorý nikdy nebol zálohovaný a vyhodiť ho z desiateho poschodia tak, aby firma neprišla o dáta a obnoviť pôvodnú prevádzku v intervale 5-10 minút?"' Ak môžte odpovedať áno, testom ste prešli \cite{web:infrastructures.org}.

Práca systémových a aplikačných administrátorov vo veľkých firmách spočíva aj v príprave na takéto úlohy. Nie je výnimkou, že sa pravidelne testuje pripravenosť personálu obnoviť pôvodnú prevádzku vypnutím niektorého z počítačov - samozrejme v kontrolovaných podmienkach a s možnosťou opätovného nasadenia pôvodného systému v prípade poruchy.

\section{Ako je možné zvýšiť dostupnosť}
%http://networksandservers.blogspot.com/2011/02/high-availability-solutions.html

Existuje obrovské množstvo spôsobov ako zabezpečiť, že systémy budú fungovať stabilnejšie a vydržia dlhšie. Tým najjednoduchším je často nákup kvalitného hardwaru a pravidelná aktualizácia, avšak každý komponent má obmedzenú životnosť a istú náchylnosť k pokazeniu sa. V nasledujúcich odstavcoch zhrniem tie najpoužívanejšie so zameraním na tri hlavné oblasti.

\subsection{Dostupnosť úložného priestoru}
Dáta sú základným kameňom, na ktorom stavia každá firma. Či už sú to firemné faktúry, zoznamy zákazníkov alebo internetové stránky, žiadna firma si nemôže dovoliť o ne prísť. Je nepredstaviteľné že by banka svojim klientom oznámila, že stratila záznamy o paňažných zostatkoch na ich účtoch. Tieto informácie sú uchovávané na pevných diskoch, ktorých dostupnosť sa dá zvýšiť jedným z nasledujúcich spôsobov.

\subsubsection{RAID}
\label{lbl:sec:raid}
Toto riešenie používa takmer každá firma - či už malá alebo veľká. \ac{RAID} môže byť hardwarový aj softwarový. Každý z nich má svoje výhody:\cite{web:cyberciti.biz}

\begin{description}
	\item[Softwarový] raid je zvyčajne zdarma ako súčasť operačného systému a poskytuje častokrát väčšiu konfiguračnú flexibilitu. S vývojom CPU sa do istej miery zvyšuje aj výkon RAIDu, avšak niekedy je práve tento výkon potrebný pre iné aplikácie, čo môže byť nevýhodou. Nástroje na ich správu a konfiguráciu sú zvyčajne špecifické pre daný operačný systém, čo vyžaduje viac času administrátorov. Softwarový raid je vhodnejší pre použitie doma alebo v menších firmách. Známy nástroj na konfiguráciu RAIDu v linuxe je napríklad mdadm.
	\item[Hardwarový] raid je fyzická karta, ktorá samozrejme nie je zdarma. Ich cena však prináša radu výhod, ako odbremenenie CPU a RAM alebo jednoduchšiu správu. Každá karta je však špecifická v závislosti na výrobcovi, čoho následkom je menej dostupná technická špecifikácia. Systém sa stáva závislým na konkrétnej karte, čo môže byť problém v prípade jej chyby a nutnosti obnovy dát pomocou iného hardwaru. Možnosti konfigurácie sa veľmi líšia v závislosti na type karty a typicky aj cene. V prípade chyby výhodu predstavuje jednoduchšia výmena diskov a podpora hot-swap. Hardwarový raid je vhodný najmä pre systémy s vysokou záťažou.
\end{description}

RAID pole je možné rozlíšiť podľa spôsobu zapojenia diskov \cite{pdf:ha-blueprints}. Tie bežné sa označujú číslovaním od RAID-0 po RAID-6. Existuje však mnoho iných konfigurácií. Niektoré z nich je možné kombinovať, čoho príkladom je RAID-10. Jedny z najrozšírenejších sú nasledujúce tri typy:
\begin{description}
	\item[RAID-0] sa vyznačuje nulovou tolaranciou k poškodeniu disku. V praxi takéto zapojenie vyzerá ako keby sme sériovo spojili 2 disky. Jeho výhodou je predovšetkým zvýšenie rýchlosti prístupu k disku a zmenšenie počtu diskových jednotiek - spája menšie disky do jedného väčšieho.
	\item[RAID-1] niekedy nazývaný ako zrkadlenie diskov. Disky sú zapojené paralelne a majú rovnaký obsah. Tento spôsob je využívaný v prípade, kedy kladieme dôraz na zachovanie dát v prípade chyby jedného z diskov. Zlyhanie užívateľ nezaznamená, všetky operácie prebiehajú na vrstve pod súborovým systémom.
	\item[RAID-5] vyžaduje minimálne 3, bežne sa však používa 5 diskov. V tomto zapojení sú disky reťazené a zároveň je časť z nich využitá na redundanciu. V prípade výpadku niektorého z nich sú stratené dáta dopočítané zo zvyšných diskov. Je vyhľadávaný hlavne kôli svojej nízkej cene, avšak nie je vhodný pre systémy, ktoré kladú dôraz na vysoký výkon.
\end{description}

Systém RAID rieši len problém na úrovni výpadku diskov, nie dostupnosť celého systému (aj keď k nej prispieva). Ak nastane chyba v akomkoľvek inom hardwarovom komponente, nastane výpadok. RAID taktiež nerieši chyby ľudského faktoru a aplikácií a v žiadnom prípade nie je náhradou pravidelného zálohovania.

\subsubsection{SAN}
\ac{SAN} je vysoko rýchlostná dedikovaná sieť, ktorá spája rôzne druhy dátových zariadení s asociovanými servrami. Jednoduchšie povedané je to sieť, ktorá spája počítače so zariadeniami na ukladanie dát. Na vybudovanie takejto siete je možné použiť viacero technológií, napríklad Fibre Channel alebo iSCSI. Technológia SAN podporuje zrkadlenie diskov, zálohovanie, obnovu a migráciu dát z jedného úložného zariadenia na iné a zdieľanie dát medzi rôznymi servrami v sieti. Oproti lokálnym diskom má viacero výhod \cite{web:storage-area-network-1-uvod}:

\begin{itemize}
	\item Odstraňuje vzdialenostné limity lokálne pripojených diskov
	\item Zvyšuje výkon, dáta sú dostupné rýchlosťou v jednotkách Gb/s
	\item Zvyšuje spoľahlivosť umožnením viacerých prístupových ciest k úložisku
	\item Jednoduchšie možnosti obnovy po havárii. Diskové polia môžu byť zrkadlené do iných lokalít
	\item Jednoduchšia administrácia a flexibilita, keďže káble a úložiská dát nemusia byť fyzicky presúvané ak ich chceme premiestniť na iný server
\end{itemize}

\subsubsection{NAS}
Sieťové úložisko alebo \ac{NAS} je počítač s pripojením na sieť, ktorý má fungovať ako dátové úložisko pre ostatné zariadenia. NAS zvyčajne nemajú konektory pre display alebo klávesnicu. Namiesto toho sú konfigurovateľné pomocou webového rozhrania. Ich operačný systém je často upravený a okresaný. Úložisko dát sa zvykne rozkladať na hot-swap diskovom poli v RAIDe. Pre prístup k dátam NAS používa protokoly ako \ac{NFS} na unixových systémoch alebo \ac{SMB} známy aj pod prezývkou Samba, ktorý je populárny na Windowsoch.

Oproti SAN systémom sú jednoducho konfigurovateľné a ich správa je jednoduchá. NAS nie je použitím limitovaný len ako dátové úložisko pre klientské a serverové stanice. Umožňuje jednoduché a nízkonákladné riešenie dátového úložiska pre servery s rozložením záťaže alebo s toleranciou k výpadku.

Hlavné rozdiely medzi SAN a NAS sa dajú zosumarizovať do troch bodov:\cite{web:Difference-Between-NAS-and-SAN-3-Considerations}
\begin{itemize}
	\item Pričinok vs disk: NAS poskytuje súborový systém (NFS, SMB), naproti tomu SAN poskytuje blokové zariadenie. Klientski sa tak musia v druhom prípade postarať o vlastný súborový systém.
	\item Výkon vs cena: SAN je zvyčajne viac výkonný ako NAS, čo sa prejavuje aj v jeho cene. Pretože SAN využívajú technológiu Fibre Channel, poskytuje oveľa vyššie rýchlosti ako IP siete
	\item Jednoduchosť: NAS zariadenia zvyčajne fungujú priamo po vybalení z krabice alebo s minimálnou konfiguráciou.
\end{itemize}

Rozdiel v použití v sieťovom zapojení môžme vidieť na obrázku \ref{image_san-vs-nas}
\myFigure{san-vs-nas}{Roziel v možnosti sieťového zapojenia NAS a SAN\cite{web:Difference-Between-NAS-and-SAN-3-Considerations}} {Zapojenie NAS vs SAN}

\subsection{Dostupnosť aplikácií}
Keď máme dátove úložisko dostatočne chránené proti výpadku, je potreba sa sústrediť na vyššiu vrstvu - aplikácie. Tie sú práve nástrojom, ktorý využívajú klienti, takže sú nedeliteľnou súčasťou každého systému. Ich dostupnosť sa dá rovnako ako pri dátovom úložisku zabezpečiť viacerými spôsobmi.

\subsubsection{Failover}
Failover znamená uvedenie náhradného systému do prevádzky v prípade, že primárny systém zlyhá. Aktuálne kópie všetkých požadovaných dát a aplikácií sú udržiavané na sekundárnom systéme, aby ho bolo možné spustiť v čo najkratšom čase. \cite{web:pcmag.com-failover}
Failover je dôležitou súčasťou kritických systémov, keďže automaticky presmeruje užívateľské požiadavky na záložný systém, ktorý má rovnakú alebo podobnú funkcionalitu ako ten pôvodný.

\subsubsection{Load balancing}
Pod pojmom load balancing sa rozumie rovnomerné rozloženie záťaže na dostupné servry v sieti, prípadne na disky v SAN. Vysoko dostupné aplikácie môžu fungovať na tomto princípe rozložením požiadavok medzi viaceré servery, ktoré v prípade výpadku niektorého z nich preberú jeho prácu. Podobne funguje aj Linux Virtual Server, ktorý popíšem bližšie v kapitole \ref{chap:mozne_riesenia}.

\subsection{Dostupnosť siete}
Nedeliteľnou súčasťou systémov vysokej dostupnosti je zabezpečenie dostupnosti siete. To sa dosahuje napríklad použitím záložných sieťových prvkov či náhradných pripojení. Aj keď je táto oblasť vysokej dostupnosti dôležitá, v práci sa venujem dostupnosti súborových systémov a aplikácií, preto sa riešením siete nebudem zaoberať.

\emptydoublepage
\chapter{Technológie}
\label{chap:technologie}

\section{Esper}
	Esper je komponenta, ktorá umožňuje spracovanie komplexných udalostí \ac{CEP}. Umožňuje vývoj aplikácií spracovávajúcich veľké množstvo udalostí - v reálnom čase ako aj historických. Tieto udalosti je možné filtrovať a analyzovať podľa potreby a reagovať v reálnom čase na predom definované stavy.  Esper je dostupný v troch verziách:
	\begin{description}
		\item[Esper] je open source s možnosťou komerčnej podpory. Táto verzia obsahuje základ potrebný pre realizáciu CEP, užívateľ však musí jednotlivé príkazy, schémy a nastavenia realizovať programovo. Je preto náročný na použitie pre ľudí, ktorí nevedia programovať. Riešenie je vhodné pre firmy, ktoré buď nevyužijú platenú verziu alebo majú špecifické požiadavky na výsledný produkt a sú schopné túto verziu podľa svojich potrieb upraviť.
		
		\item[Esper HA] je riešenie umožňujúce vysokú dostupnosť Esperu. Zabezpečuje že stav je po vypnutí alebo havárii obnoviteľný. Príkazy, schémy a iné nastavenia si EsperHA pri reštarte uchováva, čo je výhoda oproti open source verzii, kde je nutné tieto úkony riešiť programovo. Táto verzia je vhodná pre projekty závislé na vysokej dostupnosti Esperu a subjekty, pre ktoré je kritická neustála kontrola prichádzajúcich udalostí.
		EsperHA je spoplatnený, dostupná je trial len verzia, pre ktorej použitie je nutné identifikovať sa ako spoločnosť. Cena nie je na webových stránkach dostupná.
		
		\item[Esper Enterprise Edition] je kompletný produkt ``na kľúč'', obsahujúci všetky komponenty potrebné pre nasadenie do podniku. V jednom balíku je obsiahnuté GUI pre správu Esperu, restové služby poskytujúce prístup zvonku, \ac{EPL} editor, nástroje umožňujúce kontinuálne zobrazenie výsledkov v grafoch a tabuľkách. EsperEE je možné skombinovať s EsperEA pre dodatočné zabezpečenie vysokej dostupnosti. EsperEE je spoplatnený, rovnako ako pri EsperEA je dostupná trial verzia po splnení určitých podmienok. Cena nie je zverejnená a tieto dve riešenia sú určené predovšetkým pre podnikový sektor.
	\end{description}
	
	Pre tento projekt je použitá verzia Esper, ktorú som rozšíril o prístup k základným funkciám pomocou restovej api a persistenciu niektorých nastavení a nájdených výsledkov. Aktuálna verzia 5.1 je dostupná pod GNU General Public License (GPL) (GPL v2).

	\subsection{Typy udalostí}
	Každá udalosť spracovávaná Esperom je definovaná schémou, takzvaným typom udalosti. Tie môžu byť načítané pri štarte aplikácie, alebo nastavené programovo počas behu. EPL obsahuje klauzulu CREATE SCHEMA umožňujúcu definovanie typu udalosti. Prehľad základných typov udalostí je v tabuľke \ref{table:event-types}.

	\myTable{
	\begin{tabular}{ | l | p{10cm} | }
		\hline
		Trieda	&	Popis	\\ \hline
		java.lang.Object	&	Akýkoľvek Java \ac{POJO} s getter metódami. Takáto definícia je najjednoduchšia na úkor možnosti úprav počas behu programu.	\\ \hline
		java.util.Map	&	Udalosti definované ako implementácia java.util.Map interface, kde každá hodnota záznamu je vlastnosť udalosti.	\\ \hline
		Object[] (pole objektov)	&	Udalosti definované objektovým poľom, kde každá hodnota poľa je vlastnosť udalosti.	\\ \hline
		org.w3c.dom.Node	&	XML objektový model dokumentu popisujúci štruktúru udalosti.	\\ \hline
	\end{tabular}
	}{Možnosti definície typu udalosti}{table:event-types}
	
	Definície typu udalosti sú rozšíriteľné zásuvnými modulmi. Aplikácia môže používať kombináciu týchto typov, nemusí všetky typy definovať jedným spôsobom. Definície typov udalostí je možné reťaziť, kde typom udalosti môže byť iná komplexná udalosť.
	
	Z dôvodu nutnosti pridávania a mazania udalostí počas behu programu nemôže byť v tejto implementácii použitá definícia typu pomocou POJO. A pretože klient musí mať možnosť definovať typ, bola zvolená definícia pomocou XML schémy. Príklad jednoduchej schémy udalosti znázorňuje výpis \ref{lst:sample-schema}.
	
	\begin{lstlisting}[label=lst:sample-schema,caption=Príklad XML schémy udalosti]
	<?xml version="1.0" encoding="UTF-8"?>
	<xs:schema xmlns:xs="http://www.w3.org/2001/XMLSchema">
		<xs:element name="StockEvent">
			<xs:complexType>
				<xs:sequence>
			        <xs:element name="time" type="xs:string"></xs:element>
			        <xs:element name="open" type="xs:float"></xs:element>
			        <xs:element name="high" type="xs:float"></xs:element>
			        <xs:element name="low" type="xs:float"></xs:element>
			        <xs:element name="close" type="xs:float"></xs:element>
			        <xs:element name="volume" type="xs:float"></xs:element>
				</xs:sequence>
			</xs:complexType>
		</xs:element>
	</xs:schema>	
	\end{lstlisting}
	
	Po definovaní typu udalosti je Esper engine schopný prijímať udalosti v XML formáte. Príklad udalosti vyhovujúcej schéme \ref{lst:sample-schema} je vo výpise \ref{lst:sample-event}.
	\begin{lstlisting}[label=lst:sample-event,caption=Príklad XML udalosti]
	<?xml version="1.0"?>
	<events>
		<StockXsd>
			<time>2014-02-03 01:58:00.000</time>
			<open>1.34850</open>
			<high>1.34854</high>
			<low>1.34850</low>
			<close>1.34853</close>
			<volume>76.9400</volume>
		</StockXsd>
		<StockXsd>
			<time>2014-02-03 01:59:00.000</time>
			<open>1.34852</open>
			<high>1.34853</high>
			<low>1.34845</low>
			<close>1.34850</close>
			<volume>89.5800</volume>
		</StockXsd>
	</events>
	\end{lstlisting}
	Ako je z výpisu vidieť je možné udalosti zaobaliť do koreňového elementu events. Ten je vhodné použiť v prípade že na Esper engine odosielame súbory obsahujúce veľké množstvo udalostí, pretože tým obmedzíme počet HTTP volaní a predídeme možnému zahlteniu serveru. Táto funkcionalita je realizovaná v implementačnej časti aplikácie a nie je súčasťou Esperu.

	Po definovaní typu udalosti sa na ne môžeme odkazovať klauzulou FROM v EPL príkazoch. Tie sú bližšie popísané v nasledujúcej sekcii.

	\subsection{Event Processing Language}
		\ac{EPL} je jazyk umožňujúci definovanie príkazov a vzorov v CEP. Syntaxou je podobný SQL, pretože obsahuje klauzuly ako SELECT, FROM, WHERE, GROUP BY, HAVING alebo ORDER BY. Namiesto tabuliek však pracuje so tokmi udalostí, kde riadok tabuľky nahrádza prichádzajúca udalosť. Toky udalostí je možné spájať pomocou JOIN, filtrovať alebo agregovať.
		
		EPL definuje koncept pomenovaných okien (named windows), ktoré slúžia ako štruktúra uchovávajúca udalosti. Je možné do nej vkladať nové udalosti a mazať staré. Výhodou tejto štruktúry je možnosť jej použitia viacerými príkazmi, pretože je globálna, teda zdieľaná v rozsahu daného service providera.	
		
		Pomocou EPL môžeme tiež definovať premenné, ktoré sa dajú následne použiť napríklad na vkladanie parametrov do príkazov.
		
		\subsubsection{Syntax}
		Príkazy musia spĺňať pravidlá definované EPL syntaxou. Tá však nie je jednotná a jednotlivé CEP riešenia poskytujú svoje implementácie. Táto časť práce sa zaoberá pravidlami, ktoré používa jazyk EPL Esperu. Výpis \ref{lst:epl-syntax} zobrazuje štruktúru tejto syntaxe.
		
		\begin{lstlisting}[label=lst:epl-syntax,caption=Vzor EPL syntaxe \cite{web:esper-doc}]
		[annotations]
		[expression_declarations]
		[context context_name]
		[into table table_name]
		[insert into insert_into_def]
		select select_list
		from stream_def [as name] [, stream_def [as name]] [,...]
		[where search_conditions]
		[group by grouping_expression_list]
		[having grouping_search_conditions]
		[output output_specification]
		[order by order_by_expression_list]
		[limit num_rows]
		\end{lstlisting}

		Ako môžeme vidieť v tomto výpise, každý EPL príkaz musí obsahovať minimálne klauzuly SELECT a FROM. Ďalej je možné filtrovať pomocou klauzuly WHERE, spájať prúdy udalostí pomocou JOIN alebo využiť relačnú databázu ako zdroj udalostí. Nasledujúci popis rozoberá základné EPL klauzuly \cite{web:esper-doc}.
		
		\begin{description}
			\item[Select] Klauzula SELECT je povinná v každom EPL príkaze. Je v nej možné využiť náhradný znak * alebo vymenovať všetky požadované položky. Ak položka nemá unikátne meno, musí sa použiť predpona s názvom zdroja dát. 
			
			V prípade použitia znaku * v JOIN príkaze nebude výsledná udalosť obsahovať všetky položky oboch zdrojov. Namiesto toho bude pozostávať z položiek reprezentujúcich objekty daných udalostí pomenované podľa zdrojov.
			
			Syntax select klauzuly je znázornená vo výpise \ref{lst:select-syntax}. Môže obsahovať aj nepovinné parametre istream (input), irstream (input \& remove) a rstream (remove), ktoré definujú na ktoré udalosti príkaz reaguje. Prednastavené je použitie parametru istream.

			\begin{lstlisting}[label=lst:select-syntax,caption=Syntax SELECT klauzuly \cite{web:esper-doc}]
select [istream | irstream | rstream] [distinct] * | expression_list
			\end{lstlisting}
			Obsah klauzuly select tiež definuje typ udalostí vyprodukovaných daným príkazom.
			
			\item[FROM] FROM klauzula špecifikuje jeden alebo viac zdrojov, pomenovaných okien alebo tabuliek (od verzie Esper 5.1). Tie môžu byť pomenované klauzulou AS. Pre join je potrebné definovať viacero zdrojov dát. Syntax from klauzuly je vo výpise \ref{lst:from-syntax}.
			\begin{lstlisting}[label=lst:from-syntax,caption=Syntax FROM klauzuly \cite{web:esper-doc}]
from stream_def [as name] [unidirectional]
	[retain-union | retain-intersection] 
[, stream_def [as stream_name]] [, ...]
			\end{lstlisting}
			Podporovaný je tiež join s relačnou databázou ako zdrojom dát. To je možné využiť napríklad na prístup k historickým dátam.
			
			\item[WHERE] Where klauzula je nepovinná časť príkazu, ktorá špecifikuje vyhľadávacie parametre. Zvyčajne obsahuje výrazy pozostávajúce z porovnávacích operátorov =, \textless , \textgreater , \textgreater=, \textless=, !=, \textless\textgreater, exists, is null a ich kombinácie pomocou kľúčových slov AND a OR.
			\begin{lstlisting}[label=lst:where-syntax,caption=Syntax WHERE klauzuly \cite{web:esper-doc}]
where exists (
	select orderId from Settlement.win:time(1 min) 
		where settlement.orderId = order.orderId
)
			\end{lstlisting}
			Klauzula where môže obsahovať tiež vnorené výrazy, ako je to znázornené vo výpise \ref{lst:where-syntax}.
			
			\item[JOIN] Klauzula FROM môže obsahovať viacero zdrojov dát. V tom prípade sú dátové zdroje spojené pomocou JOIN. Predvolene je použitý inner join, ktorý produkuje udalosti len v prípade výskytu vyhovujúcej udalosti vo všetkých zdrojoch. V prípade použitia outer join sa chýbajúce udalosti nahradia hodnotou null.
	
			K dispozícii sú tiež varianty left outer join, right outer join a full outer join. Výpis \ref{lst:join-syntax} znázorňuje pravidlá syntaxe pri použití join.
			\begin{lstlisting}[label=lst:join-syntax,caption=Syntax JOIN klauzuly \cite{web:esper-doc}]
...from stream_def [as name] 
((left|right|full outer) | inner) join stream_def 
[on property = property [and property = property ...] ]
[ ((left|right|full outer) | inner) join stream_def [on ...]]...
			\end{lstlisting}
			Každý z prúdov dát definovaný pomocou join klauzuly obsahuje vstupný a výstupný stream. Join tak môže byť realizovaný pri prijatí udalosti v ktoromkoľvek z týchto prúdov. Join je teda viacsmerový, prípadne dvojsmerný pri použití dvoch prúdov dát.
			EPL definuje kľúčové slovo unidirectional, ktoré umožňuje identifikovať jediný prúd dát poskytujúci udalosti ktoré spustia join. Všetky ostatné prúdy sa stanú pasívnymi. Keď je prijatá udalosť pasívnym prúdom dát, negeneruje join novú udalosť.
	
			\item[OUTPUT] Output klauzula umožňuje kontrolovať rýchlosť ktorou sú produkované udalosti. Zvyčajne sa používa spolu s určením časového údaju. Tým môže byť napríklad výstup každých n sekúnd, n udalostí alebo v daný čas dňa. Časy výstupu je možné definovať tiež vo formáte cronu. Jeden zo spôsobov zápisu zobrazuje výpis \ref{lst:output-syntax}.
			\begin{lstlisting}[label=lst:output-syntax,caption=Syntax OUTPUT klauzuly \cite{web:esper-doc}]
output [after suppression_def] 
[[all | first | last | snapshot] every output_rate [seconds | events]]
			\end{lstlisting}
			V tomto výpise vidíme aj možnosť definovania toho čo sa má vyprodukovať. Je možné si vybrať produkovanie všetkých udalostí, prvej, poslednej alebo snímky. Snímka sa používa spolu s agregačnými funkciami a produkuje jedinú udalosť s hodnotou agregačnej funkcie.
		\end{description}

		Esper navyše umožňuje presmerovávať toku udalostí, prípadne ich za behu upravovať nasledujúcimi klauzulami:
		\begin{description}
			\item[INSERT INTO] Túto klauzulu je možné použiť pre vloženie výsledkov príkazu do pomenovaného okna alebo tabuľky. Tiež umožňuje presmerovať tieto výsledky ako vstupný tok pre iný príkaz. Syntax pre použitie klauzuly insert into je zobrazená vo výpise \ref{lst:insertinto-syntax}. Príklad použitia je dostupný v nasledujúcom texte.
			\begin{lstlisting}[label=lst:insertinto-syntax,caption=Syntax INSERT INTO klauzuly \cite{web:esper-doc}]
			insert [istream | irstream | rstream] into event_stream_name  [ (property_name [, property_name] ) ]
			\end{lstlisting}
			\item[UPDATE] Klauzula UPDATE slúži na úpravu vlastností udalosti a je aplikovaná pred spracovaním príkazu.
		\end{description}

		EPL tiež umožňuje definovať náhľady, ktoré predstavujú istú obdobu náhľadov (view) ako ich poznáme z databázových prostredí. CEP sa ale zaoberá prácou s tokmi dát a nie statickým pohľadom na ne, preto rozširuje tieto náhľady o viacero funkcií. Tými sú napríklad tvorenie štatistík z vlastností udalostí, ich zoskupovanie či funkcie pre umožnenie výberu udalostí, ktoré bude dátové okno obsahovať. Náhľady môžu byť reťazené. Názorný príklad fungovania náhľadov je na obrázku \ref{image_cep-windows}.
	
		\myFigure{cep-windows}{Príklad dátového okna obmedzeného časom a počtom udalostí \cite{web:softwarearchitekturen}}{Príklad dátového okna obmedzeného časom a počtom udalostí}

		Esper rozdeľuje náhľady do menných priestorov. V nasledujúcom zhrnutí sú predstavené tie základné \cite{web:esper-doc}.
		\begin{description}
			\item[Náhľady do dátových okien] definujú kĺzavé okná a nájdeme ich v mennom priestore win.
				\subitem win:length - náhľad rozširuje okno o definovaný počet udalostí do minulosti. Nové udalosti vytláčajú tie, ktoré sa do okna už nezmestia, čo vytvára štruktúru podobnú konceptu fifo.
				\subitem win:length\_batch - náhľad funguje podobne ako win:length, avšak udalosti odstraňuje nárazovo pri zaplnení okna o definovanej veľkosti.
				\subitem win:time - náhľad rozširuje dátové okno o minulé udalosti obmedzené časovou značkou.
				\subitem win:keepall - na rozdiel od predchádzajúcich náhľadov, ktoré udalosti nevyhovujúce podmienke z dátového okna odstránia, tento náhľad udržuje\ všetky prijaté udalosti. Keďže sú všetky udalosti udržiavané v pamäti je nutné dať si pozor aby náhľad nezabral všetku dostupnú operačnú pamäť.
				\subitem win:firstlength - je podobný win:length v tom že obmedzuje počet udalostí. Naproti nemu však v dátovom okne udržiava len prvých n udalostí.
				\subitem win:firsttime - je ekvivalentný s win:firstlength, avšak udalosti nie sú obmedzené počtom, ale časom.

			\item[Štandardné náhľady] Ostatné bežne používané náhľady sú dostupné s predponou std.
				\subitem std:unique - tento náhľad uchováva len najaktuálnejšiu udalosť v prípade prijatia duplicitnej udalosti.
				\subitem std:size - náhľad poskytuje prístup k premennej size, ktorá obsahuje počet udalostí prijatých daným príkazom. Náhľad vytvára novú udalosť len v prípade zmeny premennej size. 
				\subitem std:firstevent - náhľad udržiava len prvú prijatú udalosť. Všetky udalosti prijaté po nej sú ignorované. Toto správanie spôsobuje že je jeho forma podobná ako win:length o veľkosti 1.

			\item[Štatistické náhľady] Štatistické náhľady pokrýva menný priestor stat.
				\subitem stat:uni - umožňuje pristupovať k štatistickým hodnotám, napríklad priemeru, smerodajnej odchýlky, súčtu či rozptylu.
				\subitem stat:correl - počíta korelačnú hodnotu. Funkcia vyžaduje minimálne dva parametra, ktorých korelačnú hodnotu počíta.
				\subitem stat:weighted\_avg - ako názov napovedá, náhľad umožňuje vypočítať vážený priemer. Podobne ako stat:correl vyžaduje aspoň dva parametra, prvý udáva údaj z ktorého sa počíta priemer a druhý jeho váhu.
		\end{description}
		
		EPL jazyk tiež umožňuje definovať vzory. Patterny sú výrazy, ktoré hľadajú zhodu podľa definovaného vzoru. Je možné ich definovať ako samostatný výraz alebo ako súčasť príkazu. Môžu sa vyskytovať kdekoľvek v klauzule FROM, vrátane join. Vďaka tomu ich je možné použiť v kombinácii s klauzulami WHERE, GROUP BY, HAVING a INSERT INTO.
		
		V nasledujúcich výpisoch sú príklady príkazov, zobrazujúcich príklady syntaxe popisovanej v predchádzajúcom texte.
		\begin{lstlisting}[label=lst:epl-simple,caption=Jednoduchý EPL príkaz]
		select * from TweetEvent.win:time(60 sec) where message='happy'
		\end{lstlisting}
		
		\begin{lstlisting}[label=lst:output-example,caption=EPL príkaz s výstupom každých 60 sekúnd]
		select sum(price) from OrderEvent.win:time(30 min)
			output snapshot every 60 seconds
		\end{lstlisting}

		\begin{lstlisting}[label=lst:epl-join,caption=Jednoduchý EPL príkaz použitím join]
		select * from TickEvent.std:unique(symbol) as t,
			NewsEvent.std:unique(symbol) as n
		where t.symbol = n.symbol
		\end{lstlisting}

		\begin{lstlisting}[label=lst:epl-pattern,caption=EPL príkaz s použitím vzoru \cite{web:esper-doc}]
		select a.custId, sum(a.price + b.price)
		from pattern [every a=ServiceOrder -> 
			b=ProductOrder(custId = a.custId) where timer:within(1 min)].win:time(2 hour) 
		where a.name in ('Repair', b.name)
		group by a.custId
		having sum(a.price + b.price) > 100
		\end{lstlisting}
		
		\begin{lstlisting}[label=lst:insert-into,caption=Príklad použitia klauzuly INSERT INTO \cite{web:esper-doc}]
		insert into CombinedEvent
		select A.customerId as custId, A.timestamp - B.timestamp as latency
		from EventA.win:time(30 min) A, EventB.win:time(30 min) B
		where A.txnId = B.txnId
		\end{lstlisting}
		
		\begin{lstlisting}[label=lst:views,caption=Príklady použitia dátových náhľadov]
Počíta priemernú cenu akcie z udalostí prijatých v posledných 30 sekundách
select sum(price) from StockTickEvent(symbol='GE').win:time(30 sec)

Počíta počet udalostí StockTickEvent prijatých počas poslednej minúty
select size from StockTickEvent.win:time(1 min).std:size()

Počíta smerodajnú odchýlku z posledných 10 prijatých udalostí
select stddev from StockTickEvent.win:length(10).stat:uni(price)
		\end{lstlisting}
		
		\begin{lstlisting}[label=lst:update,caption=Príklady úpravy udalosti pred spracovaním \cite{web:esper-doc}]
update istream AlertEvent 
set severity = 'High'
where severity = 'Medium' and reason like '%withdrawal limit%'		
		\end{lstlisting}
		
		\subsubsection{Objektový model}	
		Objektový model je sada tried poskytujúcich objektovú reprezentáciu príkazu alebo vzoru. Tá umožňuje zostrojiť, zmeniť alebo získať údaje z EPL príkazov a vzorov na vyššom stupni ako pri práci s textovou reprezentáciou. Objektový model pozostáva z objektového grafu, ktorého prvky je jednoducho prístupné. Objektový model umožňuje plný export do textovej formy a naopak.
		
		Príkazy vo výpise \ref{lst:epl-nomodel} a \ref{lst:epl-model} sú ekvivalentné. Podobným spôsobom je možné vytvárať príkazy, vzory, definovať premenné premenné alebo vytvárať dátové okná.
		
		\begin{lstlisting}[label=lst:epl-nomodel,caption=EPL príkaz bez použitia objektového modelu \cite{web:esper-doc}]
	select line, avg(age) as avgAge 
	from ReadyEvent(line in (1, 8, 10)).win:time(10) as RE
	where RE.waverId != null
	group by line 
	having avg(age) < 0
	order by line
		\end{lstlisting}
		
		\begin{lstlisting}[label=lst:epl-model,caption=EPL príkaz s použitím objektového modelu \cite{web:esper-doc}]
	EPStatementObjectModel model = new EPStatementObjectModel();
	model.setSelectClause(SelectClause.create()
		.add("line")
		.add(Expressions.avg("age"), "avgAge"));
	Filter filter = Filter.create("com.chipmaker.ReadyEvent", Expressions.in("line", 1, 8, 10));
	model.setFromClause(FromClause.create(
		FilterStream.create(filter, "RE").addView("win", "time", 10)));
	model.setWhereClause(Expressions.isNotNull("RE.waverId"));
	model.setGroupByClause(GroupByClause.create("line"));
	model.setHavingClause(Expressions.lt(Expressions.avg("age"), Expressions.constant(0)));
	model.setOrderByClause(OrderByClause.create("line"));
		\end{lstlisting}
	
		Rovnako ako v textovej reprezentácii sú v objektovej reprezentácii klauzuly SELECT a FROM povinné. Pomocou objektového modelu je tiež možné skontrolovať syntax príkazu pred pridaním do Esper engine.

	\subsection{Api}
		Esper pre svoje ovládanie neposkytuje grafické rozhranie. Na komunikáciu používa api, ktorá definuje tieto primárne rozhrania:
		\begin{itemize}
			\item Rozhranie udalostí a ich typov
			\item Administrátorské rozhranie na vytváranie a správu EPL príkazov a vzorov a definovanie konfigurácie Esperu
			\item Runtime rozhranie, ktoré slúži na posielanie udalostí do Esperu, definovanie premenných a spúšťanie on-demand výrazov.
		\end{itemize}
		
		\subsubsection{EP Service Provider}
		EPServiceProvider reprezentuje konkrétnu inštanciu Esperu. Každá takáto inštancia je nezávislá od ostatných a má svoje vlastné administrátorské a runtime rozhranie. Pri prístupe umožňuje rozhranie voľbu ``getDefaultProvider'' bez parametrov, ktorá vráti predvolenú inštanciu, alebo ``getProvider'' s textovým parametrom URI identifikujúcim konkrétnu inštanciu. Tá je vytvorená ak ešte neexistuje. Opakované volania s rovnakým URI vracajú stále rovnakú inštanciu. Túto funkcionalitu je možné využiť napríklad pre oddelenie pracovného prostredia viacerých užívateľov.
		
		\subsubsection{EP Administrator}
		EPAdministrator umožňuje registrovanie EPL príkazov, vzorov alebo ich objektovej reprezentácie do Esperu a to metódami createPattern, createStatement a create pre objektový model. Tieto funkcie poskytujú voliteľné parametre umožňujúce definovať meno príkazu a užívateľský objekt, ktorý je v implementačnej časti tejto práce využitý na ukladanie dodatočných informácií o príkaze - napríklad definovanie TTL pri ukladaní výsledkov do databázy.
		
		Po registrácii nového EPL výrazu rozhranie vracia inštanciu vytvoreného EPStatement, pomocou ktorej môžeme ovládať už vytvorený príkaz alebo pristupovať k výsledkom. Praktická časť práce z ovládacích funkcií využíva stop() a start(), ktoré definujú, či je príkaz aktívny.
		
		Esper poskytuje tri možnosti ako pristupovať k výsledkom konkrétneho príkazu. Tieto je možné rôzne kombinovať. Možnosti sú predstavené v nasledujúcom výpise:
		\begin{description}
			\item[Listener] V prvom prípade aplikácia poskytuje implementáciu rozhraní UpdateListener alebo StatementAwareUpdateListener vytváranému príkazu. Takýto listener bude následne notifikovaný pri výskyte novej udalosti a metóde update bude predaná inštancia EventBean, ktorá obsahuje udalosť produkovanú niektorým z príkazov.
			\item[Subscriber] Týmto spôsobom Esper posiela výsledky na definovaný subscriber. Je to najrýchlejšia možnosť, pretože Esper predáva typované výsledky priamo do objektov aplikácie, nemusí teba zostavovať inštancie EventBean ako v predošlom prípade. Nevýhodou je že príkaz môže mať registrovaný maximálne jeden subscriber, naproti predošlému spôsobu, kde umožňoval definovať viacero listenerov.
			\item[Pull Api] Týmto spôsobom aplikácia pristupuje k výsledkom on-demand spôsobom, kde jednorázovou žiadosťou o výsledky daného príkazu získa zoznam EventBean prístupný pomocou iterátora. Toto je využiteľné v prípade kedy aplikácia nevyžaduje nepretržité spracovanie nových výsledkov v real-time.
		\end{description}
		V tomto projekte bol použitý prvý spôsob listenera. Aplikácia v tomto prípade použije implementáciu rozhrania StatementAwareUpdateListener, ktorá je registrovaná pri vytváraní nového príkazu metódou addListener. Vďaka použitiu rozhrania StatementAwareUpdateListener a nie UpdateListener získava aplikácia prístup k príkazu, ktorý konkrétnu udalosť vyprodukoval, pre všetky príkazy môže byť preto definovaný jediný spoločný listener.
		
		Esper podporuje tiež spracovanie udalostí, ktoré nevyhoveli žiadnemu statemenentu. Tieto výsledky získame registrovaním implementácie rozhrania UnmatchedListener. 
		
		\subsubsection{EP Runtime}
		EPRuntime rozhranie slúži na odosielanie nových udalostí do Esperu k spracovaniu, nastavenie a prístup k hodnotám premenných a spúšťanie on-demand EPL výrazov. Na odosielanie nových udalostí slúži metóda sendEvent, ktorá je preťažená. Typ parametra tejto metódy indikuje typ udalosti odosielanej do Esperu. Tieto typy boli bližšie popísané v predchádzajúcich sekciách.
		
		V prípade použitia XML definície typov udalostí sa pri spracovaní prichádzajúcej udalosti skontroluje že meno koreňového elementu prichádzajúcej udalosti je zhodné s menom typu udalosti definovanej XML schémou.
		
		Ak aplikácia nepozná EPL výrazy dopredu alebo nevyžaduje streamovanie výsledkov, je možné prostredníctvom EPRuntime spúšťať jednorázové výrazy. Tieto nie sú permanentné, po ich vykonaní je výsledok okamžite predaný aplikácii na spracovanie. Použitie nachádzajú napríklad v spojení s pomenovanými oknami a tabuľkami, ktoré je možné indexovať pre zrýchlenie prístupu.		

\section{Cassandra}
	Cassandra je databázový projekt, ktorý pôvodne vznikol vo firme Facebook. Neskôr bol zverejnený ako open-source a v roku 2009 bol prijatý do Apache inkubátora. V roku 2010 získal top prioritu a je naďalej vyvíjaný a dostupný pod Apache 2.0 licenciou \cite{what-is-Cassandra}.
	
	Cassandra je distribuovaná databáza, ktorá umožňuje spracovanie a uchovávanie veľkého množstva dát rozložených na veľkom počte menej výkonných serverov, ako je to znázornené na obrázku \ref{image_intro_Cassandra}. Táto architektúra zároveň poskytuje vysokú dostupnosť dát pri zabezpečení proti strate dát pri výpadku niektorého zo serverov. Cassandra je navrhnutá na použitie veľkého počtu počítačov (v ráde stoviek) podľa možností rozložených v rôznych častiach sveta.

	\myFigure{intro_Cassandra}{Škálovanie databázy Cassandra s rastúcou záťažou \cite{img:scaling}}{Škálovanie databázy Cassandra s rastúcou záťažou}
	
	Cassandra bola navrhnutá pre beh na cenovo dostupnom hardware a podporuje rýchly zápis veľkého množstva dát pri zachovaní efektívnosti prístupu k nim. Týmto pomáha znižovať firemné náklady na hardware.
	
	Vďaka týmto vlastnostiam je Cassandra využívaná množstvom známych firiem, medzi ktoré patrí napríklad CERN, eBay, GitHub, Netflix, Twitter alebo Cisco. Veľké produkčné nasadenia obsahujú stovky TB dát v klastroch zložených zo stoviek serverov. Pri porovnaní výkonnosti s ostatnými NoSQL databázami  Cassandra získava výborné výkonnostné výsledky aj vďaka svojej jednoduchej architektúre.
	
%TODO too much
% porovnanie s alternatívami vidíme na obrázku \ref{image_Cassandra-performance}
	
%	\myFigure{Cassandra-performance}{Porovnanie výkonnosti cassandry s alternatívnymi NoSQL databázami}{Výkonnosť Cassandra databázy}
	
	Cassandra je dostupná z dvoch zdrojov, prvým z nich stránka projektu Apache Cassandra. Tá je základnou verziou a obsahuje databázový engine a cqlsh nástroj slúžiaci ako vývojový shell. Pre jeho spustenie je nutné mať nainštalované interpretátor jazyka python.
	
	Druhou je nadstavba tretej strany DataStax Cassandra, ktorá odlišuje komerčnú a nekomerčnú verziu. V nekomerčnej verzii nájdeme rovnako ako v Apache balíku databázový engine a cqlsh. Navyše na svojich stránkach DataStax poskytuje zdarma rozšírenia, ktoré zjednodušujú prácu s databázou, a to:
	
	\begin{description}
		\item[OpsCenter] je grafický nástroj na správu databázy. Poskytuje prehľadné rozhranie pre administrátorov a vývojárov v ktorom je možné vidieť jednotlivé časti klastru. Umožňuje monitorovať stav, aktuálnu zátaž, pridávať a odoberať servery do konfigurácie klastra, nastaviť zálohovanie či generovať štatistiky. Pomocou neho je možné prehľadne spravovať klastre zložené zo stoviek serverov.

		\item[DevCenter] Pre úpravu štruktúry a údajov databázy slúži nástroj DevCenter. Grafické rozhranie umožňuje po pripojení na databázový engine vytvárať a spúšťať dotazy v CQL jazyku. Pri vytváraní dotazov je automaticky kontrolovaná syntax a sú zvýraznené chyby s popisom. 
		Obsahuje tiež interaktívne pomôcky na vytvorenie keyspace alebo tabuliek, export výsledkov alebo ukladanie dotazov.
		Tento nástroj sa dá zjednodušene vnímať ako grafická verzia cqlsh.

		\item[Java driver] Pre použitie databázy v programe napísanom v jave je potrebný ovládač, ktorý je možné stiahnuť práve na stránkach DataStax. Nájdeme tu tiež ovládače pre ďalšie vývojové platformy.
	\end{description}
	
	Aj keď v mnohom pripomína Cassandra relačnú databázu, nepodporuje plne relačný model. Namiesto toho poskytuje klientom jednoduchší dátový model a prináša návod ako niektoré chýbajúce funkcie nahradiť.
	Nasledujúci text rozoberá niektoré základné odlišnosti cassandry a relačných databáz.
	\subsection{Keyspace}
	Keyspace je možné prirovnať k schéme relačnej databázy. Slúži ako kontajner pre dáta, ktoré zdieľajú určité vlastnosti. Pri vytvorení keyspace je nutné určiť spôsob replikácie a počet kópií dát. Tieto kópie slúžia na zachovanie dátovej integrity v prípade výpadku niektorého zo serverov. Syntax pre vytvorenie nového keyspace je zobrazená vo výpise \ref{lst:create-keyspace}.

	\begin{lstlisting}[label=lst:create-keyspace,caption=Syntax pre vytvorenie nového keyspace]
	CREATE KEYSPACE <identifier> WITH <properties>
	\end{lstlisting}
	
	Po vytvorení keyspace je možné v ňom vytvárať tabuľky, ktoré už obsahujú samotné dáta. Tabuľka patrí do keyspace podobne ako v relačnom pojatí je tabuľka obsiahnutá v databáze. 
	
	\myFigure{keyspace-colfamily}{Porovnanie štruktúry SQL a NoSQL databázy \cite{img:colfamily}}{Porovnanie štruktúry SQL a NoSQL databázy}
	
	V predchádzajúcich verziách CQL bolo možné sa stretnúť s skupinami stĺpcov (column family). Tento pojem bol používaný v spojení s dynamickým modelom databázy, kedy stĺpce nebolo nutné definovať, ich definíciou sa vynucoval typ obsahu. V najnovšej verzii CQL sa však od tohoto prístupu upustilo, a dnes v aktuálnej verzii nájdeme tabuľky podobne ako ich poznáme z relačných databáz. Klauzula COLUMN FAMILY ostala ako synonym klauzuly TABLE. Porovnanie štruktúry SQL a NoSQL databázy je na obrázku \ref{image_keyspace-colfamily}.
	
	\subsection{Primárny kľúč}
	Podobne ako v relačnej databáze musí mať každý riadok unikátny identifikátor - primárny kľúč. Ten môže byť tvorený jediným údajom, alebo môže byť zložený z viacerých stĺpcov. Naproti relačnej databáze má primárny kľúč aj dodatočný význam.
	
	\begin{lstlisting}[label=lst:cql-pk,caption=Tvorenie primárneho kľúča v CQL]
	create table thesis (
		key_one text,
		key_two int,
		key_three int,
		data text,
		PRIMARY KEY(key_one, key_two, key_three)
	);
	\end{lstlisting}
	
	Vo výpise \ref{lst:cql-pk} je zobrazené názorné vytvorenie tabuľky so zloženým primárnym kľúčom. Tento kľúč má dve zložky:
	\begin{description}
		\item[Kľúč partície] (Partition key) určuje na ktorých uzloch sa uložia dáta. Je tiež zodpovedný za distribúciu naprieč jednotlivé uzly. V príklade je to key\_one.
		\item[Zoskupovací kľúč] (Clustering key) určuje zoskupovanie dát, čo je proces, ktorý usporadúva dáta v rámci daného úseku. V príklade je použitý zložený kľúč pozostávajúci z key\_two a key\_three.
	\end{description}
	Keďže Cassandra je distribuovaná databáza, tieto kľúče slúžia k vyhľadaniu miesta, kde sú uložená vyhľadávané záznamy. A práve toto vyhľadávanie je veľmi ovplyvnené architektúrou databázy.
	
	Práve tu vzniká jedna z najväčších odlišností od relačných databáz, ktorých užívatelia sú zvyknutí filtrovať údaje podľa ľubovoľného stĺpca tabuľky. Vyhľadávanie v klauzule WHERE je možné len podľa stĺpcov z ktorých sa primárny kľúč skladá a to v poradí v ktorom sú definované. Vo výpise \ref{lst:cql-pk-use} vidíme možnosti filtrovania v dotazoch do tabuľky definovanej výpisom \ref{lst:cql-pk}.
	
	\begin{lstlisting}[label=lst:cql-pk-use,caption=Príklad vyhľadávanie v tabuľke thesis]
Príklad validných dotazov:
	select * from thesis where key_one='kluc' and key_two=11
	select * from thesis where key_one='kluc' and key_two=11 and key_three=13
		
Príklad nevalidných dotazov:
	select * from thesis where key_two=11
	select * from thesis where key_three=11 and key_one='kluc'
	\end{lstlisting}
	
	Kvôli tomuto obmedzeniu je nutné prispôsobovať návrh tabuliek dotazom, ktoré ich budú využívať. To nutne vedie k duplicitám dát vo viacerých tabuľkách. Toto je ďalšia dôležitá odlišnosť a jeden z problémov v zmene myslenia pri prechode zo sveta relačných databáz, kde je správnym prístupom normalizácia dát. Pre dnešné dátové úložiská však nepredstavuje taký problém množstvo dát ako rýchlosť prístupu k nim. Preto sú duplicitné dáta prijateľným kompromisom v porovnaní s výhodami ktoré NoSQL databázy prinášajú.

	\subsection{Dátové typy}
	Ďalším z rozdielov oproti bežným relačným databázam sú dátové typy, ktoré je možné využívať. Nasledujúci zoznam zhŕňa tie, s ktorými sa v relačných databázach bežne nestretneme.
	\begin{description}
		\item[Kolekcie] Použitím dátového typu kolekcie môžeme definovať map, set a list. Pri definícii je nutné uviesť dátový typ, ktorý bude daná kolekcia obsahovať. Kolekcie sú vhodné na ukladanie relatívne malého množstva dát, napríklad telefónnych čísel daného užívateľa alebo popisov výrobku. 

		\item[TimeUUID] Typ UUID je používaný na predchádzanie kolíziám. Jeho rozšírenie TimeUUID obsahuje navyše časovú značku, čo je možné využiť v aplikáciách na vytvorenie jedinečného časového identifikátora.
		
		Na výpočet TimeUUID je použitá MAC adresa, časová značka a sekvenčné číslo. Z vygenerovaného TimeUUID je možné spätne získať časovú značku, takže funguje ako časový identifikátor, ktorý je zároveň jedinečný.
		
		\item[Tuple] umožňuje udržiavať stanovený počet vopred definovaných typov dát v jednom dátovom poli.
	\end{description}

	Výpis \ref{lst:dtypes} zobrazuje príklad použitia týchto dátových typov. V prvom kroku je vytvorená tabuľka incidentov, ktorá používa ako unikátny identifikátor riadku časovú značku a obsahuje dva stĺpce - data uchovávajúce informácie o incidente a notified, ktorý obsahuje zoznam osôb ktoré majú byť upozornené na incident.
	V druhom kroku je do tabuľky vložený vzorový incident.
	\begin{lstlisting}[label=lst:dtypes,caption=Príklad použitia dátových typov databázy Cassandra]
CREATE TABLE incident (
	tid timeuuid primary key,
	data <tuple<int, text, float>>,
	notified set<text>,
);

INSERT INTO incident (tid, data, notified) VALUES(
	now(),
	(31, 'Sunny', 77.5),
	{'f@baggins.com', 'baggins@gmail.com'}
);
	\end{lstlisting}

	\subsection{Offset, Join, Order, Count}
	V relačných databázach je samozrejmosťou výsledky radiť podľa potreby či spájať tabuľky. Cassandra má však rozdielnu architektúru, preto niektoré z typických klauzúl nenájdeme vôbec, prípadne je ich implementácia odlišná. Nasledujúci popis rozoberá hlavné rozdiely medzi Cassandrou a relačnoými databázami z pohľadu práce s týmito klauzulami.
	
	\begin{description}
	\item[OFFSET] Klauzuly limit a offset sú bežne používane v spojení so stránkovaním výsledkov vyhľadávania. Kvôli svojej distribuovanej architektúre však Cassandra neimplementuje klauzulu offset. Pri stránkovaní teda nejde jednoducho preskočiť na konkrétnu stranu. Implementácia teda zvyčajne spočíva v zobrazení výsledkov vo vzťahu ku konkrétnemu záznamu.

	\item[JOIN] Klauzula JOIN je nahradená ústupom od normalizácie. Údaje sú uchovávané v tabuľkách duplicitne a ich štruktúra je prispôsobená požiadavkám na operácie, ktoré s nimi budú vykonávané. Potrebné join sa teda musia plánovať už pri návrhu tabuľky, prípadne majú za následok vznik nových tabuliek.
	
	\item[ORDER BY] Klauzulu ORDER BY tiež nie je možné použiť ľubovoľne na každom definovanom stĺpci. V prípade že tabuľka definuje zoskupovací stĺpec, je možné ho použiť pre zoradenie výsledkov. V opačnom prípade je možné použiť len stĺpce definované v zoskupovacom kľúči.
	
	\item[COUNT] Klauzula COUNT je implementovaná rozdielne ako v relačných databázach. Operácia pre zápis v databáze Cassandra prebieha bez nutnosti čítania už existujúcich záznamov. To zvyšuje rýchlosť zápisu, avšak databáza neudržiava informáciu o aktuálnom počte záznamov. Operácia count preto musí pri zavolaní spočítať existujúce záznamy, čo je časovo náročná operácia. Preto sa táto klauzula zvyčajne používa spolu s klauzulou limit, ktorá zamedzí zbytočnému zahlteniu databázy \cite{cascount}.
	\end{description}
	
	Tieto rozdiely sú pri prechode z relačných databáz prekvapením, návrh databáz je totiž z veľkej časti obmedzujúci. Obmedzenia vyplývajú predovšetkým z architektúry dátového úložiska databázy, avšak práve táto architektúra ponúka aj mnoho výhod. Preto je použitie NoSQL databázy nutné dôkladne zvážiť a prispôsobiť potrebám konkrétnej aplikácie.
	
\section{Frameworky}
	Pre uľahčenie vývoja projektov je dnes bežné použitie frameworku. Rozsahom malé aplikácie často vyžadujú použitie funkcionality (prihlasovanie, odosielanie mailov, práca s databázou, zjednotenie grafického zobrazenia pre rôzne platformy, správu závislostí), ktorej implementácia je zdĺhavá a v málo prípadoch lepšia ako pri opätovnom použitý riešenia na to stavaného. Pri vývoji jednotlivých častí tohoto projektu boli použité voľne dostupné frameworky, ktorých krátkym popisom sa zaoberá táto sekcia.

	\subsection{Spring}
	Spring je framework napísaný v jave, distribuovaný pod Apache License verzie 2.0. Spring pozostáva z viacerých projektov zameraných na riešenie konkrétnych problémov vývoja aplikácie. V praktickej časti tejto aplikácie boli použité nasledujúce spring komponenty:
	\begin{description}
		\item[Spring Framework] svojou funkcionalitou rieši základné oblasti vývoja java aplikácií. Obsahuje základnú podporu pre injektovanie závislostí, správu transakcií, vývoj webových aplikácií alebo prístup k dátam. Táto funkcionalita je rozdelená do komponentov, z v praktickej časti sú použité spring-core, spring-jdbc a spring-webmvc.
		
		\item[Spring Boot] je spôsob ako urýchliť vývoj spring aplikácií. Rovnako ako Ruby On Rails sa prikláňa k prístupu convention over configuration. Z využitej funkcionality je dôležitý hlavne webový server Tomcat, ktorý tento komponent obsahuje. Vďaka nemu nie je nutné robiť deploy war súborov serverovej časti aplikácie, stačí jednoducho zdrojové kódy preložiť a spustiť. Je možné tiež využiť komponenty obsiahnuté v rodičovskom pom súbore. Spring boot automaticky nakonfiguruje spring aplikáciu bez nutnosti XML konfiguračných súborov s predvolenými nastaveniami.
	\end{description}
		
	\subsection{Ruby On Rails}
	Mottom frameworku Ruby On Rails je snaha o dosiahnutie spokojnosti programátora a udržateľnú produktivitu. Framework uprednostňuje prístup convention over configuration, čo znamená že konfigurácia je preddefinovaná a ak programátor používa predom dohodnuté konvencie postačujú minimálne úpravy na dosiahnutie požadovaného výsledku. Konfiguráciu je samozrejme možné podľa potreby upraviť. Framework je voľne dostupný a distribuovaný pod MIT licenciou.
	
	Vytvorenie a spustenie nového projektu pozostáva z jednoduchej série príkazov zobrazených vo výpsie \ref{lst:ror-create}. Všetky závislosti sú automaticky stiahnuté pri vytvorení projektu nástrojom bundler.
	\begin{lstlisting}[label=lst:ror-create,caption=Príklad vytvorenia a spustenia projektu v Ruby On Rails]
		rails new myweb
		cd myweb
		rails s
	\end{lstlisting}

	Adresárová štruktúra projektu je prispôsobená MVC architektúre. Jednotlivé komponenty nájdeme v adresároch model, view a controller.
	\begin{description}
		\item[Model] je vrstva reprezentujúca údaje a ich logiku. Umožňuje definovať objekty, ktorých dáta vyžadujú uloženie v databáze. Vlastnosti týchto objektov sú mapované na relačné dáta. Model je možné definovať pomocou migrácií, ktoré po spustení upravujú štruktúru databázy a umožňujú rollback k predchádzajúcemu stavu. V modeloch je možné definovať kontroly vstupných dát, asociácie, funkcie na rozšírenie prístupu k údajom alebo spätné volania v závislosti na vykonávanej akcii.
		\item[View] adresár je ďalej rozdelený do podadresárov reprezentujúcich jednotlivé controllery. Dôležitým je tiež adresár layout, ktorý ako názov napovedá obsahuje jednotný layout aplikácie. Prípona .erb súborov umožňuje použitie ruby kódu. V tomto adresári sa stretneme tiež so systémom vkladania čiastkových elementov stránky príkazom partial. Adresár helpers umožňuje definovať funkcie použiteľné v erb súboroch, ktoré sprehľadňujú štruktúru kódu.
		\item[Controller] je zodpovedný za obsluhu požiadavku a vyprodukovanie odpovedi. Jeho úloha zvyčajne pozostáva z prijatia požiadavky, získania alebo uloženia dát do databázy, definovania premenných pre zobrazenia potrebného view súboru. Controller poskytuje prístup k request a response objektom a definuje premennú flash, ktorá nesie správu o úspechu alebo chybe danej akcie.
	\end{description}
	
	Konfigurácia projektu je rozdelená do troch súborov podľa aktuálneho prostredia - test, development a production. Rovnako je možné podľa prostredia definovať rôzne databázy v súbore database.yml.
	V konfigurácii framework nájdeme tiež súbory initializers, ktoré sú spustené pri štarte projektu a inicializujú jednotlivé komponenty a súbor routes obsahujúci smerovanie prichádzajúcich požiadavkov na jednotlivé controllery. Závislosti a použité komponenty sú definované v súbore Gemfile.
	
	\subsection{Sinatra}
	Sinatra je jazyk použiteľný pre rýchly vývoj jednoduchých webových aplikácií postavených na ruby. Framework zapuzdruje jednoduchý webový server. Umožňuje definovať takzvané route, predstavujúce URL, ktoré aplikácia rozoznáva. Obsahom bloku definujúceho route je telo metódy, ktorá sa má vykonať pri zavolaní danej URL. Tieto bloky akceptujú vstupné parametre, predstavujúce GET a POST parametre. Príklad takejto routy zobrazuje nasledujúci výpis.
	\begin{lstlisting}[label=lst:sinatra-sampel,caption=Príklad definovania GET route vo frameworku Sinatra]
	get '/hello/:name' do |n|
		"Ahoj #{n}!"
	end
	\end{lstlisting}
	V tejto aplikácii bola Sinatra použitá pri demonštrácii jedného z usecase - webu kontaktov. Celá logika aplikácie pozostáva vďaka použitiu tohoto frameworku z 32 riadkov.

	\subsection{Bootstrap}
	Bootstrap framework je najpopulárnejším HTML a CSS frameworkom pre vývoj webových projektov. Umožňuje rýchly a jednoduchý vývoj front-end aplikácie. Je dostupný vo forme css súborov ako aj sass pre jednoduché použitie v rails projektoch. Distribuovaný je pod MIT licenciou.
	
	\myFigure{bootstrap-grid}{Ukážka Bootstrap grid systému \cite{bootstrap}}{Bootstrap grid systém}

	Bootstrap poskytuje triedy upravujúce zobrazovanie základných HTML komponentov. Aplikovaný je pomocou premennej class na jednotlivých komponentoch. Distribúcia tiež zahŕňa sadu ikon, písem a javascript, ktorý je zodpovedný napríklad za zobrazenie vyskakovacích okien alebo pomocných textov. 
	
	\begin{lstlisting}[label=lst:bootstrap-grid,caption=Príklad použitia Bootstrap grid systému \cite{bootstrap}]
	<div class="row">
		<div class="col-md-8">.col-md-8</div>
		<div class="col-md-4">.col-md-4</div>
	</div>
	<div class="row">
		<div class="col-md-4">.col-md-4</div>
		<div class="col-md-4">.col-md-4</div>
		<div class="col-md-4">.col-md-4</div>
	</div>
	\end{lstlisting}
	
	Jednou z jeho základných súčastí je mriežkový systém rozloženia stránky. Pred jeho použitím je potrebné pridať elementom, ktoré obaľujú stránku triedu container.	Mriežkový systém poskytuje responzívne rozhranie pozostávajúce z riadka obsahujúceho 12 stĺpcov, ktoré sa prispôsobujú podľa veľkosti obrazovky cieľového zariadenia. Tento systém umožňuje stránku rozdeliť na sekcie, ktoré je možné odsadiť, zarovnať, alebo rovnomerne rozložiť podľa potreby. Každý stĺpec funguje ako samostatná jednotka, ktorá môže obsahovať nový riadok s 12 stĺpcami. Stĺpce je možné spájať so skupín. Príklad použitia takéhoto rozloženia je vo výpise \ref{lst:bootstrap-grid}. Rozloženie stránky vytvorenej týmto kódom zobrazuje obrázok \ref{image_bootstrap-grid}.
					
	V tejto aplikácii bol bootstrap použitý pri vytváraní front-end administračného rozhrania ThesisWeb a usecase webu klientov.
	
	Využité frameworky a nástroje značne uľahčili vývoj projektu a v kapitole \ref{chap:instalacia} je bližšie popísaný spôsob ich inštalácie a použitia. Všetky použité nástroje sú voľne dostupné.
	
\section{Split-brain}
Pojem split-brain neoznačuje priamo technológiu, definuje však problém, ktorý z jej využitia vyplýva. Preto popíšem čo sa pod názvom split-brain skrýva a zmienim metódy, ktoré tento problém riešia.

Po prvotnom zoznámení s funkcionalitou Pacemakeru a iných HA riešení môže vzniknúť otázka, či je to naozaj také jednoduché. Veď len spúšťa, zastavuje a kontroluje procesy podľa vopred nakonfigurovaných pravidiel. Pri realizácii ale narazíme na problém s názvon split-brain. To je situácia, kedy sa kvôli výpadku komunikácie klaster rozdelí na 2 alebo viacero častí, ktoré začnú fungovať nezávisle na sebe. Obe časti sa pritom domnievajú, že práve tá druhá je nefunkčná. V lepšom prípade sa nič nestane, v horšom ostane disk plný poškodených dát.

Jadro problému spočíva v neschopnosti rozlíšiť či je počítač vypnutý alebo len prestal komunikovať so svojim okolím. Aký je v tom rozdiel? Predstavme si nasledujúcu situáciu \cite{web:split-brain-quo}:

Máme 3 počítače, Adam, Eva a Tom zapojené v klastri. Tom má pripojené vzdialené úložisko dát (diskové pole). Zrazu ale Tom prestane fungovať a nedá sa naňho pripojiť. Môžme bezpečne pripojiť diskové pole na Adama a pokračovať v prevádzke? Čo ak je Tom stále zapnutý a na disk zapisuje? Ocitneme sa v situácii kedy na disk zapisujú súčasne dva počítače, a to môže ľahko skončiť znehodnotením dát na ňom.

Jednoduchým riešením by bola redundancia komunikácie medzi nódmi, napríklad záložné wifi spojenie. Stále ale môže nastať situácia, kedy sa stane niečo nepredvídané a server prestane komunikovať. Preto je potrebné nastaviť fencing.

\subsection{Fencing}
Fencing je riešenie postavené na "`oplotení"' chybného nódu. To mu zabráni pristupovať k zdieľaným prostriedkom, v tomto prípade k disku. Vyriešil sa tak problém kedy bolo nemožné odpovedať na otázku či je počítač vypnutý alebo nedostupný - teraz na odpovedi nezáleží.

Riešenie pomocou odstrihnutia sa dá realizovať dvoma spôsobmi:
\begin{description}
	\item[Na úrovni zdrojov] Týmto prístupom zabránime chybnému počítaču, aby pristupoval k jednotlivým zdrojom, ktoré využíva. Napríklad ak by bol zdieľaný disk pripojený pomocou switchu, bolo by možné zakázať prístup k tomuto disku na úrovni switchu.
	\item[Na úrovni nódov] Pri tomto spôsobe nie je potrebné sa zaoberať tým, aké zdroje nód využíva, alebo ako mu k nim zabrániť prístup - zvyčajne ho jednoducho reštartujeme. Je to jednoduchšie, aj keď trochu drastickejšie riešenie, realizované pomocou techniky nazývanej \ac{STONITH}. 
\end{description}

Dôležitým znakom techniky oplotenia je, že k jeho realizácii nepotrebujeme akékoľvek informácie z chybného nódu, alebo jeho spoluprácu.

Pokračujem v príklade:
Adam a Eva použijú fencing a zabránia tak Tomovi aby zapisoval na zdieľaný disk. Adam si pripojí disk a môže pokračovať v bežnej prevádzke. Avšak čo zatiaľ robí Tom? Ak je stále zapnutý tak isto zistil že Adam a Eva sú nedostupní. Rovnako ako oni sa snaží oplotiť chybné prvky (z jeho pohľadu Adama a Evu). Ktorý z nich bude prvý? Týmto spôsobom je možné sa dostať až do cyklu, kedy sa nódy budú pri spustení navzájom reštartovať. Takéto nepredvídateľné správanie je nepriateľné, a tak je nutné využiť ďalšiu techniku - hlasovacie quorum.

\subsection{Quorum}
Quorum je technika, ktorou je možné zistiť, ktorá časť pôvodného klastra by mala ostať aktívna pri výpadku - je jej povolené zapínať služby. Najjednoduchšou technikou je vybrať tú skupinu, ktorá obsahuje viac počítačov.
Toto riešenie so sebou ale prináša jeden problém. Čo v prípade ak sa klaster skladá len z dvoch serverov? Pri výpadku komunikácie medzi nimi žiaden nemá väčšinu, takže žiaden nemôže spúšťať procesy. Nutnosť získať quorum sa síce v konfigurácii dá vypnúť, ale nedoporúča sa to. Existujú hardwarové aj softwarové metódy, ktoré tento problém riešia.

Metód je veľa, napríklad:
\begin{itemize}
	\item Hardwarovou metódou môže byť vyhradenie partície na externom disku dostupnej obom nódom. Ktorý nód dokáže túto partíciu pripojiť, ten sa považeje za funkčný
	\item Softwarovým riešením môže byť quorum démon. Existuje viacero implementácií, napríklad od Linux-HA. Funguje podobne ako prvá metóda, ale prináša niektoré výhody. Napríklad dokáže spoľahlivo fungovať pri väčších geografických vzdialenostiach
	\item Riešením môže byť aj uprednostnenie toho nódu, ktorý dokáže komunikovať s IP adresou na vonkajšej sieti. Je to síce najjednoduchšie, ale v prípade že zlyhá komunikácia medzi dvoma nódmi a pripojenie na internet ostane funkčné problém ostáva nevyriešený
\end{itemize}

V mojom príklade Adam a Eva tvoria väčšinu, získali quorum a môžu spúšťať procesy. Tom aj bez toho aby komunikoval vie, že má len 1 hlas z 3, takže (podľa konfigurácie) môže byť reštartovaný, prípadne počkať na zásah správcu \cite{web:split-brain-quo}.

\subsection{Stonith}
Pod touto skratkou vystupuje zariadenie, fungujúce presne podľa svojho názvu - jeho úlohou je zastreliť druhý nód. To znamená, že ho čo najrýchlejšie vypne alebo odpojí od zdrojov tak, aby nemohol vplývať na funkčnosť zvyšku klastra. Odstrelený nód sa totiž po reštarte dostane v rámci klastra do submisívneho postavenia a tým mu zabránime replikovať poškodené dáta na ostatné nódy \cite{web:felk.cvut.cz}.

Realizovať \ac{STONITH} môžme viacerými spôsobmi, z ktorých je dobré vybrať ten čo najmenej závisiaci na zvyšku systému. Delia sa do 5 základných kategórií \cite{web:opensuse.org}:

\begin{enumerate}
	\item UPS (Uninterruptible Power Supply) Poskytujú kontrolu záťaže a monitorovanie zariadení. Taktiež umožňujú individuálnu kontrolu napájania jednotlivých serverov
	\item PDU (Power Distribution Unit) Zdroj napájania cez ktorý sú pripojené ostatné zariadenia. V prípade výpadku zabezpečuje dočasnú dodávku energie
	\item Blade power control zariadenia sú vhodným riešením ak je klaster postavený na niekoľkých blade servroch. Musí ale byť schopný ovládať jednotlivé počítače
	\item Lights-out zariadenia sú menej kvalitné ako UPS, pretože zdieľajú zdroj napájania so svojím hostiteľom
	\item Testovacie zariadenia sú zvyčajne viac zhovievavé k hardwaru, vypínajú počítač "`jemnejšie"'. Mali by sa však využívať len pre testovacie účely. V produkčnom prostredí ich nahradzujú skutočné STONITH zariadenia.
\end{enumerate}

Väčšina pluginov umožňuje reštartovať alebo vypnúť chybný nód. Nie vždy je však vhodné štartovať CRM pri štarte OS, pretože je možné dostať sa do stavu kedy nód naštartuje a začne plne fungovať bez toho, aby sme mali šancu diagnostikovať čo bolo príčinou výpadku.
\chapter{Možné riešenia}
\label{chap:mozne_riesenia}

%http://advosys.ca/viewpoints/2007/01/linux-high-availability-clusters/
Spôsobov ako postaviť vysoko dostupné riešenie je veľa, dostupných zdarma aj komerčných. V tejto kapitole sa budem venovať len niektorým z nich, zameriam sa hlavne na tie, ktoré som v praktickej časti využil. V práci sa nesústredím na výber toho najvhodnejšieho riešenia, preto niektoré z nich zmienim len okrajovo.

\section{Linux-HA}
Linux High-Availability je nekomerčným projektom. Pozostáva z viacerých komponentov, z ktorých každý zabezpečuje inú časť funkcionality. Najhlavnejšími z nich sú \cite{pdf:approaches-for-ha}:

\begin{itemize}
	\item démon zabezpečujúci komunikáciu serverov v klastri na sieťovej úrovni
	\item manažér zodpovedný za spúšťanie skriptov a kontrolu behu jednotlivých služieb
	\item sada skriptov slúžiaca na obsluhu jednotlivých služieb, napríklad pripájanie diskov a nastavenie sieťových rozhraní. Práve jeden z týchto skriptov používam v praktickej časti
	\item databáza udržiavajúca konfiguráciu, ktorá je upravovaná na to určenými nástrojmi. Tieto nástroje tiež kontrolujú syntaktickú správnosť konfigurácie
\end{itemize}

Kedysi bol distrubovaný ako kompletné riešenie, časom sa však jednotlivé komponenty oddelili a je ich možné využiť aj v kombinácii s inými nástrojmi. V praktickej časti som použil manažéra procesov, ktorý vznikol vďaka tomuto projektu a ovládacie skripty.

\section{Red Hat Cluster Suite}
RHCS, najnovšie premenovaný na High Availability Add-On pokrýva dva rôzne kategórie vysokej dostupnosti, failover a IP load-balancing (pôvodne nazývaný Piranha). Písmeno "`R"' v RHCS naznačuje, že je dostupný len v RHEL, avšak nie je to tak. Balík je dostupný napríklad v CentOSe (prakticky RHEL bez licencie), Fedore alebo Ubuntu.

Je rovnako ako Linux-HA zložený z komponent, ktoré sa podľa zamerania dajú rozdeliť do štyroch celkov, zabezpečujúcich infraštruktúru, manažment služieb, administráciu a load-balancing \cite{web:rhcs-dokumentacia}. Avšak predpoklad, že Red Hat všetky tieto komponenty sám vyvíja sa ukázal ako chybný. Niektoré z nich totižto využívajú open-source produkty, napríklad Corosync, LVS a plánuje aj prechod na Pacemaker. Pri inštalácii GFS2 v praktickej časti sa ako jedna zo závislostí preberá napríklad CMAN, ktorý je komponentou RHCS.

\section{Linux Virtual Server}
Linux Virtual Server pristupuje k problému vysokej dostupnosti iným spôsobom ako Linux-HA a RHCS. Nerieši ho na úrovni jednotlivých serverov v klastri, namiesto toho využíva load-balancer, na ktorý sa klienti pripájajú. Ten rozdeľuje požiadavky medzi servery. Load-balancer má tiež za úlohu periodicky kontrolovať dostupnosť serverov, na ktoré požiadavky posiela a v prípade že niektorý z nich neodpovedá tak ho prestane používať. Keď server opäť začne odpovedať začne ho opäť používať. Takýmto spôsobom efektívne maskuje nedostupnosť serverov. Tiež z pohľadu administrátora je jednoduché pridať ďalší server do klastra bez nutnosti reštartovať celý systém \cite{web:linux-virtual-server}.

\section{Alternatívy}
Ako som načrtol v úvode kapitoly, neexistuje jedno univerzálne riešenie, je ich veľa a líšia sa ako cenou tak určením. Za zmienku určite stoja:

\begin{description}
	\item[Solaris MC] je operačný systém pre počítače v klastri. Umožňuje skupine serverov vystupovať ako jeden výkonný počítač. Známy je jednoduchou administráciou. Software zabezpečujúci jeho vysokú dostupnosť sa nazýva Sun Cluster
	\item[TurboLinux High Availability Cluster] obsahuje viacero komponentov zabezpečujúcich load-balancing, škálovateľnosť a vysokú dostupnosť. Virtualizuje viacero nezávislých serverov do spoločnej siete vystupujúcej pod jednou IP adresou
	\item[Steeleye Lifekeeper] umožňuje klientom použitie vlastných skriptov, ktoré budú kompatibilné s Microsoft Cluster Suite. To však prichádza s obmedzením použitia buď Lifekeeperu alebo MCS, nie oboch zároveň
	\item[Veritas Cluster Server] je produkt Symantecu fungujúci na Linuxe aj Windowse umožňujúci failover v prípade výpadku
	\item[Microsoft Cluster Service] nájdeme v serverovej edícii Windows. Tam je jedným z troch komponentov zabezpečujúcich klastrovanie. Ďalšími sú Network Load Balancing a Component Load Balancing
	\item[UCARP] je jednoduchý nástroj umožňujúci niekoľkým hostiteľom zdieľať IP adresu za účelom automatického failoveru
\end{description}

Ďalšie podobné produkty zahŕňajú napríklad Fujitsu PrimeCluster, HP Serviceguard, IBM PowerHA, NEC ExpressCluster, Oracle Clusterware alebo SUSE Linux Enterprise HA Extension.

\emptydoublepage
\chapter{Realizácia}
Pôvodný návrh tejto práce počítal s realizáciou a testovaním celého riešenia na reálnych serveroch. Tie sa mi ale v čase písania práce nepodarilo zabezpečiť, takže praktická časť prebehla na virtuálnych serveroch. Tým sú podstatne ovplyvnené niektoré z testov, avšak verím, že aj pomocou tohoto dokumentu bude možné riešenie veľmi jednoducho realizovať a otestovať aj v reálnom prostredí.

Virtualizačné riešenie, ktoré som použil je trial verzia VMware Workstation. Vybral som ho z dôvodu podpory viacerých snapshotov a dobrou integráciou s Windows 7, ako aj preto že som ho v minulosti používal a mám s ním dobré skúsenosti.

V ňom som vytvoril 2 virtuálne stroje Adama a Evu. Ich hardwarová konfigurácia je rovnaká, avšak v reálnom nasadení to nie je vyžadované. Jednotlivé komponenty nájdeme v tabuľke \ref{table:vmware-parametre}.

\myTable{
\begin{tabular}{ | l | l | }
	\hline
	Komponent & Parametre \\
	\hline
  Operačná pamäť	& 1 GB  \\ \hline
  Pevný disk			& 20 GB SCSI  \\ \hline
  Procesor				& 1 jednojadrový procesor \\ \hline
	Sieťová karta 1	& NAT pripojený na internet \\ \hline
	Sieťová karta 2 & Host-only rozhranie \\ \hline
	CD/DVD					& Pripojený iso obraz OS \\
	\hline
\end{tabular}
}{Parametre virtuálnych strojov}{table:vmware-parametre}

Ako je už v úvode spomenuté, pokúšam sa nakonfigurovať riešenie pre malé firmy alebo domáce použitie, takže je nevyhnutné brať na zreteľ cenu. Preto som vyberal technológie, ktoré sú dostupné zdarma. Hardwarové nároky použitého systému sú tiež minimálne. Schému celého riešenia, ktoré nakonfigurujem vidíme na obrázku \ref{image_klaster-schema}. 

\myFigure{klaster-schema}{Schéma finálneho riešenia klastra} {Schéma klastra}

\section{Operačný systém} % ====================  ====================
Prvé kolo rozhodovania bolo relatívne ľahké. Windows alebo Linux? Windows je síce rozšíreným operačným systémom, ale je licencovaný. Taktiež riešenia dostupné zadarmo sú väčšinou produkované komunitou, ktorá je sústredenejšia okolo Linuxu.

Druhým kolom bolo vybrať tú správnu distribúciu Linuxu. Existuje ich ale veľké množstvo, tak ktorá je tá najlepšia? Každá distribúcia má svoje pre a proti, ja som si vybral serverovú edíciu Ubuntu z nasledujúcich dôvodov:

\begin{enumerate}
	\item Ubuntu poznám. To síce nijako neopodstatňuje jeho použitie z profesionálneho hľadiska, avšak verím, že väčšina administrátorov uvažuje podobne. Prečo inštalovať neznáme riešenie na neznámy OS? Tiež v prípade, že firma už pre servery využíva konkrétny OS, nie je zvykom stažovať administrátorom prácu udržiavaním rozličných OS.
	\item Ubuntu server sa vyvíja veľmi rýchlo, nové verzie sú vydávané každých 6 mesiacov. Jedným z cieľov práce je otestovať reálne riešenie, tak prečo nie práve na menej konzervatívnom systéme
	\item Ubuntu je postavené na Debiane, ktorý obsahuje obrovské množstvo predkompilovaných balíčkov. Ubuntu túto základňu ešte viac rozširuje a novšie aplikácie sú k dispozícii oveľa skôr ako v prípade niektorých iných distribúcií
	\item Ubuntu poskytuje mimo komunitných fór aj platenú podporu. Rovnako platenú podporu poskytuje napríklad Red Hat (podpora je na fórach hodnotená dosť slabo), avšak možno práve fakt, že Ubuntu je menej komerčný bude znamenať že táto podpora bude kvalitnejšia.
\end{enumerate}

Mnou použité riešenie ale obdobne funguje aj na ostatných distribúciách. Najväčšie rozdiely budú pozorovateľné pravdepodobne pri samotnej inštalácii balíčkov a následnom hľadaní konfiguračných súborov. S výnimkou vlastného rozdelenia disku som pri inštalácii OS použil prednastavé hodnoty.

Vybral som si aktuálnu verziu Ubuntu 12.04. Všetok software som inštaloval pomocou správcu balíčkov apt zo štandardných repozitárov.

\section{Dostupnosť dát} % ====================  ====================

Počínajúc touto kapitolou sa začnem prakticky venovať vybraným technológiám z predchádzajúceho textu. Použitie RAIDu pre diskové pole je, ako som v sekcii \ref{lbl:sec:raid} popísal, prvým krokom k dosiahnutiu vyššieho zabezpečenia dát. V prípade výpadku tak nie je nutné odstaviť celý server, stačí vymeniť chybný disk. Vo virtuálnych servroch som ale túto vrstvu vynechal, pretože sa chcem zaoberať predovšetkým vysokou dostupnosťou s ohľadom na prevenciu výpadkov akéhokoľvek komponentu, nie len pevného disku.

Použité 20 GB disky som rozdelil čo najjednoduchšie. Časť z nich som nechal nevyužitú kvôli možnostiam ďalšieho testovania. Jeho presné rozdelenie ukazuje tabuľka \ref{table:rozdelenie-disku}. Disky na oboch serveroch sú rozdelené rovnako. Pri reálnom nasadení nie je nutné rovnaké rozdelenie disku, avšak partície vyhradené pre DRBD musia mať rovnakú veľkosť.

\myTable{
\begin{tabular}{ | l | l | c | c | c | c | }
	\hline
	Partícia & Bod pripojenia 			& FS 	 & Veľkosť & Typ & Využitie \\ \hline
  sda1 		 & \textbackslash root	& ext4 & 7 GB 	 & primárna & OS \\ \hline
  sda5 		 & swap					 				& swap & 1 GB 	 & logická & swap \\ \hline
  sda6 		 & nevyužitá			 			&  -	 & 5 GB 	 & logická & testy FS s DRBD \\ \hline
	sda7 		 & nevyužitá						&  -   & 5 GB 	 & logická & testy FS bez DRBD \\ \hline
	-		 		 & voľné miesto	 				&  -   & 2 GB 	 & - & rezerva \\
	\hline
\end{tabular}
}{Tabuľka rozdelenia disku}{table:rozdelenie-disku}

Partície sda6 som použil ako podklad pre DRBD zariadenie. Jeho konfiguráciu popíšem v nasledujúcej kapitole.

\subsection{DRBD}
Inštalácia nástrojov potrebných pre jeho správu prebehla bez problémov. Pri prvom použití bolo treba načítať modul jadra s názvom drbd. V niektorých distribúciách je potrebné tento modul nainštalovať, v mojom prípade ho už jadro obsahuje. Konfigurácia sa delí na 2 časti, globálnu a konfiguráciu samotnej DRBD partície.

\begin{description}
	\item[Globálna] časť umožňuje definovať správanie DRBD aplikácie. Definujeme tu napríklad požadované reakcie v prípade výpadku niektorého z diskov, timeouty, rýchlostné limity alebo protokol (synchrónny, asynchrónny) ktorý chceme použiť.
	\item[Partície] definujeme tak, že špecifikujeme podkladové partície, ktoré má DRBD použiť, adresy servrov, na ktorých sa nachádzajú a názov zariadenia, ktoré má vytvoriť. Takýchto partícií môžeme definovať viacero.
\end{description}

Konfiguračné súbory sú na oboch servroch rovnaké. DRBD nezrkadlí konkrétne súbory a priečinky, pracuje len na úrovni blokového zariadenia ako je popísané v kapitole \ref{lbl:sec:zdielane-blokove-zariadenie}. Aby som mohol toto zariadenie využiť, musel som na ňom vytvoriť súborový systém. Práve jeho výberu sa budem venovať v ďalšej kapitole. Pre zrkadlenie dát som použil samostatnú sieťovú kartu, pretože aj keď sú výsledky testov skreslené v dôsledku virtualizácie, v reálnom nasadení je to odporúčané. Konfiguračný súbor je zobrazený vo výpise \ref{lst:drbd}.

\begin{lstlisting}[label=lst:drbd,caption=Konfiguračný súbor DRBD zariadenia]
	resource r0 {
		device    /dev/drbd0;
		disk      /dev/sda6;
		meta-disk internal;
		
		on adam {
			address   10.1.1.11:7789;
		}
		on eva {
			address   10.1.1.12:7789;
		}
	}
\end{lstlisting}

Prednastavený limit maximálnej rýchlosti pre synchronizáciu je vhodné upraviť pomocou konfiguračnej položky "`rate"'. Obmedzenie rýchlosti je využiteľné najmä v prípade, kedy existuje riziko zahltenia sieťovej karty pri inicializácii. Ďalšiu replikáciu dát tento limit neovplyvňuje.

\subsection{Súborový systém}
%http://pommi.nethuis.nl/bonnie-to-google-chart/
%http://unix.stackexchange.com/questions/165/what-are-the-disadvantages-of-ext4-reiserfs-jfs-and-xfs
%http://www.abclinuxu.cz/zpravicky/benchmark-souboroveho-systemu-s-bonnieplusplus
%popis vystupu - http://www.textuality.com/bonnie/advice.html

Súborových systémov je veľa, preto som sa zameral na tie najpoužívanejšie. Medzi testované som zahrnul ext3, ext4, reiserfs, zfs, jfs a xfs. Zo systémov so zdieľaným diskom som chcel v testoch zahrnúť OCFS2 a GFS2. GFS2 sa mi však v mojej konfigurácii nepodarilo spustiť kôli problémom v komunikácii klastra, spôsobenými pravdepodobne chybou medzi vrstvami Corosync a CMAN. Porovnanie ich výkonnosti je možné nájsť napríklad v dokumente \cite{pdf:filesystem-comparison}.

Na testovanie súborových systémov som si vybral voľne dostupný nástroj bonnie++ verzie 1.96 a na prevod do grafickej podoby php skript bonnie2gchart. Test pozostával z dvoch častí, testu rýchlosti zápisu a čítania dát a testu počtu operácií s metadátami súborov, ktoré sa vykonajú za jednu sekundu. Tieto testy majú rôzne praktické využitie:

\begin{description}
	\item[I/O Dáta] Tento test je dôležitý - ak chceme súborový systém použiť na prácu s menším množstvom veľkých súborov. Vhodné využitie je napríklad pre ftp server. Test prepisovania je dôležitý v prípade, že nami využívané aplikácie často upravujú už existujúce dáta.
	\item[Metadáta] Rýchlosť práce s metadátami je dôležitá v prípade práce s menšími súbormi. Pri rozbaľovaní archívu s veľkosťou 10 MB, ktorý obsahuje stovky súborov bude rýchlosť práce s metadátami zaujímavejšia ako rýchlosť zápisu na disk. Test je vhodný napríklad pre mailové servery, tmp partíciu alebo squid proxy.
\end{description}

Do testovania som pre zaujímavosť zahrnul aj ntfs partíciu, ktorá je štandardom na windowsoch. Testy práce s dátami dopadli vyrovnane ako je vidno na obrázku \ref{image_testing-fs-io}. Pri testovaní som sa snažil nezaťažovať hostiteľský OS nepotrebnými aplikáciami, avšak pri opätovnom spustení sa výsledky toho istého testu mierne líšili. Odchýlky však boli malé, na grafe sú znázornené výsledky jedného testu.

\myFigure{testing-fs-io}{Testy rýchlosti s použitím DRBD} {Testy rýchlosti s DRBD}

Pri výbere súborového systému zameraného na prácu s malými súbormi sa ako najvhodnejší kandidáti ukázali ext3, ext4 a reiserfs. Keďže ext4 je nasledovníkom ext3 a budúcnosť reiserfs bola istú dobu neistá, mojou voľbou by bol ext4. Pri vytváraní súborových systémov ma mierne zarazil fakt, že len niektoré z nich (xfs a jfs) vyžadujú potvrdenie pri prepísaní partície s už existujúcim súborovým systémom. Pritom malou chybou v čísle partície (sda6 vs sda7) pri jeho vytváraní môže administrátor zmazať všetky údaje uložené na danej partícii. NTFS nástroje partíciu dokonca bez varovania prepíšu nulami.

\myFigure{testing-fs-metadata}{Testy rýchlosti práce s metadátami} {Testy rýchlosti metadát s DRBD}

Pre porovnanie rýchlosti som tie isté testy zopakoval bez použitia DRBD zariadenia. Zápis dát prebehol dva až tri krát rýchlejšie, rýchlosť čítania dát je porovnateľná. Presné výsledky sú znázornené na obrázku \ref{image_testing-fs-io-nodrbd}. Musím ale pripomenúť, že testovacím prostredím je VMware, ktoré využíva jediný fyzický disk hostiteľského systému. Toto obmedzenie pravdaže vyplýva z hardwarovej konfigurácie môjho počítača. Preto napríklad zápis dát pri replikácii pomocou DRBD musel prebehnúť 2 krát na tom istom disku.

\myFigure{testing-fs-io-nodrbd}{Testy rýchlosti bez použitia DRBD zariadenia}{Testy rýchlosti bez DRBD}

Výberom súborového systému a jeho inštaláciou na DRBD partíciu som dosiahol, že v prípade výpadku jedného zo strojov sú rovnaké dáta prístupné na druhom bez nutnosti obnovy zo zálohy alebo dodatočnej konfigurácie. Vytvoril som RAID-1 nezávislý na chybe v rozsahu servera. Vysoko dostupné dáta sú však bez aplikácií, ktoré ich sprístupnia užívateľom nepoužiteľné, preto v nasledujúcej časti predstavím konfiguráciu vysoko dostupného riešenia pre aplikácie.

\section{Dostupnosť aplikácií} % ====================  ====================
Pri realizácii tejto časti som sa rozhodoval, ktoré komponenty použiť. RGManager alebo Pacemaker? Corosync alebo Heartbeat? Zvolil som si cestu čo najjednoduchšieho riešenia s prihliadnutím na vyhliadky jednotlivých projektov. Corosync zabezpečuje časť funkcionality CMANu, potrebného pre RGManager. Pacemaker má časom ale RGManager nahradiť. Vybral som teda riešnie zostavené z čo najmenšieho počtu komponentov, ktoré vydržia čo najdlhšie. Corosync a Pacemaker.

\subsection{Kominukačná vrstva}
Pri inštalácii Corosyncu je potrebné vygenerovať zdieľaný kľúč, ktorý slúži na autentizáciu jednotlivých serverov a aby vedeli že do daného klastra patria. Jediným problémom, na ktorý som natrafil bola prednastavená hodnota "`start=no"' v jednom z konfiguračných súborov. Corosync kvôli nej pokusy o štart ignoroval bez výpisu akýchkoľvek dodatočných informácií.

\subsection{Manažér procesov}
Ako manažéra procesov som použil Pacemaker. Konfiguroval som ho pomocou nástroja príkazového riadku crm, ktorý poskytuje rozhranie k samotnému xml súboru, v ktorom sú nastavenia uložené. Rovnaký výsledok sa dá dosiahnúť použitím testovaných GUI zmienených v kapitole \ref{lbl:sec:gui}. Pre názornú ukážku som použil drbd v móde primárny/sekundárny a na ňom vytvoril súborový systém ext4. Na aktívnom nóde bude prístupná IP adresa, ktorú môže využívať ľubovoľná služba.

Pre samotnú konfiguráciu bolo nutné nastaviť niekoľko pravidiel, ktorých reálny zápis možno vidieť vo výpise \ref{lst:crm}. Konfigurácia sa skladala z pravidiel definujúcich:
\begin{description}
	\item[Primitívy] ktoré definujú jednotlivé služby. V tejto konfigurácii sú použité 3 primitívy pre DRBD, súborový systém a IP adresu
	\item[Kolokácie] definujú nutnosť spustenia služieb spoločne. V mojom prípade sú 2 a definujú že IP adresa môže byť spustená len na nóde s pripojeným súborovým systémom a ten bude pripojený vždy na primárnom nóde
	\item[Poradie] hovorí ako z názvu vyplýva o poradí spustenia služieb. Súborový systém nemôže byť pripojený skôr ako DRBD zariadenie
	\item[Priľnavosť] definuje ako veľmi chceme, aby služba ostala bežať na nóde, na ktorom je. V prípade že by som túto hodnotu nenastavil, služby by samovoľne migrovali podľa uváženia Pacemakeru
	\item[Vlastnosti] definujú všeobecné správanie klastra. Ja som ich použil na zrušenia vynucovania vlastností Stonith a Quorum, ktoré pre potreby ukážky nie sú nevyhnutné, avšak v produkčnom nasadení sa na ne nesmie zabudnúť
\end{description}

\begin{lstlisting}[label=lst:crm,caption=Čiastočný výpis konfigurácie crm]
root@eva:~# crm configure show
	Primitívy
		primitive ClusterIP ocf:heartbeat:IPaddr2 \
				params ip="192.168.45.101" cidr_netmask="24" \
				op monitor interval="30s"
		primitive DRBD ocf:linbit:drbd \
				params drbd_resource="r0"
		primitive fs_ext4 ocf:heartbeat:Filesystem \
				params device="/dev/drbd0" directory="/mnt" fstype="ext4" \
				meta target-role="Started"
	Kolokácie
		colocation drbd-with-ip inf: ClusterIP fs_ext4
		colocation fs_on_drbd inf: fs_ext4 msDRBDclone:Master
	Poradie
		order fs_ext4-after-DRBD inf: msDRBDclone:promote fs_ext4:start
	Vlastnosti
		property stonith-enabled="false" no-quorum-policy="ignore"
	Priľnavosť
		rsc_defaults %*\$*)id="rsc-options" resource-stickiness="100"
\end{lstlisting}

Konfigurácia sa pri zmene automaticky propaguje na ostatné nódy v klastri. Funkcionalitu riešenia v praxi názorne predvediem v ďalšej kapitole.

\section{Čo som vytvoril} % ====================  ====================
Výsledok práce predvediem na názornom príklade. Na začiatku tohoto testu sú oba servery online a všetky služby sú spustené. Test spočíva vo vypnutí primárneho serveru takzvane "`natvrdo"' pomocou VMware. Pomocou systémových nástrojov (df, grep, crm\textunderscore mon, cat) predvediem zmenu stavu častí systému, ktoré Pacemaker ovláda - DRBD disku, pripojenia súborového systému a spustenia IP adresy. Z výpisu niektorých príkazov som odstránil nedôležité detaily pre lepšiu prehľadnosť.

\subsection{Pred výpadkom}
Nástroj crm\textunderscore mon slúži na zobrazenie stavu jednotlivých služieb, jeho výpis je na oboch serveroch identický. Služby sú teraz spustené na Adamovi. Parameter -1 slúži na jednorázový výpis stavu klastra. Vo výpise je vidieť zoznam nakonfigurovaných primitívov.
\begin{lstlisting}[label=lst:done-crm-before]
root@adam:~# crm_mon -1
	Online: [ adam eva ]
		ClusterIP      (ocf::heartbeat:IPaddr2):       Started adam
		Master/Slave Set: msDRBDclone [DRBD]
			Masters: [ adam ]
			Slaves: [ eva ]
		fs_ext4        (ocf::heartbeat:Filesystem):    Started adam
\end{lstlisting}

Nasledujúce príkazy dokazujú, že Adam je primárnym serverom a je na ňom pripojená DRBD partícia, zatiaľ čo sekundárny server je neaktívny.
\begin{lstlisting}
root@adam:~# df -h | grep mnt
	/dev/drbd0      4.7G  198M  4.3G   5% /mnt
root@adam:~# cat /proc/drbd | grep cs
	0: cs:Connected ro:Primary/Secondary ds:UpToDate/UpToDate C r-----

root@eva:~# df -h | grep mnt
root@eva:~# cat /proc/drbd | grep cs
	0: cs:Connected ro:Secondary/Primary ds:UpToDate/UpToDate C r-----
\end{lstlisting}

\subsection{Po výpadku}
Po vypnutí primárneho serveru (Adam) sekundárny (Eva) detekuje jeho neprítomnosť a začne zapínať jednotlivé služby v nakonfigurovanom poradí. Po chvíli je z výpisu zrejmé, že sa všetky spustili na serveri Eva. Celý proces od detekcie po spustenie poslednej služby trval približne 5 sekúnd.
\begin{lstlisting}[label=lst:done-crm-after]
root@eva:~# crm_mon -1
	Online: [ eva ]
	OFFLINE: [ adam ]
		ClusterIP      (ocf::heartbeat:IPaddr2):       Started eva
		Master/Slave Set: msDRBDclone [DRBD]
			Masters: [ eva ]
			Stopped: [ DRBD:0 ]
		fs_ext4        (ocf::heartbeat:Filesystem):    Started eva
\end{lstlisting}

Keďže Adam je už vypnutý a Eva s ním nemá spojenie, je v DRBD výpise označený ako unknown. Pacemaker nastavil DRBD na Eve do stavu primary a pripojil súborový systém.
\begin{lstlisting}
root@eva:~# cat /proc/drbd | grep cs
	0: cs:WFConnection ro:Primary/Unknown ds:UpToDate/DUnknown C r-----
root@eva:~# df -h | grep mnt
/dev/drbd0      4.7G  198M  4.3G   5% /mnt
\end{lstlisting}

V prípade, že sa dáta na primárnom DRBD zariadení počas nedostupnosti druhého nódu zmenia, je po jeho opätovnom pripojení automaticky inicializovaná synchronizácia dát. Kopírujú sa len zmenená bloky, nie celé zariadenie ako pri inicializácii DRBD. Výpis znázorňuje priebeh synchronizácie.
\begin{lstlisting}
root@adam:/mnt# cat /proc/drbd
	0: cs:SyncSource ro:Primary/Secondary ds:UpToDate/Inconsistent C r-----
        [===========>........] sync'ed: 60.0% (8872/20188)K
        finish: 0:00:00 speed: 11,316 (11,316) K/sec
\end{lstlisting}


\emptydoublepage
\chapter*{Záver}
\addcontentsline{toc}{chapter}{\protect\numberline{}Záver}

kedze esper koncept nie je vseobecne znamy a pri spusteni som narazil na problemy - nazorna instalacia a priklady pouzitia


\emptydoublepage

% =========================================================
% ==================== Zoznam skratiek ====================
\begin{acronym}
	\acro{CEP}{Complex event processing}
	\acro{JDBC}{Java database connectivity technology}
	\acro{CQL}{Cassandra Query Language}
	%\acro{}{}
	%\acro{}{}
	%\acro{}{}
	%\acro{}{}
	%\acro{}{}
	%\acro{}{}
	%\acro{}{}
	%\acro{}{}
	%\acro{}{}
	%\acro{}{}
	%\acro{}{}
	%\acro{}{}
\end{acronym}


%--------------------------------------------------------------
\backmatter
%Bibliografia
%\phantomsection
%\nocite{*}
\bibliographystyle{ieeetr}
\bibliography{bibliografia}	\emptydoublepage

\listoffigures \emptydoublepage
\listoftables

\end{document}

% Dlhe url v bibliografii
%http://tex.stackexchange.com/questions/10924/underfull-hbox-in-bibliography
