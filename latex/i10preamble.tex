%-----------------------------------------------------------------------------------------------------------------------------------------
%\textwidth100mm
%\marginparsep10mm
%\marginparwidth30mm


%\newlength{\fullwidth} % Width of text plus margin notes
%\setlength{\fullwidth}{\textwidth}
%\addtolength{\fullwidth}{\marginparsep}
%\addtolength{\fullwidth}{\marginparwidth}

%----------------------------------------------------------------------------------
% \myBigFigure	[ LABEL_PREFIX (optional) ]
%				{ FILENAME (without extension) }
%				{ CAPTION TEXT }
%				{ SHORT VERSION OF CAPTION TEXT }
%
%picture using full width of the page
\newcommand{\myBigFigure}[4][image]
{
\begin{figure}[t!bp]
	\checkoddpage
	\ifcpoddpage
		%nothing
	\else
		\hspace{-\marginparsep}\hspace{-\marginparwidth}
	\fi
	%use minipage to center the label beneath the figure
	\begin{minipage}{\fullwidth}
		\includegraphics[width= \fullwidth]{#2}
		\caption[#4]{#3}
		\label{#1_#2}
	\end{minipage}
\end{figure}
}


%----------------------------------------------------------------------------------
% \myFrameBigFigure	[ LABEL_PREFIX (optional) ]
%					{ FILENAME (without extension) }
%					{ CAPTION TEXT }
%					{ SHORT VERSION OF CAPTION TEXT }
%
%picture with frame using the full width of the page
\newcommand{\myFrameBigFigure}[4][image]
{
\begin{figure}[t!bp]
	\checkoddpage
	\ifcpoddpage
		%nothing
	\else
		\hspace{-\marginparsep}\hspace{-\marginparwidth}
	\fi
	%use minipage to center the label beneath the figure
	\begin{minipage}{\fullwidth}
	\frame{%
		\includegraphics[width= \fullwidth]{#2}%
		}
		\caption[#4]{#3}
		\label{#1_#2}
	\end{minipage}
\end{figure}
}

%----------------------------------------------------------------------------------
% \myHUGEFigure	[ LABEL_PREFIX (optional) ]
%				{ FILENAME (without extension) }
%				{ CAPTION TEXT }
%				{ SHORT VERSION OF CAPTION TEXT }
%
%landscape picture using the full width of the rotated page
\newcommand{\myHugeFigure}[4][image]
{
\begin{sidewaysfigure}[t!bp]
	
		\includegraphics[width= \textheight]{#2}
		\caption[#4]{#3}
		\label{#1_#2}
	
\end{sidewaysfigure}
}

%----------------------------------------------------------------------------------
% \myFigure	[ LABEL_PREFIX (optional) ]
%			{ FILENAME (without extension) }
%			{ CAPTION TEXT }
%			{ SHORT VERSION OF CAPTION TEXT }
%
%picture using the width of the text column
\newcommand{\myFigure}[4][image]%
{%
\begin{figure}[ht!bp]%
	\begin{center}%
		\includegraphics[width= \textwidth]{#2}%
		\caption[#4]{#3}
		\label{#1_#2}%
	\end{center}%
\end{figure}%
}%

%----------------------------------------------------------------------------------
% \myImgRef	[ LABEL_PREFIX (optional) ]
%			{ LABEL OF THE IMAGE }
%
%reference to an image
\newcommand{\myImgRef}[2][image]%
{%
	\ref{#1_#2}%
}%

%----------------------------------------------------------------------------------
% \myBigTable	{ YOUR TABULAR DEFINITION }
%			{ CAPTION TEXT }
%			{ TABLE_LABLE }
%
%table using the full width of the page
\newcommand{\myBigTable}[3]%
{%
\begin{table}[htdp]%
	\checkoddpage%
	\ifcpoddpage%
		%nothing
	\else%
		\hspace{-\marginparsep}\hspace{-\marginparwidth}%
	\fi%
	\begin{minipage}{\fullwidth}%
		\begin{center}%
			#1%
			\caption{#2}%
			\label{#3}%
		\end{center}%	
	\end{minipage}%
\end{table}%
}%

%----------------------------------------------------------------------------------
% \myTable	{ YOUR TABULAR DEFINITION }
%			{ CAPTION TEXT }
%			{ TABLE_LABLE }
%
%table using the width of the text column
\newcommand{\myTable}[3]%
{%
\begin{table}[htdp]%
	\begin{center}%
		#1%
		\caption{#2}%
		\label{#3}%
	\end{center}%	
\end{table}%
}%

%----------------------------------------------------------------------------------
% \myTxtRef	{ LABLE }
%
%references chapters or sections, outputs number and title, e.g., 5.3---"Yaddahyaddah"
\newcommand{\myTxtRef}[1]
{%
	\ref{#1}---``\nameref{#1}''%
}

%----------------------------------------------------------------------------------
% \myUnderscore
%
%typesets a 'nice' underscore for URLs
\newcommand{\myUnderscore}{$\underline{\hspace{0.5em}}$}

%----------------------------------------------------------------------------------
%\myTilde
%
%typesets a 'nice' tilde for URLs
\newcommand{\myTilde}{$\sim$}

%----------------------------------------------------------------------------------
% \myURL	{ TYPESET VERSION OF ANCHOR }
%			{ PRISTINE URL }
%			{ TYPESET VERSION OF URL }
%
%typesets a URL
%the typographically correct version appears as a footnote,
%the anchor appears in the text, the link points to the pristine URL
\newcommand{\myURL}[3]%
{%
	\textcolor{blue}{%
		\href{#2}{#1}%
	}%
	\footnote{#3}
}

%----------------------------------------------------------------------------------
% \mySimpleURL	{ TYPESET VERSION OF ANCHOR }
%				{ PRISTINE URL }
%
%typesets a URL
%the URL appears as a footnote,
%the anchor appears in the text, the link points to the URL
\newcommand{\mySimpleURL}[2]%
{%
	\textcolor{blue}{%
		\href{#2}{#1}%
	}%
	\footnote{#2}
}

%----------------------------------------------------------------------------------
% \myProjectURL	{ TYPESET VERSION OF ANCHOR }
%				{ PRISTINE URL INSIDE PROJECT DIRECTORY }
%				{ TYPESET VERSION OF URL INSIDE PROJECT DIRECTORY }
%
%typesets a URL to hci/public from where the contents of the WebServer folder from oliver can be accessed
%the typographically correct version appears as a footnote,
%the anchor appears in the text, the link points to the pristine URL
\newcommand{\myProjectURL}[3]%
{%
	\textcolor{blue}{%
		\href{http://hci.rwth-aachen.de/public/#2}{#1}%
	}%
	\footnote{http://hci.rwth-aachen.de/public/#3}%
}

%----------------------------------------------------------------------------------
% \mnote	{ MARGIN NOTE }
%
%puts a comment into the margin in small sans-serif font
\newcommand{\mnote}[1]{\marginpar{\raggedright\textsf{{\footnotesize{#1}}}}}

%----------------------------------------------------------------------------------
% \todo	{ TODO MARGIN NOTE }
%
%puts a 'todo' comment into the margin in red
\definecolor{red}{rgb}{1,0,0}
\newcommand{\todo}[1]{\mnote{\textcolor{red}{ToDo: #1}}}

%----------------------------------------------------------------------------------
% \chapterquote	{ QUOTATION }
%				{ SOURCE }
%
%outputs a quote with its source, can be used as an introduction to chapters
\newcommand{\chapterquote}[2]{
\begin{quotation}
    \begin{flushright}
	\noindent\emph{``{#1}''\\[1.5ex]---{#2}}
    \end{flushright}
\end{quotation}
}

%----------------------------------------------------------------------------------
% \myDefBox	{ TERM }
%			{ DEFINITION }
%
%outputs a margin note and a colored box (width of the text column) containing a term and its definition
\newcommand{\myDefBox}[2]
{%
	\setlength{\fboxrule}{1mm}%
	\fcolorbox{orange_med}{orange_light}%
	{%
		\parbox{\myDefBoxWidth}{{\bfseries\scshape#1:}\\#2}%
	}%
	\mnote{Definition:\\\emph{#1}}
}

%----------------------------------------------------------------------------------
% \myBigDefBox	{ TERM }
%				{ DEFINITION }
%
%outputs a colored box (width of the page) containing a term and its definition
\newcommand{\myBigDefBox}[2]
{%
	\begin{figure}[h!]
	\setlength{\fboxrule}{1mm}%
	\checkoddpage%
	\ifcpoddpage%
		%nothing
	\else%
		\hspace{-\marginparsep}\hspace{-\marginparwidth}%
	\fi%
	\fcolorbox{orange_med}{orange_light}%
	{%
		\parbox{\myBigDefBoxWidth}{{\bfseries\scshape#1:}\\#2}%
	}%
	\end{figure}
}

%----------------------------------------------------------------------------------
% \myDownloadURL	{ TYPESET DOWNLOAD NAME }
%					{ PRISTINE VERSION OF FILENAME }
%					{ TYPESET VERSION OF FILENAME }
%
%outputs a colored box containing a download link
\newcommand{\myDownloadURL}[3]{%
\checkoddpage%
	\ifcpoddpage%
		%nothing
	\else%
		\hspace{-\marginparsep}\hspace{-\marginparwidth}%
	\fi%
\setlength{\fboxrule}{1mm}%
\fcolorbox{green_med}{green_light}{%
\begin{minipage}{\myBigDefBoxWidth}%
\begin{center}%
\myProjectURL{#1}{folder/#2}{folder/#3}%
\end{center}%
\end{minipage}%
}%
}

%----------------------------------------------------------------------------------
% \emptydoublepage
%
% Clear double page without any header or footer at end of chapters
\newcommand{\emptydoublepage}{\clearpage\thispagestyle{empty}\cleardoublepage}

%----------------------------------------------------------------------------------
% \pagebreak	[ SOME STRANGE LATEX VALUE ]
%
%pagebreaks for the final print version (last resort weapon against wrong pagebreaks by LaTeX)
\newcommand{\PB}[1][3]
{%
	\pagebreak[#1]%
}